\documentclass[12pt,french,a4paper]{memoir}
\usepackage[T1]{fontenc}
\usepackage[utf8]{inputenc}
\usepackage[frenchb]{babel}
\usepackage{amsmath}
\usepackage{amssymb}
\usepackage{mathtools}
\usepackage{proof}
\usepackage[french]{cleveref}
\usepackage{multirow}
\usepackage{tikz}

% Définition de la commande pour le signe = avec "déf" aussi dessus.
\newcommand\eqdef{\mathrel{\overset{\makebox[0pt]{\mbox{\normalfont\tiny\sffamily déf}}}{=}}}

\begin{document}

\title{Fondamentaux formels\\/\\Informatique théorique}
\author{Dr Romain Edelmann\\EPFL}
\date{2021}
\maketitle

\frontmatter

\chapter*{Remerciements}

Un grand merci à Patrick Rossi et à Murièle Jacquier pour la correction de nombreuses coquilles.

\newpage

\tableofcontents

\mainmatter

% !TEX root = ../cours.tex
\chapter{Calcul des propositions}

Le \og \textit{calcul des propositions} \fg{}, autrement appelé \og \textit{logique propositionnelle} \fg{}, est un système formel à la base de la logique mathématique. Le calcul des propositions nous intéresse pour plusieurs raisons:
\begin{enumerate}
\item
Le système nous permet de formaliser la notion de preuve et de discuter des techniques de preuve.
\item
Le système est à la base de la logique des prédicats et de la théorie des ensembles, dont nous auront besoin pour établir certains résultats liés à la cardinalité des ensembles de langages.
\item
Le problème de \og \textit{satisfiabilité} \fg{} d'une formule propositionnelle est fondamental dans l'étude des problèmes \textit{NP-complets}, concept que nous aborderons dans le cadre de la théorie de la complexité.
\end{enumerate}

Le calcul des propositions a pour objet d'étude des \og \textit{propositions} \fg{}.
Une proposition est un fait qui peut être vrai ou faux.
Les propositions sont formées de propositions atomiques (variables et constantes) combinées entre elles par des \og \textit{connecteurs} \fg{} (\textit{et}, \textit{ou}, \textit{seulement si}, etc.).

\paragraph{Example}

Pour ce premier exemple, on considère trois variables $p$, $d$ et $m$.
On donne à la variable $p$ la signification \og il pleut \fg{}, à la variable $d$ le sens \og le chat est dehors \fg{} et à la variable $m$ le sens \og le chat est mouillé \fg{}.

Ces variables peuvent être combinées à l'aide de différents connecteurs afin d'exprimer des propositions plus complexes.
Par exemple, alors que la proposition $m$ indique que le chat est mouillé, la proposition $\neg m$ indique que le chat n'est pas mouillé.
La proposition $m \iff p \wedge d$ a pour sens \og le chat est mouillé si et seulement si il pleut et le chat est dehors. \fg{}.
Bien que cette dernière proposition semble intuitivement vraie (étant donné notre interprétation de la réalité), elle n'est pas toujours vérifiée.
Par exemple, si le chat est dehors, qu'il pleut et que le chat n'est pas mouillé, alors la formule sera considérée comme fausse.
Comme on le découvrira par la suite, cette formule n'est pas \og \textit{valide} \fg{}: il existe au moins une interprétation qui rende la formule fausse.

A contrario, si on admet comme hypothèse le fait qu'un chat est mouillé si et seulement si il pleut et que le chat est dehors, et que l'on admette en plus le fait que le chat est mouillé, alors on pourra logiquement en conclure qu'il pleut.
Comme on pourra le montrer, la formule suivante, qui encode ce sens intuitif, est valide.
\[
(m \iff p \wedge d) \wedge m \implies p
\]
Peu importe les valeurs de vérité de $p$, $d$ et $m$, la formule sera toujours vraie.

Nous étudierons aussi un système de preuves formelles, la \og déduction naturelle \fg{}, afin de prouver de manière purement syntaxique de telles propositions.

\section{Syntaxe}

Pour commencer notre étude du calcul des propositions, étudions la \textit{syntaxe} des propositions, c'est-à-dire comment former des propositions (ou \og \textit{formules} \fg{}) bien formées. 

\subsection{Variables propositionnelles}

À la base des formules propositionnelles se trouvent les \og{} \textit{variables propositionnelles} \fg{}.
Chaque variable correspond à une proposition atomique qui peut être vraie ou fausse.
Par convention, on notera les variables propositionnelles avec des lettres minuscules tels que $x$, $y$ ou $z$, ou encore $p$, $d$ et $m$ comme dans l'exemple donné en introduction.

\subsection{Métavariables}

Lors de notre exposition du calcul des propositions, nous ferons aussi appel à des \og \textit{métavariables} \fg{}.
Les métavariables ne font pas partie à proprement parler du langage des propositions: il s'agit simplement d'un moyen de faire référence à des formules arbitraires dans la discussion de la logique propositionnelle.
On utilise des lettres majuscules telles que $A$, $B$ ou $C$ pour dénoter les métavariables.

\paragraph{Exemple} Dans la proposition $A \vee B$, on entend $A$ et $B$ comme des métavariables.
Comme elle contient des métavariables, la proposition $A \vee B$ n'est techniquement pas considérée comme une véritable formule, mais plutôt comme un schéma de formule.
En pratique, cette distinction importe généralement peu, et on parlera aussi de formules pour des schémas de formule.

\subsection{Constantes}

La logique propositionnelle contient deux constantes: \textit{vrai}, que l'on note $\top$ et \textit{faux}, que l'on note $\bot$.
La proposition $\top$ est toujours vraie, alors que la proposition $\bot$ est toujours fausse.

\paragraph{Remarque} Un moyen mnémotechnique pour se souvenir du sens de $\top$ et $\bot$ est de remarquer que $\top$ ressemble à un \texttt{T} comme dans le terme anglais \texttt{True}.

\subsection{Connecteurs}

En logique propositionnelle, on utilise des \textit{connecteurs} afin de composer des formules en formules plus complexes.

\subsubsection{Implication}

On dénote l'implication par une formule $A$ (l'\textit{impliquant}) d'une formule $B$ (l'\textit{impliqué}) par $A \implies B$,
que l'on prononce \og $A$ \textit{implique} $B$ \fg, \og \textit{si} $A$ \textit{alors} $B$ \fg, ou encore \og $A$ \textit{seulement si} $B$ \fg. 

La formule $A \implies B$ est vraie si et seulement si:
\begin{enumerate}
\item
$A$ est fausse, ou,
\item
$B$ est vraie.
\end{enumerate}

Intuitivement, si $A$ est vraie, alors il faut aussi que $B$ soit vraie pour que la formule $A \implies B$ soit vraie.
Dans le cas où l'impliquant $A$ est fausse, alors $A \implies B$ est vraie peu importe la valeur de $B$.
La formule $A \implies B$ ne peut être fausse que si $A$ est vraie et $B$ fausse.

On remarque que $A \implies (B \implies C)$ n'a pas la même signification que $(A \implies B) \implies C$.
On dit que le connecteur ${\implies}$ n'est pas \textit{associatif}.
Pour cette raison, il est important de clarifier ce que l'on entend lorsque l'on utilise plus d'une occurence du connecteur à la suite.
Par convention, on considère que ${\implies}$ associe sur la \textit{droite}.
Par exemple, on considèrera que la formule:
\[
A \implies B \implies C
\]
correspond à la formule:
\[
A \implies (B \implies C)
\]
et non pas à la formule:
\[
(A \implies B) \implies C
\]

\subsubsection{Conjonction}

La \textit{conjunction} de deux formules $A$ et $B$ est notée $A \wedge B$, que l'on prononce \og $A$ \textit{et} $B$ \fg{}.
La formule $A \wedge B$ est vraie si et seulement si $A$ et $B$ sont toutes deux vraies.

Par convention, on considère que le connecteur associe à droite:
\[
A \wedge B \wedge C
\]
correspond à:
\[
A \wedge (B \wedge C)
\]


\subsubsection{Disjonction}

La \textit{disjunction} de deux formules $A$ et $B$ est notée $A \vee B$, que l'on prononce \og $A$ \textit{ou} $B$ \fg{}.
La formule $A \vee B$ est fausse si et seulement si $A$ et $B$ sont toutes deux fausses.
La formule $A \vee B$ est vraie dans tous les autres cas, y compris lorsque $A$ et $B$ sont toutes deux vraies.

Par convention, on considère que le connecteur associe à droite:
\[
A \vee B \vee C
\]
correspond à:
\[
A \vee (B \vee C)
\]

\subsubsection{Négation}

On note la \textit{négation} d'une formule $A$ par $\neg A$, que l'on prononce \og \textit{non} $A$ \fg{}. Contrairement aux autres connecteurs présenté précédemment, on ne considérera pas ce connecteur comme étant \textit{primitif}, mais simplement un raccourci de notation pour exprimer la formule $A \implies \bot$.
\[
\neg A \eqdef A \implies \bot
\]

\subsubsection{Équivalence}

On dénote l'\textit{équivalence} de deux formules $A$ et $B$ par $A \iff B$, que l'on prononce \og $A$ \textit{si et seulement si} $B$ \fg{}. La formule $A \iff B$ est vraie si et seulement si les formules $A$ et $B$ ont la même valeur. Tout comme pour la négation, on ne considérera le connecteur ${\iff}$ que comme un simple raccourci de notation pour dénoter la double implication:
\[
A \iff B \eqdef (A \implies B) \wedge (B \implies A)
\]

On considère que le connecteur associe à droite.
Néanmoins, bien que tout à fait légal, on évitera de former des suites de formules connectées par ${\iff}$.
En effet, la notation:
\[
A \iff B \iff C
\]
pourrait laisser croire que l'on entend que $A$, $B$ et $C$ sont équivalentes, c'est à dire $(A \iff B) \wedge (B \iff C)$,
alors que le véritable sens en est éloigné:
\[
A \iff (B \iff C)
\] 
Ces deux interprétations diffèrent, par exemple lorsque $A$ est vraie et que $B$ et $C$ sont toutes deux fausses.

\subsection{Précédence des opérateurs}

L'usage de parenthèses pour délimiter les sous-formules est parfois verbeux.
Par convention on introduit un ordre de priorité entre les connecteurs afin de simplifier la notation.
On associe à chaque connecteur un niveau de \textit{précédence}. Les connecteurs de plus haute précédence sont appliqués avant ceux de plus basse précédence.

La précédence des connecteurs, par ordre décroissant, est donnée par:
\begin{gather*}
{\neg}\\
{\wedge}\\
{\vee}\\
{\implies}, {\iff}
\end{gather*}
Si bien que la formule:
\[
\neg A \implies B \vee \neg C \wedge D
\]
correspond à la formule:
\[
(\neg A) \implies (B \vee ((\neg C) \wedge D))
\]

\section{Sémantique}

Ayant étudié la syntaxe des propositions, nous pouvons maintenant nous attarder à leur interprétation \og \textit{sémantique} \fg{}, c'est-à-dire à leur sens.

\subsection{Interprétation}

Une \textit{interprétation} $\mathcal{I}$ d'une formule $A$ est une function qui associe à chaque variable propositionnelle de $A$ une valeur de vérité: Soit $1$ pour \og vrai \fg, soit $0$ pour \og faux \fg.
La valeur assignée par $\mathcal{I}$ à une variable $x$ est dénotée par $\mathcal{I}(x)$.

On peut aussi dénoter une interprétation sous forme d'ensemble d'associations entre variable et valeur.
Par exemple, $\{ x \mapsto 1, y \mapsto 0 \}$ dénote une interprétation $\mathcal{I}$ qui associe à $x$ la valeur $1$ et à $y$ la valeur $0$. Dans ce cas, on a que $\mathcal{I}(x)$ évalue à $1$ et $\mathcal{I}(y)$ à $0$.

Lorsque $A$ compte $n$ différentes variables propositionnelles, on dénombre $2^n$ interprétations différentes.

\subsection{Évaluation}

Étant donné une interprétation $\mathcal{I}$, une formule $A$ peut être évaluée à $1$ (\og vrai \fg) ou $0$ (\og faux \fg)
en suivant l'interprétation données aux variables propositionnelles par $\mathcal{I}$ et
en évaluant les connecteurs en suivant la table de la \cref{fig_tables_connecteurs}.
La constante $\top$ évalue toujours à $1$, alors que la constante $\bot$ évalue toujours à $0$.
On note $A^{\mathcal{I}}$ pour l'évaluation de la formule $A$ par l'interprétation $\mathcal{I}$.

\begin{figure}[h]

\begin{displaymath}
\begin{array}{cc|ccc}
A & B & A \implies B & A \wedge B & A \vee B\\
\hline
1 & 1 & 1 & 1 & 1\\
1 & 0 & 0 & 0 & 1\\
0 & 1 & 1 & 0 & 1\\
0 & 0 & 1 & 0 & 0
\end{array}
\end{displaymath}

\caption{Table de vérité pour les connecteurs primitifs. La table indique le résultat de l'évaluation des formules $A \implies B$, $A \wedge B$ et $A \vee B$ en fonction du résultat de l'évaluation des deux sous-formules $A$ et $B$.}
\label{fig_tables_connecteurs}
\end{figure}

\subsubsection{Modèle d'une formule}

Lorsque une interprétation $\mathcal{I}$ évalue une formule à $1$ (\og vrai \fg), on appelle $\mathcal{I}$ un \textit{modèle} de $A$.
Dans ce cas, on note:
\[
\mathcal{I} \vDash A
\]

Lorsque $A^{\mathcal{I}}$ évalue à $0$ (\og faux \fg), on note:
\[
\mathcal{I} \not\vDash A
\]

\subsubsection{Satisfiabilité}

Une formule $A$ est dite \og \textit{satisfiable} \fg{} s'il existe une interprétation $\mathcal{I}$ tel que $\mathcal{I} \vDash A$.
Une formule $A$ est dite \og \textit{insatisfiable} \fg{} dans le cas ou il n'existe aucun modèle de $A$. 

\subsubsection{Validité}

On dit que $A$ est \og \textit{valide} \fg{} lorsque $A^{\mathcal{I}}$ vaut $1$ pour toutes les interprétations $\mathcal{I}$ possibles – autrement dit, toutes les interprétations sont des modèles.
On dit aussi que la formule est une \og \textit{tautologie} \fg{}.
Dans ce cas, on écrit:
\[
\vDash A
\]

Une formule $A$ est \og \textit{invalide} \fg{} s'il existe une interprétation qui n'est pas un modèle de $A$.

\subsubsection{Relations en validité, invalidité, satisfiabilité et insatisfiabilité}

La~\cref{fig_carre} représente de façon schématique les relations entre les concepts de validité, invalidité, satisfiabilité et insatisfiabilité:
\begin{enumerate}
\item Deux énoncés sont \og \textit{contradictoires} \fg{} si et seulement si quand l'un est faux l'autre et vrai, et inversement. Les concepts de validité et d'invalidité sont contradictoires, ainsi que les concepts de satisfiabilité et insatisfiabilité. Par exemple, une formule est insatisfiable si et seulement si elle n'est pas satisfiable.
\item Deux énoncés sont \og \textit{contraires} \fg{} si et seulement si ils peuvent être simultanément faux mais pas simultanément vrais.
La validité et l'insatisfiabilité sont contraires. Une formule peut être à la fois non valide et non insatisfiable (c'est-à-dire satisfiable), mais une formule ne peut pas être valide et insatisfiable.
\item Deux énoncés sont \og \textit{subcontraires} \fg{} si et seulement si ils peuvent être simultanément vrais mais pas simultanément faux.
L'invalidité et la satisfiabilité sont subcontraires. Une formule peut être à la fois invalide et satisfiable, mais une formule ne peut pas être non invalide (c'est-à-dire valide) et non satisfiable.
\item Deux énoncés sont \og \textit{subalternes} \fg{} si un énoncé implique l'autre. La validité d'une formule implique sa satisfiabité, et l'insatisfiabilité d'une formule implique son invalidité. Validité et satifiabilité sont subalternes, tout comme insatisfiabilité et invalidité.
\end{enumerate}


\begin{figure}[h]
\begin{center}
\begin{tikzpicture}
\node[circle, draw, minimum size=2.5cm] (sat) at (2, 2) {Satisfiable};
\node[circle, draw, minimum size=2.5cm] (val) at (2, 8) {Valide};
\node[circle, draw, minimum size=2.5cm] (insat) at (8, 8) {Insatisfiable};
\node[circle, draw, minimum size=2.5cm] (inval) at (8, 2) {Invalide};
\draw[<->, ultra thick, shorten >=3pt, shorten <=3pt] (sat) -- (insat);
\draw[->, thick, shorten >=3pt] (val) -- (sat) node[midway,fill=white] {subalternes};
\draw[<->, shorten >=3pt, shorten <=3pt] (sat) -- (inval) node[below, midway] {subcontraires};
\draw[<->, shorten >=3pt, shorten <=3pt] (insat) -- (val) node[above, midway] {contraires};
\draw[->, thick, shorten >=3pt] (insat) -- (inval) node[midway,fill=white] {subalternes};
\draw[<->, ultra thick, shorten >=3pt, shorten <=3pt] (val) -- (inval);
\node[fill=white] at (5, 5) {contradictoires};
\end{tikzpicture}
\end{center}
\caption{Carré de vérité reliant satisfiabilité et validité.}
\label{fig_carre}
\end{figure}

\subsection{Table de vérité}

Une \og \textit{table de vérité} \fg{} est un tableau indexé par des variables ou métavariables.
Le tableau compte une ligne pour chaque combinaison de valeur de vérité de ces variables.
Les autres colones de la table indiquent le résultat de l'évaluation d'un certain nombre d'autres formules en fonction de la valeur des variables.

Pour les formules complexes, il est parfois plus aisé de rajouter des colonnes pour les sous-formules.

\begin{figure}[h]

\begin{displaymath}
\begin{array}{ccc|ccc}
x & y & z & x \wedge y \vee z & x \vee \neg x & x \iff \neg x\\
\hline
1 & 1 & 1 & 1 & 1 & 0\\
1 & 1 & 0 & 1 & 1 & 0\\
1 & 0 & 1 & 1 & 1 & 0\\
1 & 0 & 0 & 0 & 1 & 0\\
0 & 1 & 1 & 1 & 1 & 0\\
0 & 1 & 0 & 0 & 1 & 0\\
0 & 0 & 1 & 1 & 1 & 0\\
0 & 0 & 0 & 0 & 1 & 0
\end{array}
\end{displaymath}

\caption{Table de vérité indexée par les variables propositionnelles $x$, $y$ et $z$.
Le résultat de l'interprétation des formules $x \wedge y \vee z$, $x \vee \neg x$ et $x \iff \neg x$ en fonction de la valeur des variables $x$, $y$ et $z$ est indiqué dans les trois dernières colonnes.}
\label{fig_table_verite_ex}

\end{figure}

Lorsqu'une colonne ne contient que des $1$, la proposition associée est valide (aussi dit une tautologie).
Si une colonne contient au moins un $1$, alors la proposition associée est satisfiable.
Une colonne qui contient au moins un $0$ indique une proposition invalide.
Finalement, lorsqu'une colonne ne contient que des $0$, la proposition est insatisfiable. 

\paragraph{Exemple} Après examen de la table de vérité de la~\cref{fig_table_verite_ex}, on peut conclure que:
\begin{enumerate}
\item
La formule $x \wedge y \vee z$ est à la fois satisfiable et invalide.
En effet, la formule est satisfiable car il existe un modèle de la formule (par exemple $\{ x \mapsto 0, y \mapsto 1, z \mapsto 1 \}$, à la ligne 5). 
De plus, la formule est invalide car il existe une interprétation qui évalue la formule à $0$ (par exemple $\{ x \mapsto 0, y \mapsto 0, z \mapsto 0 \}$, à la ligne 8).
\item
La formule $x \vee \neg x$ est une tautologie, que l'on note $\vDash x \vee \neg x$. En effet toutes les entrées de la colonne correspondante sont à $1$.
La formule est aussi par conséquent satisfiable.
\item
La formule $x \iff \neg x$ est insatisfiable: toutes les entrées de la colonne correspondante sont à $0$.
La formule est aussi par conséquent invalide.
\end{enumerate}

\section{Déduction naturelle}

La \og \textit{déduction naturelle} \fg{} est un système de preuve pour la logique propositionnelle.
Contrairement à la méthode des tables de vérité, qui est \textit{sémantique}, la \textit{déduction naturelle} est une méthode \textit{syntaxique}.
Il ne s'agit plus d'évaluer des formules, mais de manipuler leur structure syntaxique à l'aide de régles afin d'essayer d'en prouver leur validité.

\subsection{Règle d'inférence}

La déduction naturelle est un système de preuve à base de \og \textit{règles d'inférence} \fg.
Chaque règle est composée d'une \og \textit{conclusion} \fg{}, d'un nombre arbitraire de \og \textit{prémisses} \fg{}, ainsi que d'un nom.

\[
\infer[(\text{nom de la règle})]{\text{conclusion}}{\text{prémisse 1} & \text{prémisse 2} & \dots & \text{prémisse $n$}}
\]

La conclusion et les prémisses d'une règle sont des schémas de formules qui contiennent généralement des métavariables.
Les métavariables qui apparaissent dans la conclusion et les prémisses peuvent être remplacées par n'importe quelle formule, pour autant que ce remplacement se fasse de façon unifiée sur la conclusion et toutes les prémisses dans le schéma.

Intuitivement, une règle d'inférence permet de conclure la conclusion étant donné une preuve de chaque prémisse.

\subsection{Arbre de dérivation}

En déduction naturelle, la preuve formelle d'une formule $A$ prend la forme d'un \og \textit{arbre de dérivation} \fg.
À la racine de l'arbre est une règle d'inférence instanciée avec la formule $A$ pour conclusion.
Chaque prémisses de cette règle doit elle aussi être prouvée, et selon le même procédé.
L'arbre de dérivation a pour feuilles des arbres sans prémisses, que l'on appelle \og \textit{axiomes} \fg{}.
Aussi, comme expliqué plus loin, les feuilles peuvent prendre la forme d'\og \textit{hypothèses déchargées} \fg{}, indiquée par une paire de crochets (\og [ \fg{} et \og ] \fg{}).

\paragraph{Exemple} La~\cref{fig_arbre_derivation} présente un arbre de dérivation pour la formule schématique $A \vee \bot \implies A \wedge \top$. L'arbre de dérivation fait appel à de nombreuses règles d'inférences qui vont être introduites sous peu.

\begin{figure}[h]
\[
\infer[({\Rightarrow{}}I)]{A \vee \bot \implies A \wedge \top}{%
\infer[({\vee}E)]{A \wedge \top}{[A \vee \bot] &%
\infer[({\Rightarrow{}}I)]{A \implies A \wedge \top}{%
\infer[({\wedge{}}I)]{A \wedge \top}{[A] & \infer[({\top}I)]{\top}{}}} &%
\infer[({\Rightarrow{}}I)]{\bot \implies A \wedge \top}{%
\infer[({\bot}E)]{A \wedge \top}{[\bot]}}}}
\]
\caption{Arbre de dérivation de la formule $A \vee \bot \implies A \wedge \top$. L'arbre de dérivation constitue une preuve du théorème $\vdash A \vee \bot \implies A \wedge \top$.}
\label{fig_arbre_derivation}
\end{figure}

\subsection{Théorème}

On appelle un \og \textit{théorème} \fg{} une formule (ou une formule schématique) qui a été prouvée via un arbre de dérivation.
Pour une formule $A$ (ou un schéma de formule), on note $\vdash A$ le fait que $A$ est un théorème.

\subsection{Règles de la déduction naturelle}

Les règles d'inférence de la déduction naturelle sont données ci-après.
Elle sont classifiées en deux catégories: règles d'introduction et règles d'élimination.
Les règles d'introduction introduisent une construction dans leur conclusion,
alors que les règles d'élimination permettent de faire disparaître dans la conclusion une construction apparaissant dans une prémisse.
Les règles sont nommées en fonction de la construction et d'un $I$ ou d'un $E$ suivant s'il s'agit d'une règle d'introduction ou d'élimination.

\subsubsection{Introduction de $\top$}

La règle $({\top}I)$ introduit la constante $\top$.
La règle n'a aucune prémisse, il s'agit d'un \textit{axiome}.
La règle stipule que l'on peut prouver $\top$ sans aucune prémisse.

\[
\infer[({\top}I)]{\top}{}
\]

\subsubsection{Introduction de $\implies$}

La règle $({\Rightarrow{}}I)$ permet l'introduction d'une implication $A \implies B$ étant donné une preuve de la formule impliquée $B$.
La preuve de la formule impliquée $B$ peut utiliser l'impliquant $A$ en tant que prémisse sans aucune autre preuve, et ce autant de fois que désiré.
Les preuves de l'impliquant $B$ dans l'arbre de dérivation de sont \og \textit{déchargées} \fg{} par $({\Rightarrow{}}I)$, ce que l'on note par une paire de crochets (\og [ \fg{} et \og ] \fg{}).

\[
\infer[({\Rightarrow{}}I)]{A \implies B}{\infer*{B}{[A]}}
\]

À noter que l'hypothèse est considérée déchargée uniquement une fois la règle $({\Rightarrow{}}I)$ correspondante appliquée.

\subsubsection{Élimination de $\implies$}

La règle $({\Rightarrow{}}E)$ permet de prouver l'impliquant $B$ d'une implication $A \implies B$ étant donné la preuve de l'implication et de l'impliquant $A$. Cette règle est aussi connue sous le nom de \og \textit{modus ponens} \fg{}.

\[
\infer[({\Rightarrow{}}E)]{B}{A \implies B & A}
\]

\subsubsection{Introduction de $\wedge$}

La règle $({\wedge}I)$ permet l'introduction d'une conjonction $A \wedge B$ étant donné une preuve de $A$ et une preuve de $B$.

\[
\infer[({\wedge{}}I)]{A \wedge B}{A & B}
\]

\subsubsection{Élimination de $\wedge$}

Étant donné une preuve de $A \wedge B$, il est possible d'obtenir à la fois une preuve de $A$ et une preuve de $B$.
Ce possibilité est offerte par la présence de deux règles d'inférences: $({\wedge{}}E\text{ gauche})$ et $({\wedge{}}E\text{ droite})$.

\[
\infer[({\wedge{}}E\text{ gauche})]{A}{A \wedge B}
\]

\[
\infer[({\wedge{}}E\text{ droite})]{B}{A \wedge B}
\]

\subsubsection{Introduction de $\vee$}

Le connecteur ${\vee}$ possède deux règles d'introduction, une qui permet de conclure $A \vee B$ depuis la sous-formule de gauche (en l'occurence $A$), et une depuis la sous-formule de droite (ici $B$).

\[
\infer[({\vee}I\text{ gauche})]{A \vee B}{A}
\]

\[
\infer[({\vee}I\text{ droite})]{A \vee B}{B}
\]

\subsubsection{Élimination de $\vee$}

Le connecteur ${\vee}$ dispose d'une seule règle d'élimination ${\vee}E)$.
La règle a trois prémisses:
\begin{enumerate}
\item
Une disjonction $A \vee B$,
\item
Une implication $A \implies C$ pour une certaine formule $C$, et
\item
Une implication $B \implies C$ pour la même formule $C$.
\end{enumerate}
La conclusion de la règle est la formule $C$.

\[
\infer[({\vee}E)]{C}{A \vee B & A \implies C & B \implies C}
\]

La règle d'élimination de ${\vee}$ permet de faire une analyse de cas.

%\infer[({\bot}I)]{\bot}{A & \neg A}

\subsubsection{Élimination de $\bot$}

Si la constante $\bot$ est en prémisse, elle peut-être éliminée et remplacée par n'importe quelle proposition.
Ce fait est encodé dans la règle $({\bot}E)$.

\[
\infer[({\bot}E)]{A}{\bot}
\]

\subsubsection{Élimination de la double négation}

Finalement, la règle d'inférence $({\neg}{\neg}E)$ permet de déduire une proposition de son \textit{irréfutabilité}.

\[
\infer[({\neg}{\neg}E)]{A}{\neg \neg A}
\]

\section{Logique classique et logique intuitionniste}

Les règles présentées dans cette section forment la version \og \textit{classique} \fg{} de la déduction naturelle.
Certains logiciens, du mouvement \og \textit{intuitionniste} \fg{}, réfutent l'utilisation de l'élimination de la double négation $({\neg}{\neg}E)$.

La raison pour laquelle la règle n'est pas acceptée est que la règle n'est pas \og \textit{constructive} \fg{}:
Pour un logicien intuitionniste, pour qu'une proposition soit démontrable, il faut qu'elle puisse être effectivement construite.
Le simple fait qu'une preuve de l'irréfutabilité d'une proposition existe n'implique pas, pour un intuitionniste, l'existence d'une preuve de la proposition. Ce n'est pas parce que l'on démontre que la négation d'une proposition est impossible que l'on a démontré que la proposition est vraie.

\subsection{Règles équivalentes à $({\neg}{\neg}E)$}

De nombreuses règles de déduction sont équivalentes à $({\neg}{\neg}E)$. Par exemple, les deux règles suivantes sont équivalentes à $({\neg}{\neg}E)$:

\begin{align*}
\infer[(\text{TND})]{A \vee \neg A}{} &&
\infer[(\text{RAA})]{A}{\infer*{\bot}{[\neg A]}}
\end{align*}

De ce fait, en logique intuitionniste, il n'est pas possible de faire usage du principe du \textit{tiers exclu} (TDN, pour \textit{tertium non datur}) ou du raisonnement par l'absurde (RAA, pour \textit{reductio ad absurdum}). À noter que l'absence règle RAA en logique intuitionniste n'empêche pas de prouver une négation par \og \textit{réfutation} \fg{}:

\[
\infer[(\text{Réfutation})]{\neg A}{\infer*{\bot}{[A]}}
\]

La règle de réfutation est en faite une version spécialisée de la règle $({\Rightarrow}I)$, ce qui est d'avantage évident lorsque l'on se remémore la définition de $\neg A$ comme l'implication $A \implies B$.

\subsection{Correspondance de Curry-Howard}

Ce qui fait l'attrait de la logique intuitionniste pour certains logiciens et informaticiens est sa correspondance avec un modèle de calcul appelé le \og \textit{lambda-calcul simplement typé} \fg{}. Le $\textit{lambda-calcul simplement typé}$ est un langage de programmation fonctionnel théorique à la base de nombreux langages de programmation actuels tels que Haskell ou ML.

La correspondance de Curry-Howard énonce une correspondance entre les propositions et leurs preuves en logique propositionnelle intuitionniste et les types et leurs termes en lambda-calcul simplement typé. Le tableau de la~\cref{fig_curry_howard} présente certaines de ces correspondances.

\begin{figure}[h]
\begin{center}
\begin{tabular}{|lc|lc|}
\hline
\multicolumn{2}{|p{6cm}|}{\textbf{Logique propositionnelle\newline{}intuitionniste}} & \multicolumn{2}{p{6cm}|}{\textbf{Lambda-calcul\newline{}simplement typé}} \\
\hline
\multicolumn{2}{|c|}{Proposition} & \multicolumn{2}{c|}{Type} \\
Implication & $A \implies B$ & Type de fonction & \texttt{$A$ => $B$} \\
Conjunction & $A \wedge B$ & Type de paire & \texttt{($A$, $B$)} \\
Disjunction & $A \vee B$ & Type de somme & \texttt{Either[$A$, $B$]} \\
Vrai & $\top$ & Type unitaire & \texttt{()} \\
Faux & $\bot$ & Type vide & \texttt{Void} \\
\hline
\multicolumn{2}{|c|}{Règle} & \multicolumn{2}{c|}{Construction} \\
Intro. de l'impli. & $({\Rightarrow}I)$ & Abstraction & $\lambda x. e$\\
Modus ponens & $({\Rightarrow}E)$ & Application & \texttt{$f$($x$)} \\
\multicolumn{2}{|c|}{\vdots} & \multicolumn{2}{c|}{\vdots} \\
\hline
\end{tabular}
\end{center}
\caption{Table de correspondance entre propositions de la logique propositionnelle intuitionniste et lambda-calcul simplement typé.}
\label{fig_curry_howard}
\end{figure}

\section{Complétude et cohérence}

Dans ce chapitre, nous avons vu deux façon d'établir la véracité d'une proposition.
Une approche \textit{sémantique} via l'interprétation et une approche \textit{syntaxique} via la déduction naturelle.

En logique propositionnelle classique, les deux approches sont équivalentes: Toutes les tautologies sont des théorèmes, et tous les théorèmes sont des tautologies.
À contrario, en logique propositionnelle intuitionniste, il y a certaines tautologies qui ne sont pas des théorèmes ($\vDash A \vee \neg A$ par exemple). Cependant, chaque théorème reste une tautologie.


% !TEX root = ../cours.tex

\chapter{Calcul des prédicats}

Le \og \textit{calcul des prédicats} \fg{}, parfois appelé \og logique du premier-ordre \fg{}, est une extension du calcul des propositions afin de manipuler non seulement des propositions, mais aussi des \textit{objets} d'un \textit{univers}.
Une brève introduction est donnée dans ce chapitre dans le but de vous familiariser avec la syntaxe et les concepts de ce système formel qui seront utilisés tout au long du cours.
Le sujet est vaste (et extrêmement intéressant!), et nous allons simplement l'aborder de manière superficielle dans le cadre ce cours.

Le calcul des prédicats est une extension au calcul des propositions et hérite de toutes ses constructions syntaxiques.
Le calcul des prédicats compte donc les propositions constantes ($\top$ et $\bot$), les connecteurs propositionnels (tels que $\implies$, $\wedge$, $\vee$, $\neg$, $\iff$), ainsi que les variables propositionnelles (sous la forme de prédicats d'arité $0$).
En plus de cela, le calcul des prédicats incorpore des constructions syntaxiques en rapport aux \textit{objets}.

\section{Univers}

Lorsque l'on discute le calcul des prédicats, l'on a généralement en tête un \og \textit{univers} \fg{} pour les objets.
L'univers que l'on considère peut être, par exemple, les nombres entiers, les ensembles, ou encore les participants au programme GymInf.
On requiert généralement de l'univers qu'il contienne au moins un objet.

\section{Syntaxe}

Contrairement à la logique propositionnelle, qui ne comptait qu'une seule catégorie syntaxique (celle des propositions, ou formules), la logique de premier-ordre compte une catégorie syntaxique distincte additionnelle: les \og \textit{termes} \fg{}.
Alors que l'évaluation d'une formule résulte en $1$ (\og \textit{vrai} \fg{}) ou $0$ (\og \textit{faux} \fg{}), l'évaluation d'un terme résulte en un objet de l'univers.

\subsection{Metavariables}

Comme dans le précédent chapitre, nous allons utiliser des métavariables dans la présentation du calcul des prédicats.
Ces métavariables ne font pas partie du formalisme et sont simplement un moyen de parler de termes et formules arbitraires.

Dans ce chapitre, nous utiliserons $A$, $B$, $C$ pour les métavariables associées à des formules et $X$, $Y$, $Z$ pour les métavariables associées à des termes.

\subsection{Objets constants}

Un objet constant réfère à un objet de l'univers. Par convention, on note les objets constant par un nom en minuscule.

\subsection{Fonctions entres objets}

Les fonctions prennent en arguments uniquement des termes, et retournent un terme.
On note $f(X, Y)$ l'application d'une fonction $f$ à deux expressions $X$ et $Y$.
Le nombres d'arguments qu'accepte une fonction est appelé son '\og arité \fg{}.
On considère que les fonctions ont une arité fixe.

\subsection{Prédicats sur les objets}

Les prédicats prennent en arguments uniquement des objets, et retournent une proposition.
Comme les fonctions, les prédicats ont aussi une arité qui dénote le nombre d'argument que le prédicat accepte.

\subsection{Égalité entre objets}

On incorpore un prédicat binaire (d'arité 2) distinct, que l'on note ${=}$, pour l'égalité.
Le prédicat jouit de plusieurs règles d'inférence:

\[
\infer[({=}\text{-trans})]{X = Z}{X = Y & Y = Z}
\]

\[
\infer[({=}\text{-sym})]{X = X}{}
\]

\[
\infer[({=}\text{-com})]{Y = X}{X = Y}
\]

\[
\infer[({=}\text{-fun})]{F(X) = F(Y)}{X = Y}
\]

\[
\infer[({=}\text{-pre})]{P(Y)}{X = Y & P(X)}
\]

\subsection{Quantificateurs \& variables}

Le calcul des prédicats incorpore des \og \textit{quantificateurs} \fg{} afin de pouvoir exprimer des propositions tels que \og \textit{Pour tout $x$, $x^2 \geq 0$.} \fg{} ou encore \og \textit{Il existe un homme qui rase tous les hommes qui ne se rasent pas eux-mêmes.} \fg{}.

\subsubsection{Variables}

Le calcul des prédicats incorpore des \og \textit{variables} \fg{} dont le domaine est les objets de l'univers considéré.
Contrairement aux métavariables, une variable est un terme du formalisme.
On dit d'une variable qu'elle est \og \textit{liée} \fg{} si elle se trouve dans une formule quantifiée indexée par le même nom.
Une variable est dite \og \textit{libre} \fg{} si elle n'est pas liée.

\subsubsection{Substitution}

Une opération de \og \textit{substitution} \fg{} permet de remplacer les variables libres d'un certain nom dans un terme ou une formule par un autre terme.

\begin{align*}
x[X / x] &\eqdef X \\
y[X / x] &\eqdef y \hspace{2em} (x \text{ est différent de } y)\\
f(Y_1, \dots, Y_n)[X / x] &\eqdef f(Y_1[X / x], \dots, Y_n[X / x])\\
P(Y_1, \dots, Y_n)[X / x] &\eqdef P(Y_1[X / x], \dots, Y_n[X / x])\\
(\forall x. A)[X / x] &\eqdef \forall x. A\\
(\forall y. A)[X / x] &\eqdef \forall y. A[X / x] \hspace{2em} (x \text{ est différent de } y \text{ et $y$ n'est pas libre dans $X$})\\
(\exists x. A)[X / x] &\eqdef \exists x. A\\
(\exists y. A)[X / x] &\eqdef \exists y. A[X / x] \hspace{2em} (x \text{ est différent de } y \text{ et $y$ n'est pas libre dans $X$})
\end{align*}

Dans le but d'éviter un effet de \textit{capture} de variable non-désiré, on restreint les règles de substitution liées aux quantificateurs:
La variable introduite par le quantificateur ne peut pas apparaître libre dans le terme de substitution.
Pour contourner cette restriction, il est possible de renommer les variables problématiques.

\subsubsection{Quantificateur d'universalité}

Un note $\forall x. A$ le proposition qui postule que \og \textit{Pour tout objet $x$, la proposition $A$ est vraie} \fg{}.
À noter que la proposition $A$ dans ce cas fait généralement référence à la variable $x$, mais ce n'est pas une obligation.
Le quantificateur d'universalité $\forall$ est lié aux deux règles d'inférences suivantes:

\[
\infer[({\forall}I)]{\forall x. A}{A & (*)}
\]

\[
\infer[({\forall}E)]{A[X / x]}{\forall x. A}
\]

La règle $({\forall}I)$ est annotée d'une $(*)$: Elle est en effet soumise à la condition que la variable $x$ n'est pas libre dans une hypothèse non-déchargée au moment de l'application de la règle.

\subsubsection{Quantificateur d'existence}

Un note $\exists x. A$ le proposition qui postule que \og \textit{Il existe un objet $x$ tel que la proposition $A$ est vraie} \fg{}.
Tout comme pour le quantificateur d'universalité, la proposition $A$ fait généralement référence à la variable $x$, sans que ce soit une obligation.
Le quantificateur d'existence $\exists$ est lié aux deux règles d'inférences suivantes:

\[
\infer[({\exists}I)]{\exists x. A}{A[X / x]}
\]

\[
\infer[({\exists}E)]{B}{\exists x. A & \infer*{B}{[A]} & (*)}
\]

La règle $({\exists}E)$ est annotée d'une $(*)$: Elle est en effet soumise à la condition que la variable $x$ n'est pas libre dans une hypothèse non-déchargée au moment de l'application de la règle, et que la variable n'est pas non plus libre dans la formule $B$.

\section{Théories}

Une \og \textit{théorie} \fg{} $T$ de logique du premier-ordre est une structure mathématique qui spécifie les noms d'un nombre fini de constantes, les noms d'un nombre fini de fonctions et leur arité, ainsi que les noms d'un nombre fini de prédicats et leur arité.
Une théorie contient aussi généralement un ensemble d'axiomes faisant référence aux constantes, functions et prédicats de la théorie.

On note $T \vdash A$ le fait qu'il existe une preuve (pouvant faire appel aux axiomes de $T$) de la formule $A$ en deduction naturelle dans la théorie T.

\section{Interprétation sémantique}

Étant donné une théorie $T$ de logique du premier-ordre, une \og \textit{interprétation} \fg{} $\mathcal{I}$ est une structure qui spécifie l'univers $U$ et qui associe:
\begin{enumerate}
\item à chaque constante de la théorie une valeur $v \in U$,
\item à chaque fonction d'arité $n$ de la théorie une fonction $f : U^n \to U$, et
\item à chaque prédicat d'arité $n$ de la théorie une relation $R \subseteq U^n$.
\end{enumerate}

\subsection{Évaluation}

Étant donné une interprétation $\mathcal{I}$ et une fonction $\mu$ qui associe à chaque variable libre une valeur, il est possible d'évaluer une formule de la théorie à $0$ ou $1$ en procédant de manière récursive.
La plupart des cas sont triviaux. Pour l'évaluation des quantificateurs, la fonction $\mu$ est modifiée dans les appels récursifs pour tenir compte de la nouvelle variable libre.

Lorsque une formule $A$ évalue à 1 (\og vrai \fg{}) étant donné une interprétation $\mathcal{I}$ et une fonction $\mu$, on note:
\[
\mathcal{I}, \mu \vDash A
\]
Pour les formules sans variables libres, on note simplement:
\[
\mathcal{I} \vDash A
\]

Si on formule $A$ est vraie pour toutes les interprétations $\mathcal{I}$ d'une théorie $T$, alors on note:

\[
T \vDash A
\]





% !TEX root = ../cours.tex
\chapter{Ensembles, problèmes et langages}

Dans ce chapitre, nous allons formaliser la notion de \og \textit{problèmes} \fg{} et leur encodage sous forme de \og \textit{langages} \fg{}.
Nous allons établir un premier résultat concernant l'\og \textit{indécidabilité} \fg{} de certains problèmes grâce à un raisonnement ensembliste lié à la cardinalité.
Pour cette raison, nous commencerons ce chapitre par un bref rappel de théorie des ensembles.

\section{Ensembles et appartenance}

Un ensemble est une collection d'objets.
On note $a \in x$ le prédicat binaire stipulant l'appartenance de l'objet $a$ à l'ensemble $x$.
On considère deux ensembles comme égaux si et seulement si ils contiennent les mêmes objets.
Ce fait est exprimé formellement par l'axiome suivant:

\[
\forall x. \forall y.\ x = y \iff (\forall a. a \in x \iff a \in y)
\]

\subsection{Ensemble vide}

On note l'ensemble vide $\emptyset$.
On admet:

\[
\forall x.\ x \not\in \emptyset
\]

\subsection{Notation d'ensemble}

Étant donné des éléments $x_1$, $x_2$ \dots $x_n$, on note l'ensemble qui contient ces éléments, et uniquement ces éléments, par:
\[
\{ x_1, x_2, \dots, x_n\}
\]

\section{Opérations ensemblistes}

\subsection{Union}

On note l'union de deux ensembles $X$ et $Y$ par $X \cup Y$.
Un objet appartient à l'union de deux ensembles si et seulement si il appartient à l'un ou à l'autre des deux.
Ce fait est exprimé par l'axiome:

\[
\forall x. \forall y. \forall a.\ a \in x \cup y \iff a \in x \vee a \in y
\]

\subsection{Intersection}

On note l'intersection de deux ensembles $X$ et $Y$ par $X \cap Y$.
Un objet appartient à l'intersection de deux ensembles si et seulement si il appartient aux deux ensembles.
Ce fait est exprimé par l'axiome:

\[
\forall x. \forall y. \forall a.\ a \in x \cap y \iff a \in x \wedge a \in y
\]

\subsection{Produit cartésien}

On note la paire de $a$ et $b$ par $(a, b)$.
On note le produit cartésien de deux ensembles $X$ et $Y$ par $X \times Y$.
Les éléments de $X \times Y$ sont les paires d'éléments de $X$ et de $Y$:

\[
\forall x. \forall y. \forall p.\ p \in x \times y \iff (\exists a. \exists b.\ a \in x \wedge b \in y \wedge p = (a, b))
\]

\subsection{Définition d'un ensemble en compréhension}

Pour un ensemble $X$, on dénote l'ensemble de tous les éléments $x$ de $X$ qui satisfont une proposition $A$ (qui peut faire référence à $x$) comme:
\[
\{\ x \in X\ |\ A\ \}
\]

On admet le schéma d'axiomes suivant:
\[
\forall s. \forall a.\ a \in \{ x \in s\ |\ A\ \} \iff (a \in s \wedge A[a / x])
\]

\subsection{Sous-ensemble}

On utilise le prédicat $X \subseteq Y$ pour indiquer que l'ensemble $X$ est un sous-ensemble de $Y$.
Un ensemble $X$ est un sous-ensemble d'un ensemble $Y$ si et seulement si tout élément de $X$ est un élément de $Y$. 
Ce fait est exprimé par l'axiome:

\[
\forall x. \forall y.\ x \subseteq y \iff (\forall a. a \in x \implies a \in y)
\]

\subsection{Super-ensemble}

Le super-ensemble d'un ensemble $X$ est l'ensemble de tous les sous-ensembles de $X$.
Pour un ensemble $X$, on note son super-ensemble $\mathcal{P}(X)$ (pour \textit{powerset}).
On a:

\[
\forall x. \forall y.\ x \in \mathcal{P}(y) \iff x \subseteq y
\]

\section{Les nombres naturels}

On note $\mathbb{N}$ l'ensemble des nombres naturels \{$0$, $1$, $2$, \dots\}.
On admet les opérations usuelles telles qu'addition, multiplication, puissance, plus petit ou égal, et ainsi de suite.

\section{Fonctions}

Étant donnés deux ensembles $X$ (le \textit{domaine}) et $Y$ (le \textit{codomaine}), on appelle un sous-ensemble $f$ de $X \times Y$ une \og \textit{fonction} \fg{} si et seulement si $f$ obéit aux contraintes suivantes:
\begin{enumerate}
\item Pour tout élément $x$ de $X$, il existe élément $y$ de $Y$ tel que $(x, y) \in f$.
\item Pour tout élément $x$ de $X$, s'il existe deux éléments $y_1$ et $y_2$ tels que $(x, y_1) \in f$ et $(x, y_2) \in f$, alors nécessairement $y_1 = y_2$.
\end{enumerate}
Étant donné une fonction $f$ de domaine $X$ et de codomaine $Y$ et un élément $x \in X$, on note $f(x)$ l'élément unique de $Y$ associé par $f$ à $x$.
On note aussi $f : A \to B$ pour signifier que $f$ est une fonction de domaine $A$ et de codomaine $B$.

\subsection{Injections}

On appelle une \og \textit{injection} \fg{} toute fonction qui associe au plus un élément du domaine à tout élément du codomaine.

\subsection{Surjections}

On appelle une \og \textit{surjection} \fg{} toute fonction qui associe au moins un élément du domaine à chaque élément du codomaine.

\subsection{Bijections}

On appelle une \og \textit{bijection} \fg{} toute fonction qui associe exactement un élément du domaine à chaque élément du codomaine.
Une fonction est une bijection si et seulement si elle est à la fois une injection et une surjection.

\section{Séquence}

On appelle une \og \textit{séquence} \fg{} une fonction $s$ de domaine $\{\ 1, 2, \dots, n\ \}$ pour un certain nombre naturel $n$.
On appelle $n$ la taille de la séquence $s$, que l'on note $|s|$.

\section{Cardinalité}

On appelle la cardinalité d'un ensemble sa taille. Pour un ensemble $X$, on dénote sa cardinalité par $|X|$.

La cardinalité d'un ensemble fini est un nombre naturel qui est égal au nombre d'éléments.
Pour les ensembles infinis, la réponse est plus complexe, et implique la notion de \textit{cardinaux}.
L'étude des cardinaux est un sujet vaste en soit, et nous n'auront pas le temps de l'explorer en détails.
Cependant, sans introduire la notion de cardinaux, nous ferons usage des résultats suivants:
\begin{enumerate}
\item
On considère que deux ensembles ont la même cardinalité si et seulement si il existe une bijection entre les deux ensembles.
\item
Aussi, on considère que la cardinalité d'un ensemble est plus petite ou égale à la cardinalité d'un autre ensemble si et seulement si il est possible d'établir une injection entre le premier ensemble et le second.
\end{enumerate}

\subsection{Ensemble dénombrable}

On appelle la cardinalité de l'ensemble des nombres naturels $\aleph_0$:
\[
|\mathbb{N}| = \aleph_0
\]

Tout ensemble dont la cardinalité est égal à $\aleph_0$ est appelé \og \textit{dénombrable} \fg{}.
Un ensemble dénombrable peut être mis en bijection avec les nombres naturels.

\section{Problèmes \& programmes}

Un \og \textit{problème} \fg{} est une question qui porte sur un ensemble d'éléments.
On appelle une \og \textit{instance} \fg{} un problème posé sur un élément en particulier.
Chaque instance à une réponse.
Dans ce cours, nous nous intéresserons particulièrement aux problèmes binaires, dans lesquels la réponse à chaque instance est soit \og \textit{oui} \fg{}, soit \og \textit{non} \fg{}.
On appelle une instance dont la réponse est \og oui \fg{} une \og \textit{instance positive} \fg{} du problème et une instance dont la réponse est \og non \fg{} une \og \textit{instance négative} \fg{} du problème.

Un \og \textit{programme} \fg{} est une procédure de résolution de problème qui a les particularités suivantes:
\begin{enumerate}
\item
Elle consiste en un nombre fini d'instructions, toutes de taille finie.
\item
Pour chaque instance du problème, la procédure termine et retourne le résultat correct.
\item
Les instructions sont non-ambiguës et ne requièrent aucune ingéniosité. La procédure peut être suivie par un être humain suivant les instructions à la lettre (en faisant abstraction de sa capacité de mémoire et de sa patience). 
\end{enumerate}
On parle aussi de \og procédure effective \fg{} ou d'\og algorithme \fg{} pour faire référence à un programme.

\section{Encodage}

Au vu de la taille nécessairement finie d'un programme, il faudra nécessairement \textit{encoder} les instances du problème, c'est-à-dire leur donner une représentation sous forme de suite finie de symboles tirés d'un \og \textit{alphabet} \fg{} lui aussi fini.
Sans cet encodage, il serait impossible au programme de pouvoir traiter efficacement toutes les instances.
À noter que, comme on pourra le montrer tout soudain, cet encodage n'est pas toujours possible:
Il y a \textit{trop} d'instances à encoder.
Par exemple, il est impossible d'encoder tous les nombres réels de façon unique.

\subsection{Alphabet}

On appelle un \og \textit{alphabet} \fg{} un ensemble fini et non-vide de \og \textit{symboles} \fg{}.
Chaque symbole est distinct des autres symboles de l'alphabet.
Bien que généralement implicite, on dénotera au besoin l'alphabet considéré par $\Sigma$.

\subsection{Mots}

On appelle une séquence de symboles tirés d'un alphabet $\Sigma$ un \og \textit{mot} \fg{}.
On dénote par $\Sigma^*$ l'ensemble de tous les mots basés sur l'alphabet $\Sigma$.
On note aussi $\Sigma^n$ l'ensemble de tous les mots de taille $n$.

Pour un mot $x$, on note $|x|$ le nombre naturel qui correspond à la taille du mot, c'est-à-dire à son nombre de symboles.
On note $x(1)$ le premier élément du mot, $x(2)$ le deuxième, et ainsi de suite jusqu'à $x(n)$, où $n = |x|$.

Deux mots sont égaux s'ils ont la même taille et qu'ils associent à chaque index le même symbole.

\subsubsection{Notation}

On note un mot formé d'au moins un symbole simplement en écrivant les symboles l'un à la suite de l'autre.
Par exemple, étant donnés des symboles $a$, $b$, $c$, on note $abc$ le mot de taille $3$ qui a le symbole $a$ à l'index $1$, le symbole $b$ à l'index $2$ et le symbole $c$ à l'index $3$.

On utilise la même notation pour signifier un mot de taille $1$ et le symbole qu'il comprend.
Le contexte d'utilisation permet généralement de désambiguïser la notation. 

\subsubsection{Mot vide}

On dénote le mot vide par $\epsilon$.
Le mot vide est de taille $0$.

\[
|\epsilon| = 0
\]

\subsubsection{Concatenation de mots}

On note la concaténation de deux mots $x$ et $y$ par $x \cdot y$, ou tout simplement $xy$.
La concaténation $x \cdot y$ des deux mots $x$ et $y$ est aussi un mot, dont la taille et le contenu sont définis comme-ci:

\begin{align*}
|x \cdot y| &\eqdef |x| + |y|\\
(x \cdot y)(i) &\eqdef \begin{cases}
x(i) & \text{si $i \leq |x|$}\\
y(i - |x|) & \text{autrement}
\end{cases}
\end{align*}

L'opération de concaténation est associative:
\[
\forall x_1. \forall x_2. \forall x_3.\ x_1 \cdot (x_2 \cdot x_3) = (x_1 \cdot x_2) \cdot x_3   
\]

Le mot vide est l'élément neutre de la concaténation:
\[
\forall x. x \cdot \epsilon = x \wedge \epsilon \cdot x = x
\]

\section{Langages}

On appelle un ensemble de mots un \og \textit{langage} \fg{}.
De manière intéressante, on peut représenter un problème par le langage de ses instances positives.

\subsection{Opérations sur les langages}

En tant qu'ensemble, les langages admettent toutes les opérations ensemblistes précédemment évoquées, telles que l'union ($\cup$), l'intersection ($\cap$) ou encore la définition par compréhension ($\{ x \in L\ |\ A \}$).
En plus de cela, nous considérerons des opérations propres aux langages.

\subsubsection{Concaténation de langages}

Deux langages $L_1$ et $L_2$ peuvent être concaténés, noté $L_1 \cdot L_2$, ou simplement $L_1L_2$.
Un mot est dans $L_1 \cdot L_2$ si et seulement si il peut être coupé en deux parties de telle façon à ce que la première fasse partie de $L_1$ et la seconde de $L_2$.
\[
L_1 \cdot L_2 \eqdef \{ x \in \Sigma^*\ |\ \exists x_1. \exists x_2. x_1 \in l_1 \wedge x_2 \in l_2 \wedge x = x_1 \cdot x_2\ \}
\]
Ou, de manière équivalente:
\[
\forall x.\ x \in L_1 \cdot L_2 \iff (\exists x_1. \exists x_2.\ x_1 \in L_1 \wedge x_2 \in L_2 \wedge x = x_1 \cdot x_2)
\]

\subsubsection{Exponentiation de langages}

Étant donné un langage $L$, on note $L^n$ l'ensemble des mots issus de la concaténation de $n$ mots (potentiellement différents) de $L$.
L'opération est définie de manière inductive comme suit:
\begin{align*}
L^0 &\eqdef \{ \epsilon \}\\
L^{i + 1} &\eqdef L \cdot L_i\\
\end{align*}
Ou de manière équivalente:
\begin{align*}
\forall x.\ x \in L^0 &\iff x = \epsilon\\
\forall x. \forall i.\ x \in L^{i + 1} &\iff x \in L \cdot L^i\\
\end{align*}

\subsubsection{Fermeture itérative}

La \og \textit{fermeture itérative} \fg{} d'un langage $L$, aussi appelée \og \textit{fermeture de Kleene} \fg{} de $L$, est l'ensemble de tous les mots qui peuvent être découpés en sous-parties de $L$. On dénote la fermeture iterative de $L$ par $L^*$.
L'ensemble $L^*$ est le résultat de l'union infinie des puissances de~$L$:
\[
L^* \eqdef \bigcup_{i \in \mathbb{N}} L^i
\]
Autrement dit, sous forme de proposition:
\[
\forall x.\ x \in L^* \iff (\exists i. i \in \mathbb{N} \wedge x \in L^i)
\]

\subsubsection{Complément}

Étant donné un alphabet $\Sigma$, le \og \textit{complément} \fg{} d'un langage $L$ est l'ensemble de tous les mots définis sur l'alphabet $\Sigma$ qui ne font pas partie de $L$, (c'est donc l'ensemble des mots formés par les instances négatives de ce langage ?). On dénote par $\overline{L}$ le complément de $L$.
Formellement:
\[
\overline{L} \eqdef \{ x \in \Sigma^*\ |\ x \not\in L\ \}
\]
Ou, de façon équivalente:
\[
\forall x. \ x \in \overline{L} \iff (x \in \Sigma^* \wedge \not\in L)
\]

\subsection{Précédence des opérateurs}

La précédence des opérateurs, par ordre décroissant, est donnée par:
\begin{align*}
{\overline{x}}, x^* & \hspace{1em}\text{complément, fermeture itérative}\\
{\cdot} & \hspace{1em}\text{concaténation}\\
{\cap} & \hspace{1em}\text{intersection}\\
{\cup} & \hspace{1em}\text{union}\\
{\subseteq}, {=}, \dots & \hspace{1em}\text{comparateurs}\\
{\vdots} & \hspace{1em}\text{connecteurs logiques}
\end{align*}

\section{Ensemble des langages}

Étant donné un alphabet $\Sigma$, l'ensemble de tous les langages prenant $\Sigma$ comme alphabet est simplement $\mathcal{P}(\Sigma^*)$.

\subsection{Dénombrabilité des mots}

Nous montrons qu'il existe une bijection entre $\mathbb{N}$ et $\Sigma^*$.
Pour ce faire, prenons une bijection $f$ entre l'ensemble fini $\Sigma$ et le nombre $n > 0$ qui représente sa cardinalité $|\Sigma|$.
Notons qu'il existe $n^k$ mots de taille $k$, c'est-à-dire qu'il existe une bijection entre les nombres naturels plus petit que $n^k$ et $\Sigma^k$. Pour chaque $i$, appelons $f_i$ la bijection entre ces deux ensembles.

Considérons la suite de nombres $s$ définie de la façon suivante:
\begin{align*}
s_0 &\eqdef 0\\
s_{i + 1} &\eqdef s_i + n^i
\end{align*}

Soit $g$ une fonction construite de la façon suivante:
\[
g \eqdef \bigcup_{i \in \mathbb{N}} \{\ p \in \mathbb{N} \times \Sigma^* |\ \exists j.\ p = (s_i + j, f_i(j)) \wedge j < n^i\ \}
\]
Il est aisé de démontrer (par induction naturelle) que $g$ est bien une bijection entre $\mathbb{N}$ et $\Sigma^*$, montrant ainsi que l'ensemble $\Sigma^*$ est dénombrable.

\textit{c.q.d.f.}

\subsection{Indénombrabilité de l'ensemble des langages}

Nous allons maintenant démontrer que l'ensemble des langages $\mathcal{P}(\Sigma^*)$ n'est pas dénombrable, c'est-à-dire que cet ensemble a une cardinalité plus grande que celle, déjà infinie, de l'ensemble des nombres naturels.
Pour ce faire, nous allons utiliser une preuve par \og \textit{diagonalisation} \fg{}.

Commençons par prouver que les deux ensembles n'ont pas la même cardinalité.
Pour ce faire, partons de l'hypothèse que les deux ensembles ont la même cardinalité.
Par définition, il existe donc une bijection entre $\mathbb{N}$ et $\mathcal{P}(\Sigma^*)$.
Notons cette bijection $f$.
D'après le théorème démontré précédemment, il existe aussi une bijection entre les éléments de $\Sigma^*$ et $\mathbb{N}$.
Soit $g$ une telle bijection. 
Vers une contradiction, montrons maintenant qu'il existe un langage $L$ auquel aucun nombre naturel n'a été attribué par la bijection $f$.
Le langage $L$ peut être construit ainsi: 
\[
L \eqdef \{ x \in \mathcal{P}(\Sigma^*)\ |\ x \not\in f(g(x))\ \}
\]
La fonction $f$ étant une bijection, il doit exister un nombre naturel $n$ tel que $f(n) = L$.
Considérons maintenant le mot $x$ tels que $g(x) = n$.
Posons-nous la question de savoir si $x \in L$.
Procédons par cas:
\begin{itemize}
\item
Au cas où $x \in L$, par définition nous avons que $x \not\in f(g(x))$.
Comme $f(g(x)) = f(n)$ et que $f(n) = L$, nous avons donc que $x \not\in L$.
Nous obtenons donc une contradiction.
\item
Au cas où $x \not\in L$, on a que $x \not\in f(n)$, c'est-à-dire $x \not\in f(g(x))$.
Par définition de L, nous avons donc que $x \in L$, ce qui amène aussi à une contradiction.
\end{itemize}
Comme les deux cas terminent sur une contradiction, nous pouvons conclure à la négation de notre hypothèse initiale, qui était que les ensembles $\mathbb{N}$ et $\mathcal{P}(\Sigma^*)$ avait la même cardinalité.

Pour finir la preuve que la $\mathcal{P}(\Sigma^*)$ est indénombrable, montrons que $|\mathbb{N}| \leq |\mathcal{P}(\Sigma^*)|$. Avec le résultat précédent, qui établit que $|\mathbb{N}| \neq |\mathcal{P}(\Sigma^*)|$, nous pourrons conclure que $|\mathbb{N}| < |\mathcal{P}(\Sigma^*)|$.
Pour prouver que $|\mathbb{N}| \leq |\mathcal{P}(\Sigma^*)|$, prouvons qu'il existe une injection $h$ entre $\mathbb{N}$ et $\mathcal{P}(\Sigma^*)$.
\[
h(i) \eqdef \{ x \in \Sigma^*\ |\ |x| = i\ \}
\]
La fonction $h$ associe à chaque $i$ le langage qui contient tous les mots de taille $i$. Trivialement, on a que la fonction $h$ est une injection: En effet, il est impossible pour deux naturels différents $i$ et $j$ d'obtenir la même image.

\textit{c.q.f.d}

\subsection{Dénombrabilité des procédures effectives}

Comme nous le postulerons plus tard dans le cours, il est possible de dénombrer les procédures effectives.
En effet, intuitivement, comme nous exigeons des procédures effectives qu'elle contiennent un nombre fini d'instructions finies, alors chaque procédure effective a une représentation finie.
Un procédure pourra donc être exprimée comme un mot d'un alphabet fini.
De  plus, comme démontré précédemment, l'ensemble des mots d'un alphabet est dénombrable.
Il y a donc uniquement une quantité dénombrable de procédures effectives.
À contrario, l'ensemble des problèmes est indénombrable, même si l'on se restreint aux problèmes qui admettent une fonction d'encodage.
Il y a donc plus (et même infiniment plus) de problèmes que de procédures effectives pour résoudre ces problèmes.
Il y aura donc des problèmes qui n'admettront aucune procédure effective: On parlera de problèmes \og \textit{indécidables} \fg{}.

Ce raisonnement démontre qu'il existe des problèmes indécidables, mais il n'en exhibe pas directement un.
Il est raisonnable de se poser la question s'il existe des problèmes \textit{intéressants} qui sont indécidables.
Comme nous le montrerons dans la suite du cours, la réponse est oui:
Par exemple, le problème de l'arrêt, qui pose la question de savoir si une procédure termine pour toutes ses instances, est indécidable.


 


% !TEX root = ../cours.tex
\chapter{Langages réguliers}

Dans ce chapitre, nous allons étudier une classe de langages appelés \og \textit{languages réguliers} \fg{}.
Un langage régulier est un ensemble de mots qui peut être décrit à l'aide d'une \og \textit{expression régulière} \fg{}.
Les expressions régulières sont une notation pour décrire des languages de façon algébrique.
Tel que nous allons les introduire, les \og \textit{constructeurs} \fg{} d'expressions régulières correspondront aux opérations ensemblistes sur les langages que nous avons établis au chapitre précédent.
Nous introduirons ensuite la notion d\og \textit{automates finis} \fg{} afin de donner une représentation des langages régulier qui admet une procédure effective très simple de vérification de l'appartenance d'un mot au langage.
Nous montrerons finalement qu'il existe des langages non-réguliers qui semblent pourtant admettre des procédures effectives, nous poussant ainsi à chercher d'autre formalismes de description de langages.

\section{Expressions régulières}

Une \og \textit{expression régulière} \fg{} est une représentation sous forme d'\textit{expression} d'un langage.
Une expression régulière est constituée d'expressions atomiques, tels que $\epsilon$, $\emptyset$ ou n'importe quel symbole d'un alphabet $\Sigma$, composées entre elles à l'aide de constructeurs tels que $\cdot$, $\cup$, et $*$. Formellement, étant donné un alphabet $\Sigma$, l'ensemble des expressions régulières est le \textit{plus petit} ensemble qui satisfait les propositions suivantes:
\begin{itemize}
\item $\epsilon$ est une expression régulière.
\item $\emptyset$ est une expression régulière.
\item Pour tout symbole $s \in \Sigma$, $s$ est une expression régulière.
\item $(e_1 \cdot e_2)$ est une expression régulière si et seulement si $e_1$ et $e_2$ sont des expressions régulières.
\item $(e_1 \cup e_2)$ est une expression régulière si et seulement si $e_1$ et $e_2$ sont des expressions régulières.
\item $(e)^*$ est une expression régulière si et seulement si $e$ est une expression régulière.
\end{itemize}

Quand le sens est clair, on omettra les parenthèses lors de l'écriture d'une expression régulière.
On considère que le constructeur $*$ a la plus haute précédence, suivit de $\cdot$ et enfin de $\cup$.

\subsection{Différence entre expressions et langages}

La définition des expressions régulières peut prêter à confusion.
Pourquoi définir les expressions régulières alors que nous disposons déjà de toutes ces constantes et opérations sur les langages?
La réponse est simple: Une expression régulière, contrairement à une langage, peut être \textit{manipulée symboliquement}.
La structure d'une expression régulière peut être analysée.
De plus, comme on le verra tout soudainement, nous pourrons aussi raisonner par induction sur la structure d'une expression régulière.
Contrairement à une expression, un langage n'a pas de structure: il s'agit simplement d'une collection de mots.

\subsection{Langage d'une expression régulière}

Les expressions régulières correspondent à un langage, qui est donné par la fonction $L$, définie comme suit:
\begin{align*}
L(\epsilon) &\eqdef \{\ \epsilon\ \}\\
L(\emptyset) &\eqdef \emptyset\\
L(a) &\eqdef \{\ a\ \} \hspace{2em}\text{Pour tout symbole $a \in \Sigma$}\\
L(e_1 \cdot e_2) &\eqdef L(e_1) \cdot L(e_2)\\
L(e_1 \cup e_2) &\eqdef L(e_1) \cup L(e_2)\\
L(e^*) &\eqdef L(e)^*
\end{align*}

\subsection{Langages réguliers}

On appelle \og \textit{langage régulier} \fg{} un langage $L$ si et seulement si il existe une expression régulière qui a pour langage $L$.

%De ces définitions, on obtient les propositions suivantes pour n'importe quelles expressions $e$, $e_1$, $e_2$, mot $x$ et symbole $a$:
%\begin{align*}
%x \in L(\epsilon) &\iff x = \epsilon\\
%x \in L(\emptyset) &\iff \bot\\
%x \in L(a) &\iff x = a\\
%x \in L(e_1 \cdot e_2) &\iff (\exists x_1. \exists x_2.\ x_1 \in L(e_1) \wedge x_2 \in L(e_2) \wedge x = x_1 \cdot x_2)\\
%x \in L(e_1 \cup e_2) &\iff x \in L(e_1) \vee x \in L(e_2)\\
%x \in L(e^*) &\iff x = \epsilon \vee x \in L(e) \cdot L(e^*)\\
%\end{align*}

\subsection{Principe d'induction structurelle}

Pour prouver une propriété $P[e]$ pour toute expression régulière $e$, il est possible de procéder par \og \textit{induction structurelle} \fg{}. Pour cela, il suffit de:
\begin{enumerate}
\item Prouver la propriété sur $\epsilon$, c'est-à-dire $P[\epsilon]$.
\item Prouver la propriété sur $\emptyset$, c'est-à-dire $P[\emptyset]$.
\item Prouver la propriété sur $a$ pour tout symbole $a$ de l'alphabet,  c'est-à-dire $\forall a. a \in \Sigma \implies P[a]$.
\item Prouver la propriété sur $e_1 \cdot e_2$, en admettant comme hypothèse d'induction $P[e_1]$ et $P[e_2]$.
\item Prouver la propriété sur $e_1 \cup e_2$, en admettant comme hypothèse d'induction $P[e_1]$ et $P[e_2]$.
\item Prouver la propriété sur $e^*$, en admettant comme hypothèse d'induction $P[e]$.
\end{enumerate}

\paragraph{Exemple de preuve par induction structurelle}

Prouvons le théorème suivant: \og \textit{Si une expression régulière $e$ a pour langage l'ensemble vide, alors forcément elle contient l'expression $\emptyset$ comme sous-expression.} \fg.
Procédons par induction structurelle sur $e$.
\begin{enumerate}
\item Si $e = \epsilon$, alors nous arrivons à une contradiction, car le langage de $\epsilon$ n'est pas vide, ce qui clôt le cas.
\item Si $e = \emptyset$, alors directement $e$ contient $\emptyset$ comme sous-expression, ce qui clôt le cas.
\item Si $e = a$ pour un symbole $a$ de $\Sigma$, alors nous arrivons à une contradiction, car le langage de $a$ n'est pas vide, ce qui clôt le cas.
\item Considérons le cas où $e = e_1 \cdot e_2$ pour deux expressions $e_1$ et $e_2$.
      Si le langage de $e$ est vide, alors forcément soit le langage de $e_1$ est vide, soit le langage de $e_2$ est vide.
      \begin{enumerate}
      \item Dans le cas où le langage de $e_1$ est vide, alors on a, par hypothèse d'induction,
            que $e_1$ contient le symbole $\emptyset$ comme sous-expression, et donc c'est aussi le cas pour $e$.
      \item Dans le cas où le langage de $e_2$ est vide, on procède de façon similaire.
      \end{enumerate}
\item Considérons le cas où $e = e_1 \cup e_2$ pour deux expressions $e_1$ et $e_2$.
      Si le langage de $e$ est vide, alors forcément à la fois le langage de $e_1$ est vide et le langage de $e_2$ est vide.
      Par hypothèse d'induction, l'expression $e_1$ contient donc $\emptyset$ comme sous-expression (idem pour $e_2$), et donc $e$ aussi.
\item Si $e = e_1^*$, alors immédiatement nous arrivons à une contradiction, car le langage de $e_1^*$ n'est pas vide, ce qui clôt le cas.
\end{enumerate}

\section{Automates finis déterministes}

Un \og \textit{automate fini déterministe} \fg{}, est une structure mathématique composée de:
\begin{enumerate}
\item Un ensemble fini $Q$ d'états.
\item Un alphabet $\Sigma$.
\item Une fonction de transition $\delta : Q \times \Sigma \to Q$.
\item Un état initial $s \in Q$.
\item Un ensemble d'états finaux $F \subset Q$.
\end{enumerate}

\subsection{Configuration}

On appelle une \og configuration \fg{} d'un automate fini déterministe la paire d'un état $q \in Q$ et d'un mot $x \in \Sigma^*$.

\subsection{Dérivation \& acceptation}

Étant donné un automate fini déterministe $A$, une configuration $(q_1, x_1)$ est \og \textit{dérivable en une étape} \fg{} d'une autre configuration $(q_0, x_0)$, écrit $(q_0, x_0) \vdash_A (q_1, x_1)$, si et seulement si il existe un symbole $a \in \Sigma$ tel que:
\begin{enumerate}
\item Que la fonction de transition amène $q_1$ depuis $q_0$ en voyant le symbole~$a$: $\delta(q_0, a) = q_1$
\item Que le mot $x_0$ commence par le symbole $a$ et que le reste soit $x_1$: $x_0 = a \cdot x_1$
\end{enumerate}

On dit qu'une configuration $(q_1, x_1)$ est \og \textit{dérivable} \fg{} d'une autre configuration $(q_0, x_0)$, écrit $(q_0, x_0) \vdash_A^* (q_1, x_1)$, si et seulement si il existe un nombre naturel $k$ et une série $c_1 \dots c_n$ de $k$ configurations tels que:
\begin{gather*}
c_1 = (q_0, x_0)\\
c_k = (q_1, x_1)\\
\forall i.\ i \geq 1 \wedge i < k \implies c_i \vdash_A c_{i + 1}
\end{gather*}

On dit d'un automate $A = (Q, \Sigma, \delta, s, F)$ qu'il \og \textit{accepte} \fg{} un mot $x \in \Sigma^*$ si et seulement si il existe un état $f \in F$ tel que:
\[
(s, x) \vdash_A^* (f, \epsilon)
\]
C'est-à-dire qu'il est possible d'aller de l'état initial $s$ à un état final $f$ en suivant la fonction de transition sur les symboles successifs de $x$.

Le langage d'une automate fini déterministe est l'ensemble des mots qu'il accepte.

\section{Non-déterminisme}

Les automates présentés jusqu'à présent sont appelés \textit{déterministes}. En effet, la fonction de transition de ces automates indique de façon non-ambiguë l'état suivant en fonction de l'état courant et du prochain symbole en entrée.
Il est possible de relâcher cette contrainte et donner plus de souplesse à l'automate , en introduisant une notion de choix non-déterministe: L'automate peut décider par lui-même quelle transition prendre parmi un choix fini de possibilités.
Intuitivement, on peut se représenter cette capacité non-déterministe comme celle de faire le \og \textit{bon choix} \fg, ou alors comme celle de pouvoir effectuer en parallèle des exécutions.

Le non-déterminisme joue un rôle important en théorie de la calculabilité, et encore plus en théorie de la complexité.

\section{Automates finis non-déterministes}

Un \og \textit{automate fini non-déterministe} \fg{}, est une structure mathématique composée de:
\begin{enumerate}
\item Un ensemble fini $Q$ d'états.
\item Un alphabet $\Sigma$.
\item Une relation de transition de taille finie $\Delta \subseteq Q \times \Sigma^* \times Q$.
\item Un état initial $s \in Q$.
\item Un ensemble d'états finaux $F \subset Q$.
\end{enumerate}

\subsection{Configuration}

On appelle une \og configuration \fg{} d'un automate fini non-déterministe la paire d'un état $q \in Q$ et d'un mot $x \in \Sigma^*$.

\subsection{Dérivation \& acceptation}

Étant donné un automate fini non-déterministe $N$, une configuration $(q_1, x_1)$ est \og \textit{dérivable en une étape} \fg{} d'une autre configuration $(q_0, x_0)$, écrit $(q_0, x_0) \vdash_A (q_1, x_1)$, si et seulement si il existe un mot $x \in \Sigma^*$ tel que:
\begin{enumerate}
\item Que la relation de transition permette d'amener à $q_1$ depuis $q_0$ en voyant le mot~$x$: $(q_0, x, q_1) \in \Delta$
\item Que le mot $x_0$ commence par le préfixe $x$ et que le reste soit $x_1$: $x_0 = x \cdot x_1$
\end{enumerate}

On dit qu'une configuration $(q_1, x_1)$ est \og \textit{dérivable} \fg{} d'une autre configuration $(q_0, x_0)$, écrit $(q_0, x_0) \vdash_N^* (q_1, x_1)$, si et seulement si il existe un nombre naturel $k$ et une série $c_1 \dots c_n$ de $k$ configurations tels que:
\begin{gather*}
c_1 = (q_0, x_0)\\
c_k = (q_1, x_1)\\
\forall i.\ i \geq 1 \wedge i < k \implies c_i \vdash_N c_{i + 1}
\end{gather*}

On dit d'un automate fini non-déterministe $N = (Q, \Sigma, \Delta, s, F)$ qu'il \og \textit{accepte} \fg{} un mot $x \in \Sigma^*$ si et seulement si il existe un état $f \in F$ tel que:
\[
(s, x) \vdash_N^* (f, \epsilon)
\]
C'est-à-dire qu'il est possible d'aller de l'état initial $s$ à un état final $f$ en suivant la relation de transition sur une décomposition de $x$ en sous-parties.

Le langage d'un automate fini non-déterministe est l'ensemble des mots qu'il accepte.

\section{Déterminisation}

Comme nous allons le démontrer par un procédé, il est toujours possible de passer d'un automate fini non-déterministe à un automate fini déterministe ayant le même langage.
Le procédé de \og déterminisation \fg{} d'un automate fini non-déterministe $N_1 = (Q_1, \Sigma, \Delta_1, s_1, F_1)$ consiste en deux étapes:
\begin{enumerate}
\item
Dans la première étape, un automate fini non-déterministe $N_2 = (Q_2, \Sigma, \Delta_2, s_2, F_2)$  est construit afin d'éliminer toutes les transitions ayant pour argument un mot de taille $k$ plus grande que $1$. Pour ce faire, il suffit de supprimer toutes les transitions $(q_1, a_1 \dots a_k, q_{k+1})$ et d'incorporer:
\begin{enumerate}
\item À l'ensemble $Q_2$ un nombre $k-1$ de \textit{nouveaux} états $q_2 \dots q_k$,
\item À la relation de transition $\Delta_2$, toutes les transitions $(q_i, a_i, q_{i+1})$ pour tout $i$ de $1$ à $k$.
\end{enumerate}
L'automate $N_2$ résultant est équivalent $N_1$.
\item
Deuxièmement, un automate fini déterministe $A = (Q_3, \Sigma, \delta_3, s_3, F_3)$  est construit.
Afin de gérer les transitions ayant pour argument $\epsilon$ (transitions $\epsilon$), on considère la fonction de \og \textit{fermeture-$\epsilon$} \fg{} dénotée $E : Q_2 \to \mathcal{P}(Q_2)$, qui retourne tous les états accessibles depuis un état donné en empruntant uniquement des transitions $\epsilon$:
\[
E(q) \eqdef \{\ q' \in Q_2\ |\ (q, \epsilon) \vdash_{N_2}^* (q', \epsilon)\ \} 
\]
Étant donné cette fonction, on construit l'automate fini déterministe $A = (Q_3, \Sigma, \delta_3, s_3, F_3)$ de la façon suivante:
\begin{enumerate}
\item L'ensemble d'état $Q_3$ est égal à $\mathcal{P}(Q_2)$, c'est-à-dire que chaque état de $Q_3$ est un ensemble d'état de $Q_2$.
\item L'alphabet $\Sigma$ reste inchangé.
\item La fonction de transition $\delta_3$ est définie comme suit:
\[
\delta_3(q, a) \eqdef \bigcup_{q' \in q} \{\ E(p)\ |\ (q', a, p) \in \Delta_2\ \}
\]
\item L'état initial $s_3$ est égal à $E(s_2)$,
\item L'ensemble d'états finaux $F_3$ est égal à l'ensemble des états de $Q_3$ qui contiennent comme élément au moins un état final de $Q_2$:
\[
F_3 = \{\ q \in \mathcal{P}(Q_2)\ |\ q \cap F_2 \neq \emptyset\ \}
\] 
\end{enumerate}
\end{enumerate}

L'existence d'une telle transformation démontre que l'ajout du non-déterminisme aux automates finis n'augmente pas leur expressivité. Aussi, trivialement, chaque automate fini déterministe est un automate fini non-déterministe, ce qui nous permet de conclure que les deux formalismes ont exactement la même expressivité.

Il est à noter cependant que le processus de déterminisation peut faire \textit{exploser} le nombre d'état.
Étant donné un automate non-déterministe de taille $n$, l'automate déterministe obtenu par le procédé de déterminisation a $2^n$ états.

\section{Minimisation}

Étant donné un automate fini déterministe, il est possible de le \og \textit{minimaliser} \fg{} pour obtenir un automate fini déterministe équivalent mais de taille minimale.

Le processus de minimisation est un processus itératif qui cherche à établir des classes d'équivalence entre les états d'un automate.
Considérons $n$ le nombre d'états dans $Q$, et notons $q_1, \dots, q_n$ les $n$ états.

On commence avec un tableau qui contient une entrée pour chaque paire d'état $(q_i, q_j)$ où $i < j$.
Initialement, les entrées du tableau sont mises à $1$, pour indiquer une potentielle équivalence.
Ensuite, on marque comme non-équivalente toute paire d'états $(q_i, q_j)$ où un état est final et l'autre ne l'est pas.
Ces deux états sont, de toute évidence, non-équivalents.

Ensuite, on commence un processus itératif. À chaque itération, on parcourt l'ensemble des paires d'états $(q_i, q_j)$ encore considérées équivalentes.
Pour chacune de ces paires, on parcourt tous les symboles $a$ de $\Sigma$ et on regarde si $\delta(q_i, a)$ est considéré comme équivalent à $\delta(q_j, a)$.
Si ce n'est pas le cas, on marque $(q_i, q_j)$ comme non-équivalente.
Si au moins une paire a été marquée lors de l'itération, on procède à l'itération suivante, sinon le processus s'arrête.

Quand le processus se termine, le tableau contient un $1$ pour chaque paire d'états équivalents.
L'automate peut donc être modifié afin d'adopter ces classes d'équivalence comme états.

\section{Transformation d'une expression régulière en automate}

Étant donné un alphabet $\Sigma$ et une expression régulière $e$, il est possible de construire, de façon récursive, un automate équivalent $A$.
Par simplicité, on choisit de générer un automate non-déterministe.
La procédure de transformation opère de façon récursive sur la structure de l'expression régulière:
\begin{itemize}
\item Dans le cas où $e = \emptyset$, on crée un nouvel état $q$ non final. L'automate retourné est égal à:
\[
(\{\ q\ \}, \Sigma, \emptyset, q, \emptyset)
\] 
\item Dans le cas où $e = \epsilon$, on crée un nouvel état $q$ que l'on considère comme final. L'automate retourné est égal à:
\[
(\{\ q\ \}, \Sigma, \emptyset, q, \{\ q\ \})
\]
\item Dans le cas où $e = a$ pour $a \in \Sigma$, on crée deux nouveaux états $q_1$ et $q_2$, avec $q_1$ initial et $q_2$ final.
La relation de transition $\Delta$ contient uniquement la transition de $q_1$ à $q_2$ par $a$.
\[
(\{\ q_1, q_2\ \}, \Sigma, \{\ (q_1, a, q_2)\ \}, q_1, \{\ q_2\ \})
\]
\item Dans le cas où $e = e_1 \cdot e_2$ pour deux expressions $e_1$, $e_2$, on applique la procédure récursivement à $e_1$ et à $e_2$, ce qui nous fournit deux automates $N_1 = (Q_1, \Sigma, \Delta_1, s_1, F_1)$ et $N_1 = (Q_2, \Sigma, \Delta_2, s_2, F_2)$ avec, par construction, $Q_1$ et $Q_2$ disjoints. Dans ce cas il suffit simplement d'ajouter une transition-$\epsilon$ de chaque état final de $N_1$ à l'état initial de $N_2$. En retour:
\[
(Q_1 \cup Q_2, \Sigma, \Delta_1 \cup \Delta_2 \cup \{\ (q, \epsilon, s_2)\ |\ q \in F_1\ \}, s_1, F_2)
\]
\item Dans le cas où $e = e_1 \cup e_2$ pour deux expressions $e_1$, $e_2$, on applique la procédure récursivement à $e_1$ et à $e_2$, ce qui nous fournit deux automates $N_1 = (Q_1, \Sigma, \Delta_1, s_1, F_1)$ et $N_1 = (Q_2, \Sigma, \Delta_2, s_2, F_2)$ avec, par construction, $Q_1$ et $Q_2$ disjoints.
On crée un nouvel état initial $q$, et deux transitions-$\epsilon$, une de $q$ à $s_1$ et une de $q$ à $s_2$. On retourne:
\[
(Q_1 \cup Q_2, \Sigma, \Delta_1 \cup \Delta_2 \cup \{\ (q, \epsilon, s_1), (q, \epsilon, s_2)\ \}, q, F_1 \cup F_2)
\]
\item Dans le cas où $e = e_1*$ pour une expression $e_1$, on applique la procédure récursivement à $e_1$, ce qui nous fournit un automate $N_1 = (Q_1, \Sigma, \Delta_1, s_1, F_1)$.
On crée un nouvel état initial $q$, que l'on considère comme final, et une transition-$\epsilon$ de $q$ à $s_1$. De plus, pour chaque état $f$ de $F_1$ on ajoute une transition-$\epsilon$ de $f$ à $s_1$, qui permet de boucler. On retourne:
\[
(Q_1 \cup \{\ q\ \}, \Sigma, \Delta_1 \cup \{\ (q, \epsilon, s_1)\ \} \cup \{\ (f, \epsilon, s_1)\ |\ f \in F_1\ \}, q, F_1 \cup \{\ q\ \})
\]
\end{itemize}

L'automate retourné par la procédure est un automate non-déterministe équivalent à l'expression régulière donnée en argument.
Ce théorème peut être prouvé par induction structurelle sur la structure de l'expression régulière. 

Cette transformation démontre que l'expressivité des automates non-déterministes finis est au moins aussi grande que celle des expressions régulières.
De plus, une telle transformation est souvent employée en pratique pour convertir des expressions régulières (qui sont facilement exprimée par un programmeur) en automates finis (qui sont exécutables efficacement sur une machine).
Voir par exemple l'outil \texttt{lex} et ses dérivés pour définir des \textit{lexeurs} (ou \textit{analyseurs lexicaux}).


\section{Transformation d'un automate en expression régulière}

Pour opérer la transformation d'un automate fini non-déterministe en expression régulière, on considère une extension aux automates finis où les transitions ont pour label \textit{une expression régulière}. Trivialement, chaque automate fini non-déterministe est un membre de cette classe.
Le livre de référence propose une autre présentation de la méthode qui ne nécessite pas l'introduction de cette extension mais qui est moins intuitive (pour l'instructeur).

Étant donné un automate fini non-déterministe $N = (Q, \Sigma, \Delta, s, F)$, la transformation en expression régulière commence par l'introduction d'un nouvel état initial $q_\text{initial}$ et d'un nouvel état final $q_\text{final}$. Une transition-$\epsilon$ est ajoutée de $q_\text{initial}$ à $s$ et de chaque état $f \in F$ à $q_\text{final}$. De cette façon, l'état initial n'a aucun transition entrante, et l'état final aucune transition sortante. On note $E_0$ l'automate résultant.
\begin{align*}
E_0 &\eqdef (Q_0, \Sigma, \Delta_0, q_\text{initial}, \{\ q_\text{final}\ \})\\
Q_0 &\eqdef Q \cup \{\ q_\text{initial}, q_\text{final}\ \}\\
\Delta_0 &\eqdef \Delta \cup \{\ (q_\text{initial}, \epsilon, s)\ \} \cup \{\ (f, \epsilon, q_\text{final})\ |\ f \in F\ \}
\end{align*}
Nommons les états de l'automate $N$ par $q_1, \dots, q_n$, où $n$ est égal à $|Q|$.
On procède ensuite de manière itérative pour supprimer les états $q_1, \dots, q_n$ de l'automate en les remplaçant par des transitions équivalentes.

Pour chaque $i$ de $0$ à $n-1$, on construit l'automate \[E_{i+1} = (Q_{i+1}, \Sigma, \Delta_{i+1}, q_\text{initial}, \{\ q_\text{final}\ \})\] en supprimant l'état $q_i$ de l'automate $E_i$:
\[
Q_{i+1} \eqdef Q_i - \{\ q_i\ \}
\]
Pour supprimer un état de l'automate en laissant intact le sens de l'automate, il faut mettre à jour la relation de transition.
On distingue deux cas:
\begin{enumerate}
\item Dans le cas où il existe au moins une transition $(q_i, x, q_i)$ dans $\Delta_i$, on construit $\Delta_{i+1}$ de la manière suivante:
\begin{align*}
\Delta_{i+1} \eqdef \{\ (q_a, e, a_b) \in \Delta_i\ |\ &q_a \neq q_i \wedge q_b \neq q_i\ \}\ \cup\\
\{ (q_a, e_1 \cdot e_2^* \cdot e_3, q_b)\ |\ &(q_a, e_1, q_i) \in \Delta_i\ \wedge\\
&(q_i, e_2, q_i) \in \Delta_i\ \wedge\\
&(q_i, e_3, q_b) \in \Delta_i\ \wedge\\
&q_a \neq q_i \wedge q_b \neq q_i\ \}
\end{align*}
\item
Dans le cas où il n'existe aucune transition directe de $q_i$ à $q_i$, on construit $\Delta_{i+1}$ de la manière suivante:
\begin{align*}
\Delta_{i+1} \eqdef \{\ (q_a, e, a_b) \in \Delta_i\ |\ &q_a \neq q_i \wedge q_b \neq q_i\ \}\ \cup\\
\{ (q_a, e_1 \cdot e_2, q_b)\ |\ &(q_a, e_1, q_i) \in \Delta_i\ \wedge\\
&(q_i, e_2, q_b) \in \Delta_i\ \wedge\\
&q_a \neq q_i \wedge q_b \neq q_i\ \}
\end{align*}
\end{enumerate}

Une fois l'automate $E_n$ construit, il ne reste plus que $q_\text{initial}$ et $q_\text{final}$ dans l'ensemble des états.
Il ne reste plus qu'à prendre l'union de toutes les expressions régulières apparaissant en label d'une transition de $\Delta_n$:
\[
\bigcup_{(q_\text{initial}, e, q_\text{final}) \in \Delta_n} e
\]
Le résultat est une expression régulière de même langage que l'automate non-déterministe initial $N$.

Cette transformation démontre que l'expressivité des expressions régulières est au moins aussi grande que celle des automates non-déterministes finis.
Comme la transformation inverse existe, les deux formalismes sont de même expressivité.
Cette technique, contrairement à la transformation inverse, semble peu utilisée en pratique.

\section{Théorème du gonflement}

Comme établi au chapitre précédent, étant donné que l'ensemble des expressions régulières est dénombrable, il existe forcément des langages qui ne sont pas descriptibles à l'aide d'expressions régulières.
Cependant, nous allons montrer qu'il existe certains langages qui admettent intuitivement des procédures effectives mais qui ne sont pas reconnus par une expression régulière.

Pour ce faire, nous allons démontrer que tous les langages réguliers $L$ ont une propriété intéressante:
À partir d'une certaine taille $p$ (plus grande ou égal à 1 et qui dépend uniquement du langage $L$), tous leurs mots contiennent une sous-partie non-vide comprises dans les $p$ premiers symboles qui peut être répétée un nombre arbitraire de fois tout en préservant l'appartenance du mot au langage $L$. Formellement, si $L$ est un langage régulier, alors:
\begin{align*}
\exists p. p \geq 1 \wedge \forall x.\ |x| \geq p \implies \exists x_1. \exists x_2. \exists x_3.\ &x = x_1 \cdot x_2 \cdot x_3\ \wedge\\
&|x_1 \cdot x_2| \leq p\ \wedge\\
&|x_2| \geq 1\ \wedge\\
&\forall k.\ x_1 \cdot x_2^k \cdot x_3 \in L
\end{align*}
Une partie du mot peut être \textit{gonflée} et le mot toujours appartenir au langage.
Intuitivement, le phénomène découle du fait qu'une automate fini qui reconnait le langage, n'ayant à sa disposition qu'un nombre fini d'états, est obligé de boucler sur des mots d'une taille suffisamment grande.

Le théorème du gonflement peut être prouvé de manière relativement simple en raisonnant sur les automates finis déterministes.
Comme démontré lors de ce chapitre, un langage $L$ est régulier si et seulement si il est le langage d'un automate fini déterministe. Soit $A = (Q, \Sigma, \delta, s, F)$ un tel automate. On a $L(a) = L$.
Posons $p$ la taille de l'ensemble $Q$.
\[
p \eqdef |Q|
\]
Pour chaque mot $x$ de $L$ de taille au moins $p$, nous avons trivialement les propriétés suivantes:
\begin{gather*}
x \in L\\
|x| \geq p
\end{gather*}
Considérons l'exécution de $A$ sur les $p$ premiers symboles de $x$, et notons $x\rvert_{i}$ le mot $x$ avec ses $i$ premiers symboles retirés. L'exécution passe par une série d'états $q_0, q_1, \dots, q_p$ tels que:
\[
\forall i. i < p \implies (q_i, x_a\rvert_i) \vdash_A (q_{i+1}, x_a\rvert_{i+1})
\]
Comme il existe $p + 1$ états $q_i$ mais seulement $p$ différents états dans $Q$, il y a forcément (au moins) deux indexes $i$ et $j$ tels que\footnote{%
Au passage, on appelle ce principe le \textit{principe des tiroirs}: Si des chaussettes sont rangées dans des tiroirs et qu'il y a plus de chaussettes que de tiroirs, alors il y a forcément un tiroirs qui contient plus d'une chaussette.%
}:
\begin{gather*}
i \neq j\\
q_i = q_j
\end{gather*}
Il y a donc une boucle dans l'exécution: L'exécution passe forcément à deux moments distincts par le même état $q_i$.
Le mot $x$ peut donc être découpé en trois sous-parties $x_1$, $x_2$ et $x_3$, et il existe un état $f \in F$, tels que les propriétés suivantes sont vérifiées:
\begin{align*}
x &= x_1 \cdot x_2 \cdot x_3\\
|x_2| &\geq 1\\
|x_1 + x_2| &\leq p\\
(s, x_1 \cdot x_2 \cdot x_3) &\vdash_A^* (q_i, x_2 \cdot x_3)\\
(q_i, x_2 \cdot x_3) &\vdash_A^* (q_i, x_3)\\
(q_i, x_3) &\vdash_A^* (f, \epsilon)
\end{align*}
La partie $x_1$ correspond aux symboles lus avant la boucle, la partie $x_2$ correspond aux symboles lus durant la boucle, et la partie $x_3$ aux symboles lus après la boucle.

Il reste maintenant simplement à démontrer que répéter $x_2$ un nombre arbitraire $k$ de fois préserve l'appartenance dans le langage.
On s'aidera du lemme suivant sur la relation d'exécution des automates:
\[
\forall x_1. \forall x_2. \forall x_3. \forall q_1. \forall q_2.\ (q_1, x_1 \cdot x_2) \vdash_A^* (q_2, x_2) \iff (q_1, x_1 \cdot x_3) \vdash_A^* (q_2, x_3)
\]
Intuitivement, le lemme stipule que les symboles non lus durant une dérivation ne peuvent pas en influencer le cours.

En appliquant le lemme, on obtient la proposition suivante:
\begin{align*}
(s, x_1 \cdot x_2^k \cdot x_3) &\vdash_A^* (q_i, x_2^k \cdot x_3)
\end{align*}
Il ne reste qu'à prouver $(q_i, x_2^k \cdot x_3) \vdash_A^* (q_i, x_3)$.
On procède par induction naturelle sur $k$.
\begin{enumerate}
\item Considérons le cas $k = 0$. Dans ce cas, $x_2^i = \epsilon$. Les deux côtés de la relation d'exécution réduisant à la même configuration, la proposition est trivialement vérifiée.
\item
Considérons le cas inductif $k = j + 1$ avec une hypothèse d'induction sur $j$.
En appliquant le lemme, on obtient:
\[
(q_i, x_2 \cdot x_2^j \cdot x_3) \vdash_A^* (q_i, x_2^j \cdot x_3)
\]
De plus, par hypothèse d'induction:
\[
(q_i, x_2^j \cdot x_3) \vdash_A^* (q_i, x_3)
\]
On peut donc conclure:
\[
(q_i, x_2 \cdot x_2^j \cdot x_3) \vdash_A^* (q_i, x_3)
\]
Et donc que:
\[
(q_i, x_2^k \cdot x_3) \vdash_A^* (q_i, x_3)
\]
\end{enumerate}

En résumé, nous avons maintenant montré que:
\begin{align*}
(s, x_1 \cdot x_2^k \cdot x_3) &\vdash_A^* (q_i, x_2^k \cdot x_3)\\
(q_i, x_2^k \cdot x_3) \vdash_A^* (q_i, x_3)\\
(q_i, x_3) &\vdash_A^* (f, \epsilon)
\end{align*}
Par transitivité, on conclut:
\[
(s, x_1 \cdot x_2^k \cdot x_3) \vdash_A^* (f, \epsilon)
\]
L'automate $A$ accepte donc le mot $x_1 \cdot x_2^k \cdot x_3$, on a donc $x_1 \cdot x_2^k \cdot x_3 \in L$.







\end{document}
