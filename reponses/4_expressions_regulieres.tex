\documentclass[12pt,french,a4paper]{article}

\usepackage{geometry}
 \geometry{
 a4paper,
 total={170mm,257mm},
 left=20mm,
 top=20mm,
 }
\usepackage[frenchb]{babel}
\usepackage{exsheets}
\usepackage{amsmath}
\usepackage{amssymb}
\usepackage{mathtools}
\usepackage{proof}
\usepackage{logicpuzzle}
\usepackage{hyperref}
\usepackage{cleveref}

% Définition de la commande pour le signe = avec "déf" aussi dessus.
\newcommand\eqdef{\mathrel{\overset{\makebox[0pt]{\mbox{\normalfont\tiny\sffamily déf}}}{=}}}

\begin{document}


\title{\vspace{-2cm}Réponses à la série d'exercices n°2\\\large{Fondamentaux formels / Informatique théorique\\GymInf}}
\date{\vspace{-1cm}9 août 2021}

\maketitle

%****************Exercice 1
\begin{question}
\begin{enumerate}
\item $\{a\}$
\item $\{ab\}$
\item $\{a, b\}$
\item $\{aa, ac, ba, bc\}$
\item $\{\epsilon \}$
\item $\{\}$ ou $\emptyset$
\item $\{a, b,\epsilon, aa, aaa, bb, bbb}$
\item $\{a, b, \espilon, aab, aabb, aaab, aaabb, aaabbb \}$
\item $\{\espilon\}$
\item $\{\} U \{\espilon\}=\{\espilon\}$
\end{enumerate}
\end{question}


%****************Exercice 2
\begin{question}
\begin{enumerate}
\item $a\Sigma^{*}$
\item $abbSigma^{*}c$
\item $a\Sigma^{*}c$
\item $a\Sigma^{*}a$
\item $\Sigma^{*}a\Sigma^{*}$
\item $\Sigma^{*}a\cop\b \Sigma^{*}$
\item $\Sigma^{*}a\Sigma^{*}b \Sigma^{*}\cop \Sigma^{*}b\Sigma^{*}a \Sigma^{*} $
\item $(\Sigma\Sigma)^{*}$
\end{enumerate}
\end{question}

%****************Exercice 3
\begin{question}
\begin{enumerate}

\end{enumerate}
\end{question}

%****************Exercice 4
\begin{question}
\begin{enumerate}

\end{enumerate}
\end{question}
