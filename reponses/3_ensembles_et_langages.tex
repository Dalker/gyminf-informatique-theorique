\documentclass[12pt,french,a4paper]{article}

\usepackage{ae,lmodern}
\usepackage[francais]{babel}
\usepackage[utf8]{inputenc}
\usepackage[T1]{fontenc}
\usepackage{geometry}
 \geometry{
 a4paper,
 total={170mm,257mm},
 left=20mm,
 top=20mm,
 }
\usepackage{exsheets}
\usepackage{amsmath}
\usepackage{amssymb}
\usepackage{mathtools}
\usepackage{proof}
\usepackage{hyperref}
\usepackage{cleveref}

% Définition de la commande pour le signe = avec "déf" aussi dessus.
\newcommand\eqdef{\mathrel{\overset{\makebox[0pt]{\mbox{\normalfont\tiny\sffamily déf}}}{=}}}

\begin{document}


\title{\vspace{-2cm}Réponses à la série d'exercices n°3\\\large{Fondamentaux formels / Informatique théorique\\GymInf}}
\date{\vspace{-1cm}11 août 2021}

\maketitle

%****************Exercice 1
\begin{question}
Dans cet exercice, nous devons montrer que:
\[
|\mathbb{N} \times \{\ 0, 1\ \}| = |\mathbb{N}|
\]

Pour ce faire, montrons l'existence d'une bijection entre $\mathbb{N} \times \{\ 0, 1\ \}$ et $\mathbb{N}$.
Considérons la fonction $f$ suivante:
\[
f(n, b) \eqdef 2n + b
\]
La fonction $f$ est bien une bijection entre $\mathbb{N} \times \{\ 0, 1\ \}$ et $\mathbb{N}$. En effet:
\begin{enumerate}
\item La fonction est surjective: Pour tout nombre $m$, il est possible de l'écrire sous la forme $2n + 0$ (pour $m$ pair) ou $2n + 1$ (pour $m$ impair).
\item La fonction est injective: Si $f(n, b) = f(n', b')$, alors forcément $n = n'$ et $b = b'$.
\end{enumerate}
\end{question}


%****************Exercice 2
\begin{question}
\paragraph{Partie 1}
Admettons que les ensembles $S_i$ sont mutuellement disjoints.
Montrons qu'il existe une bijection $f$ entre $\mathbb{N}$ et $S$:
\[
S \eqdef \bigcup_{i \in \mathbb{N}} S_i
\]

Construisons une fonction $s : \mathbb{N} \to \mathbb{N}$ de la manière suivante:
\[
s(0) \eqdef 0
s(i + 1) \eqdef s(i) + |S_i|
\]

Considérons pour chaque ensemble fini $S_i$ une bijection $f_i : n \to S_i$.
Construisons la bijection $f : \mathbb{N} \to S$ de la manière suivante:
\[
f \eqdef \bigcup_{i \in \mathbb{N}} \{ (n + s(i), f_i(n))\ |\ n \in |S_i| \}
\]
L'ensemble $f$ est bien une fonction, et de plus une bijection.

\paragraph{Partie 2}
Montrons que $\mathbb{N} \times \mathbb{N}$ est dénombrable en utilisant le résultat de la partie 1.
Pour ce faire, montrons que l'on peut partitionner $\mathbb{N} \times \mathbb{N}$ en une union infinie d'ensembles disjoints et de taille finie:
\[
\mathbb{N} \times \mathbb{N} = \bigcup_{i \in \mathbb{N}} \{\ (a, b) \in \mathbb{N} \times \mathbb{N}\ |\ a + b = i\ \}
\]

\paragraph{Partie 3}
L'ensemble en question est dénombrable car il s'agit en fait de $\mathbb{N} \times \mathbb{N}$, qui a été montré dénombrable en partie 2.
\[
\bigcup_{i \in \mathbb{N}} (\mathbb{N} \times \{\ i\ \}) = \mathbb{N} \times \mathbb{N}
\]

\end{question}

\newpage

%****************Exercice 3
\begin{question}
Premièrement, montrons:
\[
|x| <= |\mathcal{P}(x)|
\]
Pour ce faire, construisons une injection $f$ entre $x$ et $\mathcal{P}(x)$:
\[
f(a) \eqdef \{ a \}
\]
Clairement, $f$ est injectif, car si $f(a) = f(a')$, alors $\{ a \} = \{ a' \}$ et donc $a = a'$.

Deuxièmement, montrons qu'il n'existe pas de bijection entre $x$ et $\mathcal{P}(x)$.
Partons de l'hypothèse qu'il existe une telle bijection, et montrons une contradiction.
Soit $g$ une telle bijection.
Construisons l'ensemble $y$ sous-ensemble de $x$ de la manière suivante:
\[
y \eqdef \{\ a \in x\ |\ a \not\in f(a)\ \}
\]
Comme $f$ est une bijection, il existe forcément un $a' \in x$ tel que $f(a') = y$. Posons-nous la question de savoir si $a' \in y$.
Analysons les deux possibilités:
\begin{enumerate}
\item
Dans le cas où $a' \in y$, alors forcément par construction de $y$ on a que $a' \not\in f(a')$, et donc $a' \not\in y$. Contradiction.
\item
Dans le cas où $a' \not\in y$, alors $a' \not\in f(a')$ et donc par construction $a' \in \{\ a \in x\ |\ a \not\in f(a)\ \}$, et donc $a' \in y$. Contradiction.
\end{enumerate}

\end{question}

%****************Exercice 4
\begin{question}
\end{question}

%****************Exercice 5
\begin{question}
\end{question}

\end{document}
