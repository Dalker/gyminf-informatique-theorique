\documentclass[12pt,french,a4paper]{article}

\usepackage{geometry}
 \geometry{
 a4paper,
 total={170mm,257mm},
 left=20mm,
 top=20mm,
 }
\usepackage[frenchb]{babel}
\usepackage{exsheets}
\usepackage{amsmath}
\usepackage{amssymb}
\usepackage{mathtools}
\usepackage{proof}
\usepackage{logicpuzzle}
\usepackage{hyperref}
\usepackage{cleveref}

% Définition de la commande pour le signe = avec "déf" aussi dessus.
\newcommand\eqdef{\mathrel{\overset{\makebox[0pt]{\mbox{\normalfont\tiny\sffamily déf}}}{=}}}

\begin{document}


\title{\vspace{-2cm}Réponses à la série d'exercices n°4\\\large{Fondamentaux formels / Informatique théorique\\GymInf}}
\date{\vspace{-1cm}9 août 2021}

\maketitle

%****************Exercice 1
\begin{question}

\end{question}


%****************Exercice 2
\begin{question}

\end{question}

%****************Exercice 3
\begin{question}
\begin{enumerate}
\item

\item

Montrons que pour tout $i \in \mathbb{N}$, mot non-vide $x$ et langage $L$, si $x \in L^{i+1}$, alors il existe deux mots $x_1$ et $x_2$ tels que:
\begin{enumerate}
\item $x = x_1 \cdot x_2$
\item $x_1 \neq \epsilon$
\item $x_1 \in L$
\item $x_2 \in L^i$
\end{enumerate}
Procédons par induction sur $i$.
\begin{itemize}
\item Considérons le cas où $i = 0$. On a $x \in L^{0+1}$. Observons que $L^{0+1} = L^1 = L$. Posons $x_1 = x$ et $x_2 = \epsilon$. La décomposition $x_1$ et $x_2$ vérifie bien que: 
\begin{enumerate}
\item $x = x_1 \cdot x_2$
\item $x_1 \neq \epsilon$
\item $x_1 \in L$
\item $x_2 \in L^i = L^0 = \{\ \epsilon\ \}$
\end{enumerate}

\item Considérons maintenant le cas inductif, où $i = n + 1$ pour un certain $n$. Notre hypothèse d'induction stipule que:
\[
\forall x'.\ (x' \neq \epsilon \wedge x' \in L^{n+1}) \implies \exists x_1'. \exists x_2'.\ x' = x_1' \cdot x_2' \wedge x_1' \neq \epsilon \wedge x_1' \in L \wedge x_2' \in L^n
\]
Par hypothèse, on a que $x \in L^{(n + 1) + 1}$. Observons $L^{(n + 1) + 1} = L \cdot L^{n + 1}$.
Il existe donc un mot $x_1$ et un mot $x_2$ tels que $x = x_1 \cdot x_2$ avec $x_1 \in L$ et $x_2 \in L^{n + 1}$.
Si $x_1 \neq \epsilon$, alors la preuve conclut immédiatement, avec $x_1$ et $x_2$ comme décomposition.
Considérons donc le cas où $x_1 = \epsilon$. Dans ce cas, on remarque de manière cruciale que $\epsilon \in L$.

Dans ce cas, on observe que $x = x_1 \cdot x_2 = \epsilon \cdot x_2 = x_2$, et donc $x = x_2$. Donc, on a que $x \in L^{n + 1}$.
Par hypothèse d'induction, on a donc qu'il existe un $x_1'$ et un $x_2'$ tels que:
\begin{enumerate}
\item $x = x_1' \cdot x_2'$
\item $x_1' \neq \epsilon$
\item $x_1' \in L$
\item $x_2' \in L^n$
\end{enumerate}
Prouvons que la décomposition $x_1'$ et $x_2'$ est une solution valable.
La seule condition non-triviale est:
\[
x_2' \in L^{n + 1}
\]
Or, comme $\epsilon \in L$, on a que:
\[
\epsilon \cdot x_2' \in L \cdot L^n
\]
Et donc, comme $L \cdot L^n = L^{n + 1}$: 
\[
\epsilon \cdot x_2' \in L^{n + 1}
\]
Et donc:
\[
x_2' \in L^{n + 1}
\]
Ce qui conclut la preuve.
\end{itemize}
\item
\end{enumerate}
\end{question}

%****************Exercice 4
\begin{question}

\end{question}


\end{document}
