\documentclass[12pt,french,a4paper]{article}

\usepackage{ae,lmodern}
\usepackage[francais]{babel}
\usepackage[utf8]{inputenc}
\usepackage[T1]{fontenc}
\usepackage{geometry}
 \geometry{
 a4paper,
 total={170mm,257mm},
 left=20mm,
 top=20mm,
 }
\usepackage{exsheets}
\usepackage{amsmath}
\usepackage{amssymb}
\usepackage{mathtools}
\usepackage{proof}
\usepackage{logicpuzzle}
\usepackage{hyperref}
\usepackage{cleveref}
\usepackage{xcolor}

% Définition de la commande pour le signe = avec "déf" aussi dessus.
\newcommand\eqdef{\mathrel{\overset{\makebox[0pt]{\mbox{\normalfont\tiny\sffamily déf}}}{=}}}

\begin{document}


\title{\vspace{-2cm}Réponses à la série d'exercices n°2\\\large{Fondamentaux formels / Informatique théorique\\GymInf}}
\date{\vspace{-1cm}10 août 2021}

\maketitle

%****************Exercice 1
\begin{question}
En rouge les variables libres. Les variables liées sont de la même couleur que le quantificateur qui les lie.

\begin{itemize}
\item $f(\textcolor{red}{x}, \textcolor{red}{y})$
\item $P(\textcolor{red}{x}) \wedge P(f(\textcolor{red}{y}, \textcolor{red}{x}))$
\item $\forall y. P(\textcolor{red}{x})$
\item $\textcolor{blue}{\forall} x.\ P(\textcolor{red}{y}, \textcolor{blue}{x}) \vee Q(\textcolor{red}{z})$
\item $\textcolor{blue}{\forall} x. (\textcolor{violet}{\exists} x. P(\textcolor{violet}{x}, \textcolor{violet}{x})) \wedge Q(\textcolor{blue}{x})$
\item $(\textcolor{blue}{\exists} x. P(\textcolor{blue}{x})) \wedge Q(\textcolor{red}{x}) \iff \textcolor{violet}{\forall} y. P(\textcolor{red}{x}, \textcolor{violet}{y})$
\end{itemize}
\end{question}


%****************Exercice 2
\begin{question}
On utilise le prédicat $\texttt{aime}$ d'arité deux. La formule $\texttt{aime}(x, y)$ indique que $x$ aime $y$.
\begin{enumerate}
\item $\forall x. \exists y.\ \texttt{aime}(x, y)$.
\item $\exists y. \forall x.\ \texttt{aime}(x, y)$.
\item $\forall x. (\neg \exists y. \texttt{aime}(y, x)) \implies \forall y. \neg \texttt{aime}(x, y)$
\item $\forall x. (\exists y.\ \texttt{aime}(x, y)) \implies \exists y.\ \texttt{aime}(y, x)$
\item $\forall x. \forall y.\ \texttt{aime}(x, y) \implies \texttt{aime}(y, x)$
\item $(\forall x. \exists y.\ \texttt{aime}(y, x)) \vee \exists x. \forall y.\ \neg \texttt{aime}(y, x)$
\item $\neg (\forall x.\ (\forall y.\ \texttt{aime}(y, x)) \implies \forall y. \texttt{aime}(x, y))$
\end{enumerate}
\end{question}

%****************Exercice 3
\begin{question}
\begin{enumerate}
\item

\[
\infer[(\Rightarrow{}I)]{(\forall x.\ Q(x)) \implies (\exists x.\ Q(x))}{
\infer[(\exists I)]{\exists x.\ Q(x)}{
\infer[(\forall E)]{Q(C)}{[\forall x.\ Q(x)]}
}
}
\]
\item
Considérons l'univers $\{\ 1\ \}$, avec l'interprétation $\mathcal{I}$ telle que:
\begin{align*}
\mathcal{I}(C) &\eqdef 1\\
\mathcal{I}(P) &\eqdef \emptyset\\
\mathcal{I}(Q) &\eqdef \emptyset\\
\mathcal{I}(R) &\eqdef \emptyset
\end{align*}
L'interprétation $\mathcal{I}$ évalue la formule à $0$, et est donc un contre-exemple.
\item La même interprétation $\mathcal{I}$ qu'au point précédent est un contre-exemple.
\item
\[
\infer[(\Rightarrow{}I)]{(\forall x.\ R(x, x)) \implies (\forall y. \exists z.\ R(y, z))}{
\infer[(\forall I)]{\forall y. \exists z.\ R(y, z)}{
\infer[(\exists I)]{\exists z.\ R(y, z)}{
\infer[(\forall E)]{R(y, y)}{[\forall x.\ R(x, x)]}
}
}
}
\]
\item
Considérons l'univers $\{\ 1, 2\ \}$, avec l'interprétation $\mathcal{I}$ telle que:
\begin{align*}
\mathcal{I}(C) &\eqdef 1\\
\mathcal{I}(P) &\eqdef \emptyset\\
\mathcal{I}(Q) &\eqdef \emptyset\\
\mathcal{I}(R) &\eqdef \{\ (1, 1), (2, 2)\ \}
\end{align*}
L'interprétation $\mathcal{I}$ évalue la formule à $0$, et est donc un contre-exemple.
\end{enumerate}
\end{question}

%****************Exercice 4
\begin{question}
\paragraph{Partie 1}
La théorie a pour prédicats unaires $\texttt{homme}$ et $\texttt{mortel}$, et comme constante $\texttt{socrate}$.
Elle admet comme axiomes:
\begin{enumerate}
\item $\forall x.\ \texttt{homme}(x) \implies \texttt{mortel}(x)$
\item $\texttt{homme}(\texttt{socrate})$
\end{enumerate}

\paragraph{Partie 2}

Considérons, par exemple, l'univers $\{ 1, 2 \}$, et une interprétation $\mathcal{I}$ telle que:
\begin{align*}
\mathcal{I}(\texttt{socrate}) &\eqdef 1\\
\mathcal{I}(\texttt{mortel}) &\eqdef \{\ 1, 2\ \}\\
\mathcal{I}(\texttt{homme}) &\eqdef \{\ 1\ \}\\
\end{align*}

\paragraph{Partie 3}

\[
\infer[(\Rightarrow{}E)]{\texttt{mortel}(\texttt{socrate})}{
\infer[(\forall E)]{\texttt{homme}(\texttt{socrate}) \implies \texttt{mortel}(\texttt{socrate})}{
\infer[]{\forall x.\ \texttt{homme}(x) \implies \texttt{mortel}(x)}{}
} & \infer[]{\texttt{homme}(\texttt{socrate})}{}}
\]

\paragraph{Partie 4}

La proposition $\forall x.\ \texttt{mortel}(x)$ est \textit{indépendante} de la théorie.
\end{question}

\begin{question}
\paragraph{Partie 1}
Considérons la théorie avec un seul prédicat binaire $\texttt{rase}$ et un seul axiome:
\[
\exists x. \forall y. \neg \texttt{rase}(y, y) \iff \texttt{rase}(x, y)
\]

\paragraph{Partie 2}
Montrons qu'il est possible de dériver le théorème $\vdash \bot$ dans la théorie. En utilisant la règle $(\bot E)$, il sera donc possible de montrer n'importe quelle proposition.

Soit $x$ l'homme dont l'existence est postulée par l'axiome. On a donc que:
\[
\forall y. \neg \texttt{rase}(y, y) \iff \texttt{rase}(x, y)
\]
De plus, en appliquant cette proposition universelle à $x$:
\[
\neg \texttt{rase}(x, x) \iff \texttt{rase}(x, x)
\]
D'après le principe du tiers exclu, soit $\texttt{rase}(x, x)$, soit $\neg \texttt{rase}(x, x)$. Montrons une contradiction dans chaque cas:
\begin{itemize}
\item Dans le cas où $\texttt{rase}(x, x)$, par hypothèse on a que $\texttt{rase}(x, x) \implies \neg \texttt{rase}(x, x)$. On peut donc conclure à fois $\texttt{rase}(x, x)$ et $\neg \texttt{rase}(x, x)$, et donc on peut montrer $\bot$.
\item Idem dans le cas où $\neg \texttt{rase}(x, x)$. Par hypothèse on a que $\neg \texttt{rase}(x, x) \implies \texttt{rase}(x, x)$, et donc on peut aussi conclure $\bot$.
\end{itemize}

En utilisant la règle du tiers exclu (TND), on pourrait construire explicitement cette arbre de dérivation pour montrer formellement la contradiction. L'arbre est donné à la page suivante.

Le principe du tiers exclu avait été démontré dans la série précédente.
On considère donc qu'il peut être utilisé sans répéter la preuve à chaque fois.
Répéter la preuve reviendrait à l'incorporer directement dans l'arbre, ce qui est tout à fait possible de faire.
Par exemple, voir le second arbre de la page suivante.
\newpage

\begin{center}
\begin{tabular}{ccc}
\rotatebox{90}{
\resizebox{26cm}{!} 
{
$
\infer[(\exists E)]{\bot}{
\infer[]{\exists x. \forall y. \neg \texttt{rase}(y, y) \iff \texttt{rase}(x, y)}{} &
\infer[(\vee E)]{\bot}{
\infer[(TND)]{\texttt{rase}(x, x) \vee \neg\texttt{rase}(x, x)}{}
&
\infer[(\Rightarrow{}E)]{\bot}{
\infer[(\Rightarrow{}E)]{\neg\texttt{rase}(x, x)}{
\infer[(\wedge E \text{ droite})]{\texttt{rase}(x, x) \implies \neg \texttt{rase}(x, x)}{
\infer[(\forall E)]{\neg\texttt{rase}(x, x) \iff \texttt{rase}(x, x)}{[\forall y. \neg \texttt{rase}(y, y) \iff \texttt{rase}(x, y)]}
} &
[\texttt{rase}(x, x)]
}
&
[\texttt{rase}(x, x)]}
&
\infer[(\Rightarrow{}E)]{\bot}{
[\neg\texttt{rase}(x, x)] &
\infer[(\Rightarrow{}E)]{\texttt{rase}(x, x)}{
\infer[(\wedge E \text{ gauche})]{\neg\texttt{rase}(x, x) \implies \texttt{rase}(x, x)}{
\infer[(\forall E)]{\neg\texttt{rase}(x, x) \iff \texttt{rase}(x, x)}{[\forall y. \neg \texttt{rase}(y, y) \iff \texttt{rase}(x, y)]}
} &
[\neg\texttt{rase}(x, x)]
}
}
}
}
$
}
}
&
\hspace{3cm}
&
\rotatebox{90}{
\resizebox{26cm}{!} 
{
$
\infer[(\exists E)]{\bot}{
\infer[]{\exists x. \forall y. \neg \texttt{rase}(y, y) \iff \texttt{rase}(x, y)}{} &
\infer[(\vee E)]{\bot}{
\infer[(\neg \neg E)]{\texttt{rase}(x, x) \vee (\texttt{rase}(x, x) \implies \bot)}{
\infer[(\Rightarrow{}I)]{\Big( \big(\texttt{rase}(x, x) \vee (\texttt{rase}(x, x) \implies \bot) \big) \implies \bot \Big) \implies \bot}{
\infer[(\Rightarrow{}E)]{\bot}{[\big(\texttt{rase}(x, x) \vee (\texttt{rase}(x, x) \implies \bot) \big) \implies \bot] & 
\infer[(\vee I \text{ droite})]{\texttt{rase}(x, x) \vee (\texttt{rase}(x, x) \implies \bot)}{
\infer[(\Rightarrow{}I)]{\texttt{rase}(x, x) \implies \bot}{
\infer[(\Rightarrow{}E)]{\bot}{[\big(\texttt{rase}(x, x) \vee (\texttt{rase}(x, x) \implies \bot) \big) \implies \bot] & 
\infer[(\vee I \text{ gauche})]{\texttt{rase}(x, x) \vee (\texttt{rase}(x, x) \implies \bot)}{[\texttt{rase}(x, x)]}
}
}
}
}
}
}
&
\infer[(\Rightarrow{}E)]{\bot}{
\infer[(\Rightarrow{}E)]{\neg\texttt{rase}(x, x)}{
\infer[(\wedge E \text{ droite})]{\texttt{rase}(x, x) \implies \neg \texttt{rase}(x, x)}{
\infer[(\forall E)]{\neg\texttt{rase}(x, x) \iff \texttt{rase}(x, x)}{[\forall y. \neg \texttt{rase}(y, y) \iff \texttt{rase}(x, y)]}
} &
[\texttt{rase}(x, x)]
}
&
[\texttt{rase}(x, x)]}
&
\infer[(\Rightarrow{}E)]{\bot}{
[\neg\texttt{rase}(x, x)] &
\infer[(\Rightarrow{}E)]{\texttt{rase}(x, x)}{
\infer[(\wedge E \text{ gauche})]{\neg\texttt{rase}(x, x) \implies \texttt{rase}(x, x)}{
\infer[(\forall E)]{\neg\texttt{rase}(x, x) \iff \texttt{rase}(x, x)}{[\forall y. \neg \texttt{rase}(y, y) \iff \texttt{rase}(x, y)]}
} &
[\neg\texttt{rase}(x, x)]
}
}
}
}
$
}
}
\end{tabular}
\end{center}
\end{question}

\newpage

\begin{question}

\paragraph{Partie 1}

\begin{enumerate}
\item Montrons $\texttt{Pres} \vdash \forall x.\ 0 + x = x$ par induction sur $x$. Pour le cas $0$, on doit montrer $0 + 0 = 0$, ce qui est immédiat par l'axiome 3. Dans le cas $x + 1$, on doit montrer:
\[
0 + (x + 1) = x + 1
\]
Or, par l'axiome 4, on a que:
\[
0 + (x + 1) = (0 + x) + 1
\]
De plus, par hypothèse d'induction ($0 + x = x$), on a que:
\[
0 + (x + 1) = x + 1
\]
Ce qui conclut le cas $x + 1$ et la preuve.
\item Montrons $\texttt{Pres} \vdash \forall x. \forall y.\ x + (y + 1) = (x + 1) + y$ par induction sur $y$.
Le cas $0$ est immédiat par le résultat du point précédent et par l'axiome 3.
Considérons le cas $y + 1$. On doit montrer:
\[
x + ((y + 1) + 1) = (x + 1) + (y + 1)
\]
Avec comme hypothèse d'induction:
\[
x + (y + 1) = (x + 1) + y
\]
Par l'axiome 4, on a que:
\[
x + ((y + 1) + 1) = (x + (y + 1)) + 1
\]
Par hypothèse d'induction, on a donc que:
\[
x + ((y + 1) + 1) = ((x + 1) + y) + 1
\]
En appliquant encore une fois l'axiome 4, on obtient:
\[
x + ((y + 1) + 1) = (x + 1) + (y + 1)
\]
Ce qui conclut le cas inductif et la preuve.
\item Montrons $\texttt{Pres} \vdash \forall x. \forall y.\ x + y = y + x$ par induction sur $y$.
Le cas $0$ est immédiat en appliquant le résultat du point 1 et l'axiome 3.
Considérons le cas $y + 1$. On doit montrer:
\[
x + (y + 1) = (y + 1) + x
\]
Avec comme hypothèse d'induction:
\[
x + y = y + x
\]
Par l'axiome 4, on a que:
\[
x + (y + 1) = (x + y) + 1
\]
Par hypothèse d'induction, on a donc:
\[
x + (y + 1) = (y + x) + 1
\]
Puis, par l'axiome 4:
\[
x + (y + 1) = y + (x + 1)
\]
Finalement, en appliquant le résultat du point 2:
\[
x + (y + 1) = (y + 1) + x
\]
Ce qui conclut le cas inductif et la preuve.
\end{enumerate}

\end{question}


\end{document}
