% \documentclass[12pt,french,a4paper]{article}
\documentclass[12pt]{scrartcl}  % mise en page "européenne" (la classe article
                                % est adapté aux USA à la base) - gère bien la page
% \usepackage{geometry}
%  \geometry{
%  a4paper,
%  total={170mm,257mm},
%  left=20mm,
%  top=20mm,
%  }
\usepackage[utf8]{inputenc}
\usepackage[T1]{fontenc}  % recommandé pour français
\usepackage{lmodern}
\usepackage[frenchb]{babel}
\usepackage{exsheets}
\usepackage{amsmath}
\usepackage{amssymb}
\usepackage{mathtools}
\usepackage{proof}
\usepackage{hyperref}
\usepackage{cleveref}
\usepackage{tikz}
\usetikzlibrary{automata, arrows.meta, positioning}

\newenvironment{automate}{
  % référence: https://latexdraw.com/automata-diagrams-in-latex/
  \begin{tikzpicture}[node distance=3cm, on grid, auto,
    every path/.style={-stealth}, % flèches pour transitions
    every loop/.style={stealth-}, % boucle=revenir "en arrière"
    ]
  }{
  \end{tikzpicture}
}
\newcommand\autoitem{\item\phantom+\par\vspace{-2ex}} % éviter que la figure
                                % soit au-dessus du numéro de sous-question

\begin{document}
\title{\vspace{-2cm}Réponses à la série d'exercices n°5\\\large{Fondamentaux formels / Informatique théorique\\GymInf}}
\date{\vspace{-1cm}17 août 2021}

\maketitle

%****************Exercice 1
\begin{question}
  
\begin{enumerate}
\autoitem\begin{automate}
  \node (e1) [state, initial, initial text={}, accepting] {1};
  \path (e1) edge[loop above] node{$a$}  (e1);
\end{automate}

\autoitem\begin{automate}
  \node (e1) [state, initial, initial text={}] {1};
  \node (e2) [state, right=of e1] {2};
  \node (e3) [state, accepting, right=of e2] {3};
  \path (e1) edge node{$a$}  (e2);
  \path (e2) edge node{$b$}  (e3);
\end{automate}

\autoitem\begin{automate}
  \node (e1) [state, initial, initial text={}] {1};
  \node (e2) [state, right=of e1] {2};
  \node (e3) [state, accepting, right=of e2] {3};
  \path (e1) edge node{$a$}  (e2);
  \path (e2) edge node{$a$}  (e3);
  \path (e3) edge[loop right] node{$a$} (e3);
\end{automate}

\autoitem\begin{automate}
  \node (e1) [state, initial, initial text={}] {1};
  \node (e2) [state, accepting, right=of e1] {2};
  \path (e1) edge[bend left] node{$a$} (e2);
  \path (e2) edge[bend left] node{$b$} (e1);
\end{automate}

\item
  Il est plus facile de commencer par un automate fini non déterministe, par exemple:
\begin{automate}
  \node (e1) [state, initial, initial text={}] {1};
  \node (e2) [state, right=of e1] {2};
  \node (e3) [state, above right=of e2] {3};
  \node (e4) [state, accepting, right=of e2] {4};
  \path (e1) edge[bend left] node{$a$} (e2);
  \path (e2) edge[bend left] node{$a,b$} (e1);
  \path (e2) edge[bend left] node{$b$} (e3);
  \path (e2) edge[] node{$a$} (e4);
  \path (e3) edge[bend left] node{$a$} (e4);
\end{automate}

On procède ensuite à la transformation vers un automate déterministe en
définissant des nouveaux états comme des ensembles des états de l'automate
précédent. Si un état acceptant est présent dans l'ensemble, le nouvel état est
acceptant.

\begin{automate}
  \node (e1) [state, initial, initial text={}] {$\{1\}$};
  \node (e2) [state, right=of e1] {$\{2\}$};
  \node (e3) [state, accepting, above right=of e2] {$\{1,4\}$};
  \node (e4) [state, below right=of e2] {$\{1,3\}$};
  \node (e5) [state, accepting, below right=of e3] {$\{2,4\}$};
  \path (e1) edge node{$a$} (e2);
  \path (e2) edge[bend left] node{$a$} (e3);
  \path (e3) edge[bend left] node{$a$} (e2);
  \path (e2) edge[bend right] node{$b$} (e4);
  \path (e4) edge[bend left] node{$a$} (e5);
  \path (e5) edge[bend left] node{$b$} (e4);
  \path (e5) edge[bend right] node{$a$} (e3);
\end{automate}

Enfin, on renomme les nouveaux états:

\begin{automate}
  \node (e1) [state, initial, initial text={}] {1};
  \node (e2) [state, right=of e1] {2};
  \node (e3) [state, accepting, above right=of e2] {3};
  \node (e4) [state, below right=of e2] {4};
  \node (e5) [state, accepting, below right=of e3] {5};
  \path (e1) edge node{$a$} (e2);
  \path (e2) edge[bend left] node{$a$} (e3);
  \path (e3) edge[bend left] node{$a$} (e2);
  \path (e2) edge[bend right] node{$b$} (e4);
  \path (e4) edge[bend left] node{$a$} (e5);
  \path (e5) edge[bend left] node{$b$} (e4);
  \path (e5) edge[bend right] node{$a$} (e3);
\end{automate}

  
\end{enumerate}
\end{question}

\end{document}
