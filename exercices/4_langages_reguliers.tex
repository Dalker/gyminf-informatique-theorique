\documentclass[12pt,french,a4paper]{article}

\usepackage{geometry}
 \geometry{
 a4paper,
 total={170mm,257mm},
 left=20mm,
 top=20mm,
 }
\usepackage[frenchb]{babel}
\usepackage{exsheets}
\usepackage{amsmath}
\usepackage{amssymb}
\usepackage{mathtools}
\usepackage{proof}
\usepackage{logicpuzzle}
\usepackage{hyperref}
\usepackage{cleveref}

% Définition de la commande pour le signe = avec "déf" aussi dessus.
\newcommand\eqdef{\mathrel{\overset{\makebox[0pt]{\mbox{\normalfont\tiny\sffamily déf}}}{=}}}

\begin{document}

\title{\vspace{-2cm}Série d'exercices n°4\\\large{Fondamentaux formels / Informatique théorique\\GymInf}}
\date{\vspace{-1cm}10 août 2021}

\maketitle

\begin{question}

\paragraph{Partie 1}
Prouvez que pour tout nombre naturel $i$, langage $L$ et mot \textit{non-vide} $x$,
si $x \in L^{i + 1}$, alors soit il existe une décomposition de $x$ en un mot \textit{non-vide} $x_1$ de $L$ et un mot $x_2$ de $L^i$, soit $x$ est un mot de $L^i$.

\paragraph{Partie 2}
Prouvez par induction naturelle que tout nombre naturel $i$, langage $L$ et mot \textit{non-vide} $x$,
si $x \in L^{i + 1}$, alors il existe une décomposition de $x$ en un mot \textit{non-vide} $x_1$ de $L$ et un mot $x_2$ de $L^i$.

\paragraph{Partie 3}
Concluez que pour tout langage $L$ et mot \textit{non-vide} $x$,
si $x \in L^*$, alors il existe une décomposition de $x$ en un mot \textit{non-vide} $x_1$ de $L$ et un mot $x_2$ de $L^*$.
Observez que le mot $x_2$ a une taille plus petite que $x$.
\end{question}

\begin{question}
Rappeler vous la relation de dérivation en une étape $\vdash_A$ d'un automate fini déterministe $A = (Q, \Sigma, \delta, s, F)$.
Montrez le lemme suivant:
\[
\forall a. \forall q_1. \forall q_2. \forall x_1. \forall x_2.\ (q_1, a \cdot x_1) \vdash_A (q_2, x_1) \iff  (q_1, a \cdot x_2) \vdash_A (q_2, x_2)
\]

Étendez ensuite ce résultat à la relation de dérivation $\vdash_A^*$.
En d'autres termes, montrez le lemme suivant:
\[
\forall x_0. \forall q_1. \forall q_2. \forall x_1. \forall x_2.\ (q_1, x_0 \cdot x_1) \vdash_A^* (q_2, x_1) \iff  (q_1, x_0 \cdot x_2) \vdash_A^* (q_2, x_2)
\]

\paragraph{Indice} Utilisez le fait que, pour tout $q_i$, $q_j$, $a$, $x$ et $y$:
\begin{gather*}
(q_i, a \cdot x \cdot y) \vdash_A^* (q_j, y)\\
\iff\\
\exists q_k. (q_i, a \cdot x \cdot y) \vdash_A (q_k, x \cdot y) \wedge (q_k, x \cdot y) \vdash_A^* (q_j, \cdot y)
\end{gather*}

\paragraph{Indice} Procédez par induction sur la taille de $x_0$, et essayez de rester général quant au reste des variables universellement quantifiées.

\end{question}

\begin{question}
Pour cet exercice, vous allez construire une autre preuve du théorème du gonflement.
À la place d'une preuve se basant sur les automates finis déterministes, la preuve que vous allez développer reposera uniquement sur les expressions régulières et ne fera pas appel aux automates.

Pour rappel, le théorème du gonflement stipule que si un langage $L$ est régulier, alors:
\begin{align*}
\exists p. p \geq 1 \wedge \forall x.\ |x| \geq p \implies \exists x_1. \exists x_2. \exists x_3.\ &x = x_1 \cdot x_2 \cdot x_3\ \wedge\\
&|x_1 \cdot x_2| \leq p\ \wedge\\
&|x_2| \geq 1\ \wedge\\
&\forall k.\ x_1 \cdot x_2^k \cdot x_3 \in L
\end{align*}

Prouvez le théorème du gonflement par induction structurelle sur les expressions régulières.
Partez du postulat qu'un langage est régulier si et seulement si il est le langage d'une expression régulière.

\paragraph{Indice} Pour chaque constructeur d'expression régulière, montrez qu'il existe un nombre $p \geq 1$ qui satisfait la propriété.

\paragraph{Indice} Dans les cas où le langage de l'expression considérée est fini, il est possible de choisir un $p$ suffisamment grand pour qu'il n'y aie aucun mot de taille plus grande ou égal à $p$ dans le langage de l'expression.

\paragraph{Indice} Dans les cas inductifs, utilisez le fait que vous avez à votre disposition une variable de gonflement $p_i$ pour chacune des sous-expressions directe $e_i$.

\paragraph{Indice} Pour le cas de la fermeture inductive, utilisez le lemme suivant que vous avez prouvé précédemment:
\[
\forall L. \forall x.\ x \in L^* \wedge x \neq \epsilon \implies \exists x_1. \exists x_2.\ x = x_1 \cdot x_2 \wedge x_1 \neq \epsilon \wedge x_1 \in L \wedge x_2 \in L^* 
\]
\end{question}

\end{document}