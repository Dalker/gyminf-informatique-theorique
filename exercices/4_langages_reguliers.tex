\documentclass[12pt,french,a4paper]{article}

\usepackage[T1]{fontenc}
\usepackage[utf8]{inputenc}
\usepackage{geometry}
 \geometry{
 a4paper,
 total={170mm,257mm},
 left=20mm,
 top=20mm,
 }
\usepackage[frenchb]{babel}
\usepackage{exsheets}
\usepackage{amsmath}
\usepackage{amssymb}
\usepackage{mathtools}
\usepackage{proof}
\usepackage{logicpuzzle}
\usepackage{hyperref}
\usepackage{cleveref}

% Définition de la commande pour le signe = avec "déf" aussi dessus.
\newcommand\eqdef{\mathrel{\overset{\makebox[0pt]{\mbox{\normalfont\tiny\sffamily déf}}}{=}}}

\begin{document}

\title{\vspace{-2cm}Série d'exercices n°4\\\large{Fondamentaux formels / Informatique théorique\\GymInf}}
\date{\vspace{-1cm}12 août 2021}

\maketitle

\begin{question}
Considérez l'alphabet $\Sigma = \{\ \texttt{a}, \texttt{b}, \texttt{c}\ \}$.
Donnez des exemples de mots appartenant au langage des expressions régulières suivantes.
Dans le cas d'un langage fini, donnez tous les mots. Pour un langage de taille infinie, donnez au moins 3 exemples.
\begin{enumerate}
\item $\texttt{a}$
\item $\texttt{a} \cdot \texttt{b}$
\item $\texttt{a} \cup \texttt{b}$
\item $(\texttt{a} \cup \texttt{b}) \cdot (\texttt{a} \cup \texttt{c})$
\item $\epsilon$
\item $\emptyset$
\item $\texttt{a}^* \cup \texttt{b}^*$
\item $(\texttt{a} \cup \texttt{b})^*$
\item $\epsilon^*$
\item $\emptyset^*$
\end{enumerate}
\end{question}

\begin{question}
Considérez l'alphabet $\Sigma = \{\ \texttt{a}, \texttt{b}, \texttt{c}\ \}$.
Pour chaque langage suivant, donnez une expression régulière qui décrit le langage:
\begin{enumerate}
\item Le langage de tous les mots qui commencent par un $\texttt{a}$.
\item Le langage de tous les mots qui commencent par un $\texttt{a}$, suivit d'au moins deux $\texttt{b}$ et finalement d'un $\texttt{c}$.
\item Le langage de tous les mots qui commencent par un $\texttt{a}$ et finissent par un $\texttt{c}$.
\item Le langage de tous les mots qui commencent par un $\texttt{a}$ et finissent par un $\texttt{a}$.
\item Le langage de tous les mots qui contiennent au moins un $\texttt{a}$.
\item Le langage de tous les mots qui contiennent au moins un $\texttt{a}$ ou un $\texttt{b}$.
\item Le langage de tous les mots qui contiennent au moins un $\texttt{a}$ et un $\texttt{b}$ (\texttt{a} et \texttt{b} peuvent apparaître dans n'importe quel ordre).
\item Le langage de tous les mots de taille paire.
\end{enumerate}
\end{question}

\begin{question}

\paragraph{Partie 1}
Prouvez que pour tout nombre naturel $i$, langage $L$ et mot \textit{non-vide} $x$,
si $x \in L^{i + 1}$, alors soit il existe une décomposition de $x$ en un mot \textit{non-vide} $x_1$ de $L$ et un mot $x_2$ de $L^i$, soit $x$ est un mot de $L^i$.

\paragraph{Partie 2}
Prouvez par induction naturelle que tout nombre naturel $i$, langage $L$ et mot \textit{non-vide} $x$,
si $x \in L^{i + 1}$, alors il existe une décomposition de $x$ en un mot \textit{non-vide} $x_1$ de $L$ et un mot $x_2$ de $L^i$.

\paragraph{Partie 3}
Concluez que pour tout langage $L$ et mot \textit{non-vide} $x$,
si $x \in L^*$, alors il existe une décomposition de $x$ en un mot \textit{non-vide} $x_1$ de $L$ et un mot $x_2$ de $L^*$.
Observez que le mot $x_2$ a une taille plus petite que $x$.
\end{question}

\begin{question}
Pour cet exercice, vous allez prouver le \textit{théorème du gonflement} des langages réguliers.
Le théorème du gonflement stipule que si un langage $L$ est régulier, alors:
\begin{align*}
\exists p. p \geq 1 \wedge \forall x.\ |x| \geq p \implies \exists x_1. \exists x_2. \exists x_3.\ &x = x_1 \cdot x_2 \cdot x_3\ \wedge\\
&|x_1 \cdot x_2| \leq p\ \wedge\\
&|x_2| \geq 1\ \wedge\\
&\forall k.\ x_1 \cdot x_2^k \cdot x_3 \in L
\end{align*}

En d'autres terme, les mots du langage d'une expression régulière, à partir d'une certaine taille $p$ (plus grande ou égal à 1, et qui dépend uniquement de l'expression régulière), tous les mots contiennent une séquence non-vide de symboles qui peut être répétée un nombre arbitraire de fois tout en préservant l'appartenance au langage.
Le théorème spécifie en plus que cette séquence apparaît entièrement dans les $p$ premiers symboles du mot. 

\paragraph{Objectif} Prouvez le théorème du gonflement par \textbf{induction structurelle} sur les expressions régulières.
Partez du postulat qu'un langage est régulier si et seulement si il est le langage d'une expression régulière.

\paragraph{Indice} Pour chaque constructeur d'expression régulière, montrez qu'il existe un nombre $p \geq 1$ qui satisfait la propriété.

\paragraph{Indice} Dans les cas où le langage de l'expression considérée est fini, il est possible de choisir un $p$ suffisamment grand pour qu'il n'y aie aucun mot de taille plus grande ou égal à $p$ dans le langage de l'expression.

\paragraph{Indice} Dans les cas inductifs, utilisez le fait que vous avez à votre disposition une variable de gonflement $p_i$ pour chacune des sous-expressions directe $e_i$.

\paragraph{Indice} Pour le cas de la fermeture inductive, utilisez le lemme suivant que vous avez prouvé précédemment:
\[
\forall L. \forall x.\ x \in L^* \wedge x \neq \epsilon \implies \exists x_1. \exists x_2.\ x = x_1 \cdot x_2 \wedge x_1 \neq \epsilon \wedge x_1 \in L \wedge x_2 \in L^* 
\]
\end{question}

\end{document}