\documentclass[12pt,french,a4paper]{article}

\usepackage{geometry}
 \geometry{
 a4paper,
 total={170mm,257mm},
 left=20mm,
 top=20mm,
 }
\usepackage[frenchb]{babel}
\usepackage{exsheets}
\usepackage{amsmath}
\usepackage{amssymb}
\usepackage{mathtools}
\usepackage{proof}
\usepackage{logicpuzzle}
\usepackage{hyperref}
\usepackage{cleveref}

% Définition de la commande pour le signe = avec "déf" aussi dessus.
\newcommand\eqdef{\mathrel{\overset{\makebox[0pt]{\mbox{\normalfont\tiny\sffamily déf}}}{=}}}

\begin{document}

\title{\vspace{-2cm}Série d'exercices n°5\\\large{Fondamentaux formels / Informatique théorique\\GymInf}}
\date{\vspace{-1cm}13 août 2021}

\maketitle

\begin{question}
Rappeler vous la relation de dérivation en une étape $\vdash_A$ d'un automate fini déterministe $A = (Q, \Sigma, \delta, s, F)$.
Montrez le lemme suivant:
\[
\forall a. \forall q_1. \forall q_2. \forall x_1. \forall x_2.\ (q_1, a \cdot x_1) \vdash_A (q_2, x_1) \iff  (q_1, a \cdot x_2) \vdash_A (q_2, x_2)
\]

Étendez ensuite ce résultat à la relation de dérivation $\vdash_A^*$.
En d'autres termes, montrez le lemme suivant:
\[
\forall x_0. \forall q_1. \forall q_2. \forall x_1. \forall x_2.\ (q_1, x_0 \cdot x_1) \vdash_A^* (q_2, x_1) \iff  (q_1, x_0 \cdot x_2) \vdash_A^* (q_2, x_2)
\]

\paragraph{Indice} Utilisez le fait que, pour tout $q_i$, $q_j$, $a$, $x$ et $y$:
\begin{gather*}
(q_i, a \cdot x \cdot y) \vdash_A^* (q_j, y)\\
\iff\\
\exists q_k. (q_i, a \cdot x \cdot y) \vdash_A (q_k, x \cdot y) \wedge (q_k, x \cdot y) \vdash_A^* (q_j, \cdot y)
\end{gather*}

\paragraph{Indice} Procédez par induction sur la taille de $x_0$, et essayez de rester général quant au reste des variables universellement quantifiées.

\end{question}

\end{document}