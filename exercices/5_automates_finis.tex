\documentclass[12pt,french,a4paper]{article}

\usepackage[T1]{fontenc}
\usepackage[utf8]{inputenc}
\usepackage{geometry}
 \geometry{
 a4paper,
 total={170mm,257mm},
 left=20mm,
 top=20mm,
 }
\usepackage[frenchb]{babel}
\usepackage{exsheets}
\usepackage{amsmath}
\usepackage{amssymb}
\usepackage{mathtools}
\usepackage{proof}
\usepackage{logicpuzzle}
\usepackage{hyperref}
\usepackage{cleveref}

% Définition de la commande pour le signe = avec "déf" aussi dessus.
\newcommand\eqdef{\mathrel{\overset{\makebox[0pt]{\mbox{\normalfont\tiny\sffamily déf}}}{=}}}

\begin{document}

\title{\vspace{-2cm}Série d'exercices n°5\\\large{Fondamentaux formels / Informatique théorique\\GymInf}}
\date{\vspace{-1cm}13 août 2021}

\maketitle

\begin{question}
Étant donné l'alphabet $\Sigma = \{\ \texttt{a}, \texttt{b}\ \}$, donnez un automate fini déterministe pour les expressions régulières suivantes:
\begin{enumerate}
\item $\texttt{a}^*$
\item $\texttt{a} \cdot \texttt{b}$
\item $\texttt{a} \cdot \texttt{a} \cdot \texttt{a}^*$
\item $\texttt{a} \cdot (\texttt{b} \cdot \texttt{a})^*$
\item $\texttt{a} \cdot (\texttt{a} \cup \texttt{b} \cdot \texttt{a})^* \cdot (\texttt{b} \cup \epsilon) \cdot \texttt{a}$
\end{enumerate}
\end{question}

\vspace{1cm}

\begin{question}
\paragraph{Partie 1}
Étant donné l'alphabet $\Sigma = \{\ \texttt{a}, \texttt{b}\ \}$, construisez un automate fini non-déterministe pour le langage de tous les mots qui contiennent la séquence $\texttt{abba}$ ou la séquence $\texttt{baba}$.

\paragraph{Partie 2}
Déterminisez l'automate obtenu dans la première partie.

\paragraph{Partie 3}
Minimalisez l'automate obtenu dans la deuxième partie.
\end{question}

\vspace{1cm}

\begin{question}
Étant donné l'alphabet $\Sigma = \{\ \texttt{0}, \texttt{1}\ \}$, donnez un automate fini déterministe qui accepte la représentation binaire de tous les entiers multiples de trois. On considère que le \textit{bit} de poids le plus fort débute le mot.

À quoi ressemblerait l'automate si les bits étaient lus dans l'ordre inverse, c'est-à-dire du bit de poids le plus faible au bit de poids le plus fort ?
\end{question}

\vspace{1cm}

\begin{question}
Considérez l'opération d'inversion d'un mot $m$, que l'on note $m^R$:
\begin{align*}
|m^R| &\eqdef |m|\\
m^R(i) &\eqdef m(|m| + 1 - i)
\end{align*}
Étant donné un alphabet $\Sigma$ et un langage \textit{régulier} $L$, montrez que le langage qui consiste en l'inversion des mots de $L$, que l'on note $L^R$, est régulier:
\[
L^R \eqdef \{\ m^R \in \Sigma^*\ |\ m \in L\  \}
\]
\end{question}

\newpage

\begin{question}
Étant donné l'alphabet $\Sigma = \{\ \texttt{I}\ \}$, montrez que le langage suivant n'est pas régulier:
\[
\{\ m \in \Sigma^*\ |\ |m| \text{ est un nombre premier}\ \}
\]
\paragraph{Rappel}
Un nombre $n$ est premier si et seulement si il n'admet qu'exactement deux diviseurs: $1$ et $n$.
\end{question}

\vspace{3cm}

\begin{question}
Rappeler vous la relation de dérivation en une étape $\vdash_A$ d'un automate fini déterministe $A = (Q, \Sigma, \delta, s, F)$.
Montrez le lemme suivant:
\[
\forall a. \forall q_1. \forall q_2. \forall x_1. \forall x_2.\ (q_1, a \cdot x_1) \vdash_A (q_2, x_1) \iff  (q_1, a \cdot x_2) \vdash_A (q_2, x_2)
\]

Étendez ensuite ce résultat à la relation de dérivation $\vdash_A^*$.
En d'autres termes, montrez le lemme suivant:
\[
\forall x_0. \forall q_1. \forall q_2. \forall x_1. \forall x_2.\ (q_1, x_0 \cdot x_1) \vdash_A^* (q_2, x_1) \iff  (q_1, x_0 \cdot x_2) \vdash_A^* (q_2, x_2)
\]

\paragraph{Indice} Utilisez le fait que, pour tout $q_i$, $q_j$, $a$, $x$ et $y$:
\begin{gather*}
(q_i, a \cdot x \cdot y) \vdash_A^* (q_j, y)\\
\iff\\
\exists q_k. (q_i, a \cdot x \cdot y) \vdash_A (q_k, x \cdot y) \wedge (q_k, x \cdot y) \vdash_A^* (q_j, \cdot y)
\end{gather*}

\paragraph{Indice} Procédez par induction sur la taille de $x_0$, et essayez de rester général quant au reste des variables universellement quantifiées.

\end{question}

\end{document}