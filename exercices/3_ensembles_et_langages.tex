\documentclass[12pt,french,a4paper]{article}

\usepackage[T1]{fontenc}
\usepackage[utf8]{inputenc}
\usepackage{geometry}
 \geometry{
 a4paper,
 total={170mm,257mm},
 left=20mm,
 top=20mm,
 }
\usepackage[frenchb]{babel}
\usepackage{exsheets}
\usepackage{amsmath}
\usepackage{amssymb}
\usepackage{mathtools}
\usepackage{proof}
\usepackage{logicpuzzle}
\usepackage{hyperref}
\usepackage{cleveref}

% Définition de la commande pour le signe = avec "déf" aussi dessus.
\newcommand\eqdef{\mathrel{\overset{\makebox[0pt]{\mbox{\normalfont\tiny\sffamily déf}}}{=}}}

\begin{document}

\title{\vspace{-2cm}Série d'exercices n°3\\\large{Fondamentaux formels / Informatique théorique\\GymInf}}
\date{\vspace{-1cm}11 août 2021}

\maketitle

\begin{question}
Montrez que le produit cartésien de $\mathbb{N}$ avec l'ensemble \{\ 0, 1\ \}, noté $\mathbb{N} \times \{\ 0, 1\ \}$, est dénombrable, c'est-à-dire:
\[
|\mathbb{N} \times \{\ 0, 1\ \}| = |\mathbb{N}|
\]
\end{question}

\begin{question}
\paragraph{Partie 1}
Considérons, pour chaque nombre naturel $i$, un ensemble \textit{fini} $S_i$.
Admettons que les ensembles $S_i$ sont mutuellement disjoints.
Montrez que l'ensemble suivant est dénombrable:
\[
\bigcup_{i \in \mathbb{N}} S_i
\]

\paragraph{Partie 2}
Montrez que $\mathbb{N} \times \mathbb{N}$ est dénombrable en utilisant le résultat de la partie 1.

\paragraph{Partie 3}
Montrez que l'ensemble suivant est dénombrable:
\[
\bigcup_{i \in \mathbb{N}} (\mathbb{N} \times \{\ i\ \})
\]
\end{question}

\begin{question}
Démontrez le théorème de Cantor en utilisant la technique de la diagonalisation.
Le théorème de Cantor énonce que, pour tout ensemble $x$:
\[
|x| < |\mathcal{P}(x)|
\]

\paragraph{Indice} Pour démontrer ce théorème, procédez en deux temps:
\begin{enumerate}
\item
Dans un premier temps, prouvez que $|x| \leq |\mathcal{P}(x)|$ en montrant qu'il existe une injection entre $x$ et $\mathcal{P}(x)$.
\item
Dans un second temps, prouvez que $|x| \neq |\mathcal{P}(x)|$ en montrant qu'il n'existe pas de bijection entre $x$ et $\mathcal{P}(x)$.
Pour cette seconde partie, partez de l'hypothèse qu'il existe une telle bijection $f$, et utilisez le procédé de la diagonalisation pour construire un sous-ensemble $y$ de $x$, tel que la paire $(f^{-1}(y), y)$ soit, de façon contradictoire, un élément de $f$ si et seulement si la paire n'est pas un élément de $f$.
\end{enumerate}
\end{question}

\begin{question}
Pour cet exercice, nous nous intéressons aux propriétés de la concaténation de langages, de l'exponentiation de langage et de la fermeture iterative.

\paragraph{Partie 1} Montrez que la concatenation de langages est associative, c'est-à-dire, pour tous langages $L_1$, $L_2$, $L_3$:
\[
L_1 \cdot (L_2 \cdot L_3) = (L_1 \cdot L_2) \cdot L_3
\]
Utilisez le fait que deux ensembles sont égaux si et seulement si ils contiennent les mêmes éléments. Vous pouvez aussi utiliser le fait que la concaténation de mots est associative.

\paragraph{Partie 2} Montrez la propriété suivante pour tout nombre naturel $i$, et ce par induction naturelle:
\[
L \cdot L^i = L^i \cdot L 
\]
Vous pouvez utiliser le fait que $\epsilon$ est l'élément neutre de la concaténation:
\[
\forall x. x \cdot \epsilon = x \wedge \epsilon \cdot x = x
\]

\paragraph{Induction naturelle} Le principe d'induction naturelle stipule que pour prouver une propriété pour tout nombre naturel $i$, il suffit de:
\begin{itemize}
\item prouver la propriété pour $0$,
\item prouver que la propriété pour $i$ implique la propriété pour $i + 1$.
\end{itemize}

\paragraph{Partie 3} Montrez la propriété suivante de l'exponentiation de langages pour tout langage $L$ et tous entiers naturels $i$ et $j$:
\[
L^{i + j} = L^i \cdot L^j
\]

\paragraph{Partie 4} Montrez la propriété suivante de l'exponentiation de langages pour tout langage $L$ et tous entiers naturels $i$ et $j$:
\[
L^{i * j} = (L^i)^j
\]

\paragraph{Partie 5} Montrez la propriété suivante pour tout langage $L$:
\[
(L^*)^* = L^*
\]
\end{question}

\begin{question}
Pour les affirmations suivantes sur les langages, donnez une preuve ou un contre-exemple:
\begin{enumerate}
\item $(L_1 \cup L_2)^* = L_1^* \cup L_2^*$
\item $(L_1 \cap L_2)^* = L_1^* \cap L_2^*$
\item $(L_1 \cdot L_2)^* = L_1^* \cdot L_2^*$
\item $\overline{L_1^*} = \overline{L_1}^*$
\item $(L_1 \cup L_2) \cdot L_3 = L_1 \cdot L_3 \cup L_2 \cdot L_3$
\item $(L_1 \cap L_2) \cdot L_3 = L_1 \cdot L_3 \cap L_2 \cdot L_3$
\item $(L_1 \cdot L_2)^* = (L_2 \cdot L_1)^*$
\item $(L_1 \cdot L_2)^* = L_1 \cdot (L_2 \cdot L_1)^* \cdot L_2 \cup \emptyset$
\end{enumerate}
\end{question}

\end{document}