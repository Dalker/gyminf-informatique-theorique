\documentclass[12pt,french,a4paper]{article}

\usepackage{geometry}
 \geometry{
 a4paper,
 total={170mm,257mm},
 left=20mm,
 top=20mm,
 }
\usepackage[frenchb]{babel}
\usepackage{exsheets}
\usepackage{amsmath}
\usepackage{amssymb}
\usepackage{mathtools}
\usepackage{proof}
\usepackage{logicpuzzle}
\usepackage{hyperref}
\usepackage{cleveref}

% Définition de la commande pour le signe = avec "déf" aussi dessus.
\newcommand\eqdef{\mathrel{\overset{\makebox[0pt]{\mbox{\normalfont\tiny\sffamily déf}}}{=}}}

\begin{document}

\title{\vspace{-2cm}Série d'exercices n°2\\\large{Fondamentaux formels / Informatique théorique\\GymInf}}
\date{\vspace{-1cm}10 août 2021}

\maketitle

\begin{question}

Dans les formules et termes suivants, indiquez les variables libres et les variables liées.
Pour chaque variable liée, indiquez le quantificateur (le plus proche) qui la lie.

\begin{enumerate}
\item $f(x, y)$
\item $P(x) \wedge P(f(y, x))$
\item $\forall y. P(x)$
\item $\forall x.\ P(y, x) \vee Q(z)$
\item $\forall x. (\exists x. P(x, x)) \wedge Q(x)$
\item $(\exists x. P(x)) \wedge Q(x) \iff \forall y. P(x, y)$
\end{enumerate}


\end{question}

\begin{question}

Donnez une formule de logique du premier ordre pour exprimer les phrases suivantes:
\begin{enumerate}
\item Tout le monde aime quelqu'un.
\item Il y a quelqu'un qui est aimé de tous.
\item Tout ceux qui ne sont pas aimé n'aiment personne.
\item On ne peut aimer que si quelqu'un nous aime.
\item On ne peut aimer une personne que si elle nous aime en retour.
\item Soit tout le monde est aimé, soit il y a une personne que tout le monde n'aime pas.
\item Ce n'est pas parce que tout le monde aime quelqu'un que cette personne aime tout le monde.
\end{enumerate}
\end{question}

\begin{question}

Considérez une théorie du premier ordre $T$ qui définit au moins un terme constant $C$, deux prédicats unaires $P$ et $Q$, et un prédicat binaire $R$.
Pour chacune des affirmations suivantes, prouvez ou donnez un contre-exemple:
\begin{enumerate}
\item $(\forall x.\ Q(x)) \implies (\exists x.\ Q(x))$
\item $(\forall x.\ P(x) \implies Q(x)) \implies (\exists x.\ P(x) \wedge Q(x))$
\item $(\forall x.\ P(x) \implies Q(x)) \implies (\exists x.\ P(x) \vee Q(x))$
\item $(\forall x.\ R(x, x)) \implies (\forall y. \exists z.\ R(y, z))$
\item $(\forall x.\ R(x, x)) \implies (\forall y.\ (\exists z.\ R(y, z)) \implies (\forall z.\ y = z))$
\end{enumerate}
\end{question}


\begin{question}

\paragraph{Partie 1}
Décrivez une théorie du premier ordre (un ensemble de constantes, fonctions, prédicats et axiomes) qui incorpore le fait que tout homme est mortel et que Socrate est un homme.

\paragraph{Partie 2}
Donnez un modèle (un univers et une interprétation des symboles qui respecte les axiomes) de votre théorie.

\paragraph{Partie 3}
Dans votre théorie, prouvez à l'aide d'un arbre de dérivation que Socrate est mortel.

\paragraph{Partie 4}
Dans votre théorie, montrez une proposition $A$ telle que ni $T \vdash A$, ni $T \vdash \neg A$ (si une telle proposition existe).
\end{question}

\begin{question}

\paragraph{Partie 1}
Décrivez une théorie du premier ordre qui incorpore comme axiome le fait qu'il existe un homme qui rase tous les hommes qui ne se rasent pas eux-même, et uniquement ces hommes.

\paragraph{Partie 2}
Montrez que votre théorie n'est pas cohérente en décrivant un moyen de prouver n'importe quelle formule exprimée dans votre théorie.
\end{question}

\begin{question}
Considérez la théorie de l'arithmétique de Presburger, que l'on notera $\texttt{Pres}$.
La théorie définit les symboles constants $0$ et $1$, ainsi que le symbole de fonction $+$ d'arité $2$.
Elle a pour axiomes les formules suivante:
\begin{enumerate}
\item $\forall x.\ (x + 1 \neq 0)$
\item $\forall x. \forall y.\ (x + 1 = y + 1) \implies x = y$
\item $\forall x.\ x + 0 = x$
\item $\forall x. \forall y.\ (x + y) + 1 = x + (y + 1)$
\end{enumerate}
En plus de cela, on considère un \textit{schéma d'axiomes} pour l'\textit{induction naturelle}.
Pour n'importe quelle variables $x$, $y_1$, $\dots$, $y_n$, et formule $A[x, y_1, \dots, y_n]$ avec pour variables libres $x$, $y_1$, $\dots$, $y_n$, on a l'axiome suivant:
\begin{align*}
\forall y_1.\ \dots \forall y_n.\ &A[0, y_1, \dots, y_n]\ \wedge\\
&(\forall x.\ A[x, y_1, \dots, y_n]\ \implies A[x + 1, y_1, \dots, y_n])\ \implies\\
&(\forall x.\ A[x, y_1, \dots, y_n])
\end{align*}

\paragraph{Partie 1}

Montrez les théorèmes suivants:
\begin{enumerate}
\item $\texttt{Pres} \vdash \forall x.\ 0 + x = x$
\item $\texttt{Pres} \vdash \forall x. \forall y.\ x + (y + 1) = (x + 1) + y$
\item $\texttt{Pres} \vdash \forall x. \forall y.\ x + y = y + x$
\end{enumerate}
Vous n'êtes pas obligé de construire explicitement les arbres de dérivation, et pouvez à la place faire des preuves en français.
Essayez toutefois de faire en sorte qu'un arbre de dérivation puisse être construit par un lecteur avec suffisamment de patience et de volonté.

\paragraph{Partie 2}

L'arithmétique de Presburger est intéressante car elle est \textit{complète}: Pour chaque énoncé $A$ de la théorie, on peut soit démontrer $\texttt{Pres} \vdash A$, soit démontrer $\texttt{Pres} \vdash \neg A$.
L'arithmétique de Presburger, ne contenant pas la multiplication, échappe au premier théorème d'incomplétude de Gödel.
Étant une théorie récursivement axiomatisable complète, il existe une procédure effective qui, étant donné n'importe quelle formule $A$ de la théorie, décide de sa validité. Pouvez-vous imaginer une telle procédure ?
\end{question}

\end{document}