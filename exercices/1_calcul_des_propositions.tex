\documentclass[12pt,french,a4paper]{article}

\usepackage{geometry}
 \geometry{
 a4paper,
 total={170mm,257mm},
 left=20mm,
 top=20mm,
 }
\usepackage[frenchb]{babel}
\usepackage{exsheets}
\usepackage{amsmath}
\usepackage{amssymb}
\usepackage{mathtools}
\usepackage{proof}
\usepackage{logicpuzzle}
\usepackage{hyperref}
\usepackage{cleveref}

% Définition de la commande pour le signe = avec "déf" aussi dessus.
\newcommand\eqdef{\mathrel{\overset{\makebox[0pt]{\mbox{\normalfont\tiny\sffamily déf}}}{=}}}

\begin{document}

\title{\vspace{-2cm}Série d'exercices n°1\\\large{Fondamentaux formels / Informatique théorique\\GymInf}}
\date{\vspace{-1cm}9 août 2021}

\maketitle

\begin{question}
Montrez les affirmations suivantes à l'aide de tables de vérité:
\begin{enumerate}
\item $\vDash A \vee \neg A$ \hspace{2em} (\textit{Tiers exclu})
\item $\vDash A \implies A \implies A$
\item $\not\vDash (A \implies A) \implies A$
\item $\vDash ((A \implies B) \implies A) \implies A$ \hspace{2em} (\textit{Loi de Peirce})
\item $\vDash \neg (A \wedge B) \iff \neg A \vee \neg B$ \hspace{2em} (\textit{Loi de De Morgan})
\item $\vDash \neg (A \vee B) \iff \neg A \wedge \neg B$ \hspace{2em} (\textit{Loi de De Morgan})
\end{enumerate}
\end{question}

\vspace{1cm}

\begin{question}
Pour les formules suivantes, indiquez lesquelles sont satisfiables.
Dans le cas d'une formule satisfiable, donnez un modèle de la formule.
\begin{enumerate}
\item $A \wedge \neg B$
\item $A \implies \bot$
\item $\top$
\item $(\top \implies A) \wedge (A \implies \bot \wedge B)$
\item $(A \vee B \vee C) \wedge (\neg A \vee C) \wedge (\neg C \vee \neg B) \wedge (\neg B \vee \neg A)$
\end{enumerate}
\end{question}

\vspace{1cm}

\begin{question}
Définissez une formule $F$ contenant les métavariables $A$ et $B$ qui admet la table de vérité:

\begin{displaymath}
\begin{array}{cc|c}
A & B & F\\
\hline
1 & 1 & 0\\
1 & 0 & 1\\
0 & 1 & 1\\
0 & 0 & 0
\end{array}
\end{displaymath}

\end{question}

\vspace{1cm}

\begin{question}
Démontrez les théorèmes suivants à l'aide de la déduction naturelle:
\begin{enumerate}
\item $\vdash A \implies A$
\item $\vdash A \wedge B \implies B \wedge A$
\item $\vdash A \wedge (B \wedge C) \implies (A \wedge B) \wedge C$
\item $\vdash A \vee B \implies B \vee A$
\end{enumerate}
\end{question}

\newpage

\begin{question}
Certaines présentations de la déduction naturelle incluent une règle d'introduction pour $\bot$.
Cette règle d'introduction, nommée $({\bot}I)$, est généralement exprimée comme suit:
\[
\infer[({\bot}I)]{\bot}{A & \neg A}
\]

\vspace{1cm}

L'ajout d'une telle règle permettrait-elle d'inférer plus de théorèmes en logique classique ?
Si vous pensez que c'est le cas, justifiez votre réponse en donnant une tautologie qui n'est pas démontrable sans la règle $({\bot}E)$.
Dans le cas contraire, démontrez le théorème suivant à l'aide de la déduction naturelle:
\[
\vdash A \wedge \neg A \implies \bot
\]
Puis, à l'aide de ce théorème, montrez comment transformer un arbre de dérivation qui fait usage de la règle $({\bot}E)$ en un autre arbre de dérivation (avec la même conclusion) ne faisant plus usage de cette règle additionnelle.

\paragraph{Indice} Utilisez la définition de ${\neg}$ en terme de ${\implies}$ et $\bot$.

\end{question}

\vspace{3cm}

\begin{question}
Dans un premier temps, démontrez grâce à la déduction naturelle le théorème suivant en \textit{logique intuitionniste}, c'est à dire sans utiliser la règle $(\neg{}\neg{}E)$:
\[
\vdash \neg\neg(A \vee \neg A)
\]
En d'autres termes, démontrez l'\textit{irréfutabilité} du \textit{tiers exclu} en logique intuitionniste.

En second temps, à l'aide du résultat précédent, démontrez en logique classique l'affirmation suivante:
\[
\vdash A \vee \neg A
\]

\end{question}

\newpage

\begin{question}
Les formules propositionnelles sont capables d'encoder de nombreux \textit{problèmes}.
Dans la suite de ce cours, nous étudierons un tel encodage qui s'avérera fondamental en théorie de la complexité.
Pour ce dernier exercice, le but sera d'encoder des jeux de \textit{Sudoku} à l'aide de la logique propositionnelle.

\paragraph{Sudoku}

Le \textit{Sudoku} est un jeu de réflexion qui se joue sur une grille de dimension 9 par 9, divisée en 9 \textit{régions} de taille 3 par 3.
Initialement, certains cases de la grille sont vides, alors qu'un certain nombre de cases contiennent un chiffre entre 1 et 9 compris.
Le but du jeu est de compléter la grille de manière à ce que:
\begin{enumerate}
\item
Chaque case de la grille contienne un chiffre entre 1 et 9.
\item
Chaque case ne contienne qu'un seul chiffre.
\item
Chaque ligne de la grille comporte tous les chiffres de 1 à 9.
\item
Chaque colonne de la grille comporte tous les chiffres de 1 à 9.
\item
Chaque région de la grille comporte tous les chiffres de 1 à 9.
\item
Les chiffres initialement entrés dans la grille sont préservés.
\end{enumerate}

\paragraph{Variables} Dans le but d'encoder une grille de \textit{Sudoku}, considérons les variables $x_{i,j}^c$ pour chaque $i$, $j$, et $c$ allant de $1$ à $9$, pour un total de $9^3 = 729$ variables propositionnelles. La signification de $x_{i,j}^c$ doit être comprise comme \og \textit{la case à la ligne $i$ et à la colonne $j$ contient le chiffre $c$} \fg{}.

\vspace{1cm}

\begin{figure}[h]

\begin{center}
  \begin{lpsudoku}
    \setrow{9}{6,{},{},{},{},{},{},3,{}}
    \setrow{8}{{},{},9,{},5,8,{},{},4}
    \setrow{7}{{},{},{},{},{},2,5,{},8}
    \setrow{6}{{},{},6,{},3,{},{},7,{}}
    \setrow{5}{{},{},{},{},{},{},{},{},{}}
    \setrow{4}{4,{},{},{},{},1,2,{},{}}
    \setrow{3}{{},{},{},{},6,{},{},{},5}
    \setrow{2}{{},{},8,{},{},{},{},{},{}}
    \setrow{1}{{},9,7,5,2,{},{},{},{}}
  \end{lpsudoku}
\end{center}

\caption{Exemple de \textit{Sudoku} de difficulté \textit{expert} généré par la plateforme \url{sudoku.com}.}
\label{fig_sudoku}

\end{figure}

\newpage

\paragraph{Conjunctions bornées} Étant donné le grand nombre de variables dans cet exercice, il est intéressant d'introduire une notation pour exprimer de larges conjunctions. Étant donné deux nombres naturels $n$ et $m$, la notation $\bigwedge_{k=n}^m F_k$ dénote la conjonction des formules $F_k$ pour $k$ allant de $n$ à $m$. Par exemple:
\[
\bigwedge_{k=1}^9 F_k \eqdef F_1 \wedge F_2 \wedge F_3 \wedge F_4 \wedge F_5 \wedge F_6 \wedge F_7 \wedge F_8 \wedge F_9
\]

\vspace{1cm}

\paragraph{Ennoncé}

Étant donné une instance d'un \textit{Sudoku} (telle que celle de la~\cref{fig_sudoku}), le but de l'exercice est de concevoir une formule propositionnelle $S$ qui est satisfiable si et seulement si le sudoku peut être résolu.
Pour faciliter la tâche, l'exercice est divisé en plusieurs sous-parties.

\paragraph{Partie 1}
Expliquez comment extraire d'un modèle de la formule $S$ une solution d'un \textit{Sudoku} correspondant.

\paragraph{Partie 2}

Donnez une formule propositionnelle $C_1$ pour encoder la contrainte:
\begin{center}
\og \textit{Chaque case de la grille contient un chiffre entre 1 et 9} \fg{}
\end{center}

\paragraph{Partie 3}

Donnez une formule propositionnelle $C_2$ pour encoder la contrainte:
\begin{center}
\og \textit{Chaque case de la grille ne contient qu'un seul chiffre} \fg{}
\end{center}

\paragraph{Partie 4}

Donnez une formule propositionnelle $C_3$ pour encoder la contrainte:
\begin{center}
\og \textit{Chaque ligne de la grille contient tous les chiffres de 1 à 9} \fg{}
\end{center}

\paragraph{Partie 5}

Donnez une formule propositionnelle $C_4$ pour encoder la contrainte:
\begin{center}
\og \textit{Chaque colonne de la grille contient tous les chiffres de 1 à 9} \fg{}
\end{center}

\paragraph{Partie 6}

Donnez une formule propositionnelle $C_5$ pour encoder la contrainte:
\begin{center}
\og \textit{Chaque région de la grille contient tous les chiffres de 1 à 9} \fg{}
\end{center}

\paragraph{Partie 7}

Expliquez comment tenir compte des chiffres initialement présents dans la grille.

\paragraph{Partie 8}

Esquissez la formule correspondante au \textit{Sudoku} de la~\cref{fig_sudoku}.

\paragraph{Pour aller plus loin}
Le papier \textit{A SAT-based Sudoku solver} de Tjark Weber présente en 5 petites pages une implémentation d'un \textit{solver} de Sudoku se basant sur un solver pour SAT (le problème de satisfiabilité des formules de logique propositionnelle).
Tjark Weber présente un petit nombre d'optimisations importantes en pratique.

\end{question}

\end{document}