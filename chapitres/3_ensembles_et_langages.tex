% !TEX root = ../cours.tex
\chapter{Ensembles, problèmes et langages}

Dans ce chapitre, nous allons formaliser la notion de \og \textit{problèmes} \fg{} et leur encodage sous forme de \og \textit{langages} \fg{}.
Nous allons établir un premier résultat concernant l'\og \textit{indécidabilité} \fg{} de certains problèmes grâce à un raisonnement ensembliste lié à la cardinalité.
Pour cette raison, nous commencerons ce chapitre par un bref rappel de théorie des ensembles.

\section{Ensembles et appartenance}

Un ensemble est une collection d'objets.
On note $a \in x$ le prédicat binaire stipulant l'appartenance de l'objet $a$ à l'ensemble $x$.
On considère deux ensembles comme égaux si et seulement si ils contiennent les mêmes objets.
Ce fait est exprimé formellement par l'axiome suivant:

\[
\forall x. \forall y.\ x = y \iff (\forall a. a \in x \iff a \in y)
\]

\subsection{Ensemble vide}

On note l'ensemble vide $\emptyset$.
On admet:

\[
\forall x.\ x \not\in \emptyset
\]

\subsection{Notation d'ensemble}

Étant donné des éléments $x_1$, $x_2$ \dots $x_n$, on note l'ensemble qui contient ces éléments, et uniquement ces éléments, par:
\[
\{ x_1, x_2, \dots, x_n\}
\]

\section{Opérations ensemblistes}

\subsection{Union}

On note l'union de deux ensembles $X$ et $Y$ par $X \cup Y$.
Un objet appartient à l'union de deux ensembles si et seulement si il appartient à l'un ou à l'autre des deux.
Ce fait est exprimé par l'axiome:

\[
\forall x. \forall y. \forall a.\ a \in x \cup y \iff a \in x \vee a \in y
\]

\subsection{Intersection}

On note l'intersection de deux ensembles $X$ et $Y$ par $X \cap Y$.
Un objet appartient à l'intersection de deux ensembles si et seulement si il appartient aux deux ensembles.
Ce fait est exprimé par l'axiome:

\[
\forall x. \forall y. \forall a.\ a \in x \cap y \iff a \in x \wedge a \in y
\]

\subsection{Produit cartésien}

On note la paire de $a$ et $b$ par $(a, b)$.
On note le produit cartésien de deux ensembles $X$ et $Y$ par $X \times Y$.
Les éléments de $X \times Y$ sont les paires d'éléments de $X$ et de $Y$:

\[
\forall x. \forall y. \forall p.\ p \in x \times y \iff (\exists a. \exists b.\ a \in x \wedge b \in y \wedge p = (a, b))
\]

\subsection{Définition d'un ensemble en compréhension}

Pour un ensemble $X$, on dénote l'ensemble de tous les éléments $x$ de $X$ qui satisfont une proposition $A$ (qui peut faire référence à $x$) comme:
\[
\{\ x \in X\ |\ A\ \}
\]

On admet le schéma d'axiomes suivant:
\[
\forall s. \forall a.\ a \in \{ x \in s\ |\ A\ \} \iff (a \in s \wedge A[a / x])
\]

\subsection{Sous-ensemble}

On utilise le prédicat $X \subseteq Y$ pour indiquer que l'ensemble $X$ est un sous-ensemble de $Y$.
Un ensemble $X$ est un sous-ensemble d'un ensemble $Y$ si et seulement si tout élément de $X$ est un élément de $Y$. 
Ce fait est exprimé par l'axiome:

\[
\forall x. \forall y.\ x \subseteq y \iff (\forall a. a \in x \implies a \in y)
\]

\subsection{Super-ensemble}

Le super-ensemble d'un ensemble $X$ est l'ensemble de tous les sous-ensembles de $X$.
Pour un ensemble $X$, on note son super-ensemble $\mathcal{P}(X)$ (pour \textit{powerset}).
On a:

\[
\forall x. \forall y.\ x \in \mathcal{P}(y) \iff x \subseteq y
\]

\section{Les nombres naturels}

On note $\mathbb{N}$ l'ensemble des nombres naturels \{$0$, $1$, $2$, \dots\}.
On admet les opérations usuelles telles qu'addition, multiplication, puissance, plus petit ou égal, et ainsi de suite.

\section{Fonctions}

Étant donnés deux ensembles $X$ (le \textit{domaine}) et $Y$ (le \textit{codomaine}), on appelle un sous-ensemble $f$ de $X \times Y$ une \og \textit{fonction} \fg{} si et seulement si $f$ obéit aux contraintes suivantes:
\begin{enumerate}
\item Pour tout élément $x$ de $X$, il existe élément $y$ de $Y$ tel que $(x, y) \in f$.
\item Pour tout élément $x$ de $X$, s'il existe deux éléments $y_1$ et $y_2$ tels que $(x, y_1) \in f$ et $(x, y_2) \in f$, alors nécessairement $y_1 = y_2$.
\end{enumerate}
Étant donné une fonction $f$ de domaine $X$ et de codomaine $Y$ et un élément $x \in X$, on note $f(x)$ l'élément unique de $Y$ associé par $f$ à $x$.
On note aussi $f : A \to B$ pour signifier que $f$ est une fonction de domaine $A$ et de codomaine $B$.

\subsection{Injections}

On appelle une \og \textit{injection} \fg{} toute fonction qui associe au plus un élément du domaine à tout élément du codomaine.

\subsection{Surjections}

On appelle une \og \textit{surjection} \fg{} toute fonction qui associe au moins un élément du domaine à chaque élément du codomaine.

\subsection{Bijections}

On appelle une \og \textit{bijection} \fg{} toute fonction qui associe exactement un élément du domaine à chaque élément du codomaine.
Une fonction est une bijection si et seulement si elle est à la fois une injection et une surjection.

\section{Séquence}

On appelle une \og \textit{séquence} \fg{} une fonction $s$ de domaine $\{\ 1, 2, \dots, n\ \}$ pour un certain nombre naturel $n$.
On appelle $n$ la taille de la séquence $s$, que l'on note $|s|$.

\section{Cardinalité}

On appelle la cardinalité d'un ensemble sa taille. Pour un ensemble $X$, on dénote sa cardinalité par $|X|$.

La cardinalité d'un ensemble fini est un nombre naturel qui est égal au nombre d'éléments.
Pour les ensembles infinis, la réponse est plus complexe, et implique la notion de \textit{cardinaux}.
L'étude des cardinaux est un sujet vaste en soit, et nous n'auront pas le temps de l'explorer en détails.
Cependant, sans introduire la notion de cardinaux, nous ferons usage des résultats suivants:
\begin{enumerate}
\item
On considère que deux ensembles ont la même cardinalité si et seulement si il existe une bijection entre les deux ensembles.
\item
Aussi, on considère que la cardinalité d'un ensemble est plus petite ou égale à la cardinalité d'un autre ensemble si et seulement si il est possible d'établir une injection entre le premier ensemble et le second.
\end{enumerate}

\subsection{Ensemble dénombrable}

On appelle la cardinalité de l'ensemble des nombres naturels $\aleph_0$:
\[
|\mathbb{N}| = \aleph_0
\]

Tout ensemble dont la cardinalité est égal à $\aleph_0$ est appelé \og \textit{dénombrable} \fg{}.
Un ensemble dénombrable peut être mis en bijection avec les nombres naturels.

\section{Problèmes \& programmes}

Un \og \textit{problème} \fg{} est une question qui porte sur un ensemble d'éléments.
On appelle une \og \textit{instance} \fg{} un problème posé sur un élément en particulier.
Chaque instance à une réponse.
Dans ce cours, nous nous intéresserons particulièrement aux problèmes binaires, dans lesquels la réponse à chaque instance est soit \og \textit{oui} \fg{}, soit \og \textit{non} \fg{}.
On appelle une instance dont la réponse est \og oui \fg{} une \og \textit{instance positive} \fg{} du problème et une instance dont la réponse est \og non \fg{} une \og \textit{instance négative} \fg{} du problème.

Un \og \textit{programme} \fg{} est une procédure de résolution de problème qui a les particularités suivantes:
\begin{enumerate}
\item
Elle consiste en un nombre fini d'instructions, toutes de taille finie.
\item
Pour chaque instance du problème, la procédure termine et retourne le résultat correct.
\item
Les instructions sont non-ambiguës et ne requièrent aucune ingéniosité. La procédure peut être suivie par un être humain suivant les instructions à la lettre (en faisant abstraction de sa capacité de mémoire et de sa patience). 
\end{enumerate}
On parle aussi de \og procédure effective \fg{} ou d'\og algorithme \fg{} pour faire référence à un programme.

\section{Encodage}

Au vu de la taille nécessairement finie d'un programme, il faudra nécessairement \textit{encoder} les instances du problème, c'est-à-dire leur donner une représentation sous forme de suite finie de symboles tirés d'un \og \textit{alphabet} \fg{} lui aussi fini.
Sans cet encodage, il serait impossible au programme de pouvoir traiter efficacement toutes les instances.
À noter que, comme on pourra le montrer tout soudain, cet encodage n'est pas toujours possible:
Il y a \textit{trop} d'instances à encoder.
Par exemple, il est impossible d'encoder tous les nombres réels de façon unique.

\subsection{Alphabet}

On appelle un \og \textit{alphabet} \fg{} un ensemble fini et non-vide de \og \textit{symboles} \fg{}.
Chaque symbole est distinct des autres symboles de l'alphabet.
Bien que généralement implicite, on dénotera au besoin l'alphabet considéré par $\Sigma$.

\subsection{Mots}

On appelle une séquence de symboles tirés d'un alphabet $\Sigma$ un \og \textit{mot} \fg{}.
On dénote par $\Sigma^*$ l'ensemble de tous les mots basés sur l'alphabet $\Sigma$.
On note aussi $\Sigma^n$ l'ensemble de tous les mots de taille $n$.

Pour un mot $x$, on note $|x|$ le nombre naturel qui correspond à la taille du mot, c'est-à-dire à son nombre de symboles.
On note $x(1)$ le premier élément du mot, $x(2)$ le deuxième, et ainsi de suite jusqu'à $x(n)$, où $n = |x|$.

Deux mots sont égaux s'ils ont la même taille et qu'ils associent à chaque index le même symbole.

\subsubsection{Notation}

On note un mot formé d'au moins un symbole simplement en écrivant les symboles l'un à la suite de l'autre.
Par exemple, étant donnés des symboles $a$, $b$, $c$, on note $abc$ le mot de taille $3$ qui a le symbole $a$ à l'index $1$, le symbole $b$ à l'index $2$ et le symbole $c$ à l'index $3$.

On utilise la même notation pour signifier un mot de taille $1$ et le symbole qu'il comprend.
Le contexte d'utilisation permet généralement de désambiguïser la notation. 

\subsubsection{Mot vide}

On dénote le mot vide par $\epsilon$.
Le mot vide est de taille $0$.

\[
|\epsilon| = 0
\]

\subsubsection{Concatenation de mots}

On note la concaténation de deux mots $x$ et $y$ par $x \cdot y$, ou tout simplement $xy$.
La concaténation $x \cdot y$ des deux mots $x$ et $y$ est aussi un mot, dont la taille et le contenu sont définis comme-ci:

\begin{align*}
|x \cdot y| &\eqdef |x| + |y|\\
(x \cdot y)(i) &\eqdef \begin{cases}
x(i) & \text{si $i \leq |x|$}\\
y(i - |x|) & \text{autrement}
\end{cases}
\end{align*}

L'opération de concaténation est associative:
\[
\forall x_1. \forall x_2. \forall x_3.\ x_1 \cdot (x_2 \cdot x_3) = (x_1 \cdot x_2) \cdot x_3   
\]

Le mot vide est l'élément neutre de la concaténation:
\[
\forall x. x \cdot \epsilon = x \wedge \epsilon \cdot x = x
\]

\section{Langages}

On appelle un ensemble de mots un \og \textit{langage} \fg{}.
De manière intéressante, on peut représenter un problème par le langage de ses instances positives.

\subsection{Opérations sur les langages}

En tant qu'ensemble, les langages admettent toutes les opérations ensemblistes précédemment évoquées, telles que l'union ($\cup$), l'intersection ($\cap$) ou encore la définition par compréhension ($\{ x \in L\ |\ A \}$).
En plus de cela, nous considérerons des opérations propres aux langages.

\subsubsection{Concaténation de langages}

Deux langages $L_1$ et $L_2$ peuvent être concaténés, noté $L_1 \cdot L_2$, ou simplement $L_1L_2$.
Un mot est dans $L_1 \cdot L_2$ si et seulement si il peut être coupé en deux parties de telle façon à ce que la première fasse partie de $L_1$ et la seconde de $L_2$.
\[
L_1 \cdot L_2 \eqdef \{ x \in \Sigma^*\ |\ \exists x_1. \exists x_2. x_1 \in L_1 \wedge x_2 \in L_2 \wedge x = x_1 \cdot x_2\ \}
\]
Ou, de manière équivalente:
\[
\forall x.\ x \in L_1 \cdot L_2 \iff (\exists x_1. \exists x_2.\ x_1 \in L_1 \wedge x_2 \in L_2 \wedge x = x_1 \cdot x_2)
\]

\subsubsection{Exponentiation de langages}

Étant donné un langage $L$, on note $L^n$ l'ensemble des mots issus de la concaténation de $n$ mots (potentiellement différents) de $L$.
L'opération est définie de manière inductive comme suit:
\begin{align*}
L^0 &\eqdef \{ \epsilon \}\\
L^{i + 1} &\eqdef L \cdot L^i\\
\end{align*}
Ou de manière équivalente:
\begin{align*}
\forall x.\ x \in L^0 &\iff x = \epsilon\\
\forall x. \forall i.\ x \in L^{i + 1} &\iff x \in L \cdot L^i\\
\end{align*}

\subsubsection{Fermeture itérative}

La \og \textit{fermeture itérative} \fg{} d'un langage $L$, aussi appelée \og \textit{fermeture de Kleene} \fg{} de $L$, est l'ensemble de tous les mots qui peuvent être découpés en sous-parties de $L$. On dénote la fermeture iterative de $L$ par $L^*$.
L'ensemble $L^*$ est le résultat de l'union infinie des puissances de~$L$:
\[
L^* \eqdef \bigcup_{i \in \mathbb{N}} L^i
\]
Autrement dit, sous forme de proposition:
\[
\forall x.\ x \in L^* \iff (\exists i. i \in \mathbb{N} \wedge x \in L^i)
\]

\subsubsection{Complément}

Étant donné un alphabet $\Sigma$, le \og \textit{complément} \fg{} d'un langage $L$ est l'ensemble de tous les mots définis sur l'alphabet $\Sigma$ qui ne font pas partie de $L$. Si $L$ est le langage des instances positives d'un problème, alors le complément de $L$ est l'ensemble des instances négatives. On dénote par $\overline{L}$ le complément de $L$.
Formellement:
\[
\overline{L} \eqdef \{ x \in \Sigma^*\ |\ x \not\in L\ \}
\]
Ou, de façon équivalente:
\[
\forall x. \ x \in \overline{L} \iff (x \in \Sigma^* \wedge \not\in L)
\]

\subsection{Précédence des opérateurs}

La précédence des opérateurs, par ordre décroissant, est donnée par:
\begin{align*}
{\overline{x}}, x^* & \hspace{1em}\text{complément, fermeture itérative}\\
{\cdot} & \hspace{1em}\text{concaténation}\\
{\cap} & \hspace{1em}\text{intersection}\\
{\cup} & \hspace{1em}\text{union}\\
{\subseteq}, {=}, \dots & \hspace{1em}\text{comparateurs}\\
{\vdots} & \hspace{1em}\text{connecteurs logiques}
\end{align*}

\section{Ensemble des langages}

Étant donné un alphabet $\Sigma$, l'ensemble de tous les langages prenant $\Sigma$ comme alphabet est simplement $\mathcal{P}(\Sigma^*)$.

\subsection{Dénombrabilité des mots}

Nous montrons qu'il existe une bijection entre $\mathbb{N}$ et $\Sigma^*$.
Pour ce faire, prenons une bijection $f$ entre l'ensemble fini $\Sigma$ et le nombre $n > 0$ qui représente sa cardinalité $|\Sigma|$.
Notons qu'il existe $n^k$ mots de taille $k$, c'est-à-dire qu'il existe une bijection entre les nombres naturels plus petit que $n^k$ et $\Sigma^k$. Pour chaque $i$, appelons $f_i$ la bijection entre ces deux ensembles.

Considérons la suite de nombres $s$ définie de la façon suivante:
\begin{align*}
s_0 &\eqdef 0\\
s_{i + 1} &\eqdef s_i + n^i
\end{align*}

Soit $g$ une fonction construite de la façon suivante:
\[
g \eqdef \bigcup_{i \in \mathbb{N}} \{\ p \in \mathbb{N} \times \Sigma^* |\ \exists j.\ p = (s_i + j, f_i(j)) \wedge j < n^i\ \}
\]
Il est aisé de démontrer (par induction naturelle) que $g$ est bien une bijection entre $\mathbb{N}$ et $\Sigma^*$, montrant ainsi que l'ensemble $\Sigma^*$ est dénombrable.

\textit{c.q.d.f.}

\subsection{Indénombrabilité de l'ensemble des langages}

Nous allons maintenant démontrer que l'ensemble des langages $\mathcal{P}(\Sigma^*)$ n'est pas dénombrable, c'est-à-dire que cet ensemble a une cardinalité plus grande que celle, déjà infinie, de l'ensemble des nombres naturels.
Pour ce faire, nous allons utiliser une preuve par \og \textit{diagonalisation} \fg{}.

Commençons par prouver que les deux ensembles n'ont pas la même cardinalité.
Pour ce faire, partons de l'hypothèse que les deux ensembles ont la même cardinalité.
Par définition, il existe donc une bijection entre $\mathbb{N}$ et $\mathcal{P}(\Sigma^*)$.
Notons cette bijection $f$.
D'après le théorème démontré précédemment, il existe aussi une bijection entre les éléments de $\Sigma^*$ et $\mathbb{N}$.
Soit $g$ une telle bijection. 
Vers une contradiction, montrons maintenant qu'il existe un langage $L$ auquel aucun nombre naturel n'a été attribué par la bijection $f$.
Le langage $L$ peut être construit ainsi: 
\[
L \eqdef \{ x \in \mathcal{P}(\Sigma^*)\ |\ x \not\in f(g(x))\ \}
\]
La fonction $f$ étant une bijection, il doit exister un nombre naturel $n$ tel que $f(n) = L$.
Considérons maintenant le mot $x$ tels que $g(x) = n$.
Posons-nous la question de savoir si $x \in L$.
Procédons par cas:
\begin{itemize}
\item
Au cas où $x \in L$, par définition nous avons que $x \not\in f(g(x))$.
Comme $f(g(x)) = f(n)$ et que $f(n) = L$, nous avons donc que $x \not\in L$.
Nous obtenons donc une contradiction.
\item
Au cas où $x \not\in L$, on a que $x \not\in f(n)$, c'est-à-dire $x \not\in f(g(x))$.
Par définition de L, nous avons donc que $x \in L$, ce qui amène aussi à une contradiction.
\end{itemize}
Comme les deux cas terminent sur une contradiction, nous pouvons conclure à la négation de notre hypothèse initiale, qui était que les ensembles $\mathbb{N}$ et $\mathcal{P}(\Sigma^*)$ avait la même cardinalité.

Pour finir la preuve que la $\mathcal{P}(\Sigma^*)$ est indénombrable, montrons que $|\mathbb{N}| \leq |\mathcal{P}(\Sigma^*)|$. Avec le résultat précédent, qui établit que $|\mathbb{N}| \neq |\mathcal{P}(\Sigma^*)|$, nous pourrons conclure que $|\mathbb{N}| < |\mathcal{P}(\Sigma^*)|$.
Pour prouver que $|\mathbb{N}| \leq |\mathcal{P}(\Sigma^*)|$, prouvons qu'il existe une injection $h$ entre $\mathbb{N}$ et $\mathcal{P}(\Sigma^*)$.
\[
h(i) \eqdef \{ x \in \Sigma^*\ |\ |x| = i\ \}
\]
La fonction $h$ associe à chaque $i$ le langage qui contient tous les mots de taille $i$. Trivialement, on a que la fonction $h$ est une injection: En effet, il est impossible pour deux naturels différents $i$ et $j$ d'obtenir la même image.

\textit{c.q.f.d}

\subsection{Dénombrabilité des procédures effectives}

Comme nous le postulerons plus tard dans le cours, il est possible de dénombrer les procédures effectives.
En effet, intuitivement, comme nous exigeons des procédures effectives qu'elle contiennent un nombre fini d'instructions finies, alors chaque procédure effective a une représentation finie.
Un procédure pourra donc être exprimée comme un mot d'un alphabet fini.
De  plus, comme démontré précédemment, l'ensemble des mots d'un alphabet est dénombrable.
Il y a donc uniquement une quantité dénombrable de procédures effectives.
À contrario, l'ensemble des problèmes est indénombrable, même si l'on se restreint aux problèmes qui admettent une fonction d'encodage.
Il y a donc plus (et même infiniment plus) de problèmes que de procédures effectives pour résoudre ces problèmes.
Il y aura donc des problèmes qui n'admettront aucune procédure effective: On parlera de problèmes \og \textit{indécidables} \fg{}.

Ce raisonnement démontre qu'il existe des problèmes indécidables, mais il n'en exhibe pas directement un.
Il est raisonnable de se poser la question s'il existe des problèmes \textit{intéressants} qui sont indécidables.
Comme nous le montrerons dans la suite du cours, la réponse est oui:
Par exemple, le problème de l'arrêt, qui pose la question de savoir si une procédure termine pour toutes ses instances, est indécidable.


 
