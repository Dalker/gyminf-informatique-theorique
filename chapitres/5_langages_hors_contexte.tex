% !TEX root = ../cours.tex
\chapter{Langages hors contexte}

\section{Automates à pile}

Un \og \textit{automate à pile non-déterministe} \fg, ou simplement \og \textit{automate à pile} \fg, est une extension aux automates finis non-déterministes qui leur ajoute une mémoire non-bornée sous forme de \textit{pile}.
Une pile est une structure de donnée qui représente une séquence d'éléments accessible de manière \textit{last-in/first-out} (LIFO): les derniers éléments ajoutés sont les premiers à être retirés.
À chaque transition, l'automate peut consulter le sommet de cette pile et le replacer par une autre séquence de symbole.

Formellement, un automate à pile est un $n$-tuplet:
\[
(Q, \Sigma, \Gamma, \Delta, Z, s, F)
\]
Avec pour composants:
\begin{enumerate}
\item
Un ensemble d'états $Q$.
\item
Un alphabet d'entrée $\Sigma$.
\item
Un alphabet de pile $\Gamma$.
\item
Une relation de transition $\Delta \subseteq (Q \times \Sigma^* \times \Gamma^*) \times (Q \times \Gamma^*)$.
\item
Un symbole initial de pile $Z \in \Gamma$.
\item
Un état initial $s \in Q$.
\item
Un ensemble d'états finaux $F \subseteq Q$.
\end{enumerate}

\subsection{Configuration}

Une \og \textit{configuration} \fg{} d'un automate à pile contient l'état de l'automate, le reste du mot d'entrée, ainsi que le contenu de la pile.
Ainsi, les configurations d'un automate à pile $(Q, \Sigma, \Gamma, \Delta, Z, s, F)$ sont des éléments de l'ensemble:
\[
Q \times \Sigma^* \times \Gamma^*
\]

\subsection{Dérivabilité}

Étant donné un automate à pile $P = (Q, \Sigma, \Gamma, \Delta, Z, s, F)$, on dit qu'une configuration $(q', w_2, p' \cdot r)$ est \og \textit{dérivable en une étape} \fg{} d'une configuration $(q, w_1 \cdot w_2, p \cdot r)$ si et seulement si:
\[
((q, w_1, p), (q', p')) \in \Delta
\]
Intuitivement, la transition se fait de l'état $q$ à l'état $q'$, consomme en entrée le mot $w_1$, et remplace la séquence $p$ au sommet de la pile par la séquence~$p'$.

Étant donnés un automate à pile $P$ et deux configurations $(q, w, p)$ et $(q', w', p')$ de cet automate, on note $(q, w, p) \vdash_P (q', w', p')$ le fait que $(q', w', p')$ est dérivable en une étape de $(q, w, p)$.

Une configuration $(q', w', p')$ est \og \textit{dérivable} \fg{} d'une configuration $(q, w, p)$ si et seulement si il existe un nombre $n > 1$ de configurations $(q_i, w_i, p_i)$ pour $i$ allant de $1$ à $n$ telles que:
\begin{enumerate}
\item $q_1 = q, w_1 = w, p_1 = p$
\item $q_n = q', w_n = w', p_n = p'$
\item $(q_i, w_i, p_i) \vdash_P (q_{i+1}, w_{i+1}, p_{i+1})$ pour tout $i < n$.
\end{enumerate}
Étant données deux configurations $(q, w, p)$ et $(q', w', p')$, on note $(q, w, p) \vdash_P^* (q', w', p')$ le fait que $(q', w', p')$ est dérivable de $(q, w, p)$.

\subsection{Acceptation}

Étant donnés un automate à pile $P = (Q, \Sigma, \Gamma, \Delta, Z, s, F)$ et un mot $w$, on dit que $P$ \og \textit{accepte} \fg{} $w$ si et seulement si il existe une pile $p \in \Gamma^*$ et un état final $f \in F$ tels que $(s, w, Z) \vdash_P^* (f, \epsilon, p)$.
Le langage d'un automate à pile est l'ensemble des mots qu'il accepte.

On considère qu'un automate à pile accepte un mot s'il est possible de le traiter en entier et de terminer dans un état acceptant, peut importe le contenu de la pile dans la dernière configuration.
On parle d'automate \og \textit{sur état final} \fg{}. Une définition alternative, et d'expressivité équivalente, consiste à considérer une configuration comme finale lorsque à la fois le mot d'entrée et la pile sont vides, et ce peu importe l'état (acceptant ou non).
On parle dans ce cas d'automate \og \textit{sur pile vide} \fg{}.

\subsection{Théorème du gonflement}

Étant donné un langage $L$ décrit par un automate à pile $P$, il existe un nombre $p$ tel que tout mot $w$ de $L$ de taille supérieure où égale à $p$ peut être décomposé en 5 sous-parties $x, u, y, v, z$ telles que:
\begin{enumerate}
\item $w = x \cdot u \cdot y \cdot v \cdot z$,
\item $|u \cdot v| > 0$
\item $|u \cdot y \cdot v| \leq p$
\item $\forall i, x \cdot u^i \cdot y \cdot v^i \cdot z \in L$. 
\end{enumerate}

\subsection{Automates à pile déterministe}

Les automates à pile tels que nous les avons décrit sont non-déterministes.
Il existe cependant une version déterministe de ce formalisme.

Pour définir la notion d'automate à pile déterministe, on introduit la notion de \og \textit{transitions compatibles} \fg{}.
Étant donné deux transitions $t_1, t_2 \in \Delta$, on dit que les transitions sont compatibles si et seulement si il existe une configuration dans laquelle les deux transitions sont possibles.
Quand un automate à pile n'a pas de transitions compatibles, alors on parle d'un automate à pile déterministe.

À noter, et peut-être de façon surprenante, les automates à pile déterministes sont moins expressifs que les automates non-déterministes:
il existe des langages décrits par des automates à pile non-déterministes qui ne
sont pas descriptibles par un automate à pile déterministe.

\section{Grammaires non-contextuelles}

Une \og \textit{grammaire non-contextuelle} \fg{} $G$ est composée de:
\begin{enumerate}
\item Un ensemble fini de symboles $V$,
\item Un alphabet $\Sigma \subseteq V$ de symboles \og \textit{terminaux} \fg{},
\item Un ensemble fini de règles $R \subseteq (V - \Sigma) \times V^*$,
\item Un symbole initial $S \in (V - \Sigma)$.
\end{enumerate}
On note $G = (V, \Sigma, R, S)$.
On appelle les éléments de $(V - \Sigma)$ des symboles \og \textit{non-terminaux} \fg{}.
On utilise généralement des lettres majuscules pour dénoter les non-terminaux et des lettres minuscules pour les terminaux.
Chaque règle d'une grammaire non-contextuelle met en relation un symbole non-terminal de la grammaire avec une séquence finie de symboles, qu'ils soient terminaux ou non-terminaux. On note généralement $A \to w$ la règle $(A, w) \in R$.

\subsection{Dérivation}

Étant donné une séquence non-vide de symboles $s \in V^+$ et une séquence de symboles $s' \in V^*$, une grammaire non-contextuelle $G = (V, \Sigma, R, S)$ permet de \og \textit{dériver en une étape} \fg{} $s'$ de $s$ si et seulement si:
\begin{enumerate}
\item $s = s_1 \cdot A \cdot s_2$ pour certains $A \in V - \Sigma$ et $s_1, s_2 \in V^*$,
\item $s' = s_1 \cdot w \cdot s_2$ pour un mot $w \in V^*$, et
\item la grammaire $G$ contient règle $A \to w$.
\end{enumerate}
Dans ce cas, on note:
\[
s \Rightarrow_G s'
\]
On dit qu'une grammaire $G$ permet de \og \textit{dériver} \fg{} une séquence de symbole $s' \in V^*$ d'une séquence $s \in V^*$ si et seulement si il existe un nombre $n \geq 1$ et $n$ mots $w_i$ pour $i$ de $1$ à $n$ tels que:
\begin{enumerate}
\item $w_1 = s$,
\item $w_n = s'$,
\item $w_i \Rightarrow_G w_{i+1}$ pour chaque $i < n$.
\end{enumerate}
Dans ce cas, on note:
\[
s \Rightarrow_G^* s'
\]
Une grammaire permet de dériver un mot $w \in \Sigma^*$ si et seulement si:
\[
S \Rightarrow_G^* w
\]
On appelle la séquence des $w_1 \dots w_n$ une \og \textit{dérivation} \fg{} de $w$.
Une dérivation dans laquelle le symbole non-terminal le plus à gauche de chaque séquence est remplacé à chaque étape est appelé une \og \textit{dérivation à gauche} \fg.

\subsection{Langage d'une grammaire}

On note $L(G)$ le langage d'une grammaire $G = (V, \Sigma, R, S)$.
\[
L(G) \eqdef \{\ w \in \Sigma^*\ |\ S \Rightarrow_G^* w\ \}
\]

\subsection{Équivalence grammaires non-contextuelles / automates à pile}

Les grammaires non-contextuelles décrivent exactement les mêmes langages que les automates à pile:
Pour chaque langage généré par une grammaire, il existe un automate à pile qui décrit le langage, et inversement.

\section{Arbres d'analyse}

Un arbre d'analyse est un arbre où chaque noeud intérieur est annoté d'un symbole non-terminal et chaque feuille de l'arbre est annotée d'un symbole terminal ou de $\epsilon$.
Pour chaque noeud intérieur annoté d'un symbole non-terminal $A$, les noeuds descendants directs sont chacun annoté d'un symbole $s_i \in V$ tels que la grammaire comprend la règle $A \to s_1 \cdot \ldots \cdot s_n$.
Chaque feuille $\epsilon$ de l'arbre d'analyse est l'unique descendant de leur parent. Dans ce cas, la grammaire comprend la règle $A \to \epsilon$.

La concaténation des annotations des feuilles d'un arbre d'analyse résulte toujours en un mot généré par la grammaire.
De plus, pour chaque mot généré par la grammaire, il existe au moins un arbre d'analyse.

On dit qu'un mot $w$ est \og \textit{ambigu} \fg{} dans une grammaire s'il existe au moins deux arbres d'analyse distincts.
Une grammaire $G$ est dite \og \textit{ambiguë} \fg{} si au moins un de ses mots est ambigu.

\section{Forme normale de Chomsky}

Une grammaire $G = (V, \Sigma, R, S)$ est dite sous \og \textit{forme normale de Chomsky} \fg{} si et seulement si chaque règle de $R$ a la forme:
\begin{itemize}
\item $A \to B \cdot C$ pour des non-terminaux $A$, $B$ et $C$,
\item $A \to a$ pour un non-terminal $A$ et un terminal $a$, ou
\item $S \to \epsilon$.
\end{itemize}
Si la grammaire a pour règle $S \to \epsilon$, alors $S$ ne peut pas apparaitre dans la partie droite d'une règle.

Chaque grammaire peut être convertie en une grammaire équivalente sous forme normale de Chomsky.
Le processus de conversion opère en 5 étapes:
\begin{enumerate}
\item Lors de la première étape, un nouveau symbole initial $S_0$, est introduit, ainsi que la règle $S_0 \to S$.
\item Pour chaque symbole terminal $a$, on introduit un nouveau symbole non-terminal $N_a$, ainsi que la règle $N_a \to a$.
Chaque autre règle remplace les occurrences des terminaux $a$ par le non-terminal $N_a$.
\item Chaque règle $A \to B_1 \cdot \ldots \cdot B_n$ avec $n > 2$ est remplacé par de nouvelles règles avec des parties droites de taille $2$:
\begin{align*}
A &\to B_1 \cdot Z_1\\
Z_1 &\to B_2 \cdot Z_2\\
&\hspace{2em} \vdots\\
Z_{n - 3} &\to B_{n - 2} \cdot Z_{n - 2}\\
Z_{n - 2} &\to B_{n - 1} \cdot B_n
\end{align*}
\item En quatrième temps, toutes les règles de la forme $A \to \epsilon$ pour $A$ différent du symbole initial sont éliminées.
Pour ce faire, les non-terminaux \og \textit{annulables} \fg{} sont calculés: Un terminal est annulable s'il peut générer le mot vide $\epsilon$.
Ensuite, pour chaque règle $A \to B \cdot C$ qui contient un symbole annulable $B$, une nouvelle règle $A \to C$ est ajoutée.
De même dans le cas où $C$ serait annulable: dans ce cas, la règle $A \to B$ est ajoutée.
Si le symbole $S_0$ initial est annulable, alors la règle $S_0 \to \epsilon$ est ajoutée.
Toutes les autres règles de la forme $A \to \epsilon$ sont retirées.
\item
Finalement, dans un cinquième temps, les règles dont la partie de droite est un unique non-terminal sont supprimées.
Ces règles sont appelées règles \og \textit{unitaires} \fg{}.
Étant donné une règle unitaire $A \to B$, on ajoute pour chaque règle $B \to s$ une règle $A \to s$. Une fois effectué, on peut supprimer la règle $A \to B$. On n'ajoute cependant pas une règle si elle a été préalablement supprimée dans le courant de l'exécution de cette étape, comme cela pourrait se produire en cas de cycles. 	 
\end{enumerate}

\section{Algorithme de Cocke-Younger-Kasami}

L'algorithme de Cocke-Younger-Kasami (CYK) est un algorithme d'analyse syntaxique opérant sur les grammaires non-contextuelles sous forme normale de Chomsky. L'algorithme fait usage d'une technique algorithmique appelée \textit{programmation dynamique}.

\section{Types de grammaires}

Les grammaires que nous avons considérées dans ce chapitre sont appelées non-contextuelles:
la partie gauche des règles est uniquement composée d'un seul non-terminal.
Le symbole n'a pas de \textit{contexte}.
On appelle les grammaires non-contextuelles des grammaires de \textbf{type 2}.
Il existe cependant d'autres types de grammaires:
\begin{itemize}
\item Les grammaires de \textbf{type 0} n'ont pas de restriction sur les règles. La partie de gauche des règles peut comprendre des symboles arbitraires.
\item Les grammaires de \textbf{type 1} ont pour restriction que la partie de gauche soit de taille plus petite ou égale à la partie de droite.
On autorise cependant une règle à enfreindre cette contrainte: la règle $S \to \epsilon$ pour le symbole initial $S$, et ce à condition que $S$ n'apparaisse pas dans la partie droite d'une règle.
\item Les grammaires de \textbf{type 2} sont traitées dans ce chapitre.
\item les grammaires de \textbf{type 3} sont appelées \og \textit{grammaires régulières} \fg{}. Chaque règle d'une telle grammaire est de la forme $A \to wB$ ou $A \to w$ pour des non-terminaux $A$ et $B$, et un mot $w \in \Sigma^*$. Les grammaires régulières génèrent exactement les langages réguliers.
\end{itemize}
L'expressivité des différents types de grammaires est strictement décroissante:
il existe des langages qui sont décrit par des grammaires de type 0, mais pas de type 1, 2, ou 3.
De même, il existe des langages décrits par des grammaires de type 1, mais pas de type 2 ou 3, et ainsi de suite.
Par contre, pour les langages décrits par une grammaire de type $n > 0$, il existe forcément une grammaire de type $n - 1$ qui décrit ce langage.
À noter qu'il s'agit parfois simplement de la même grammaire.