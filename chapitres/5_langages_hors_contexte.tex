% !TEX root = ../cours.tex
\chapter{Langages hors contexte}

\section{Automates à pile}

Un \og \textit{automate à pile non-déterministe} \fg, ou simplement \og \textit{automate à pile} \fg, est une extension aux automates finis non-déterministes qui leur ajoute une mémoire non-bornée sous forme de \textit{pile}.
Une pile est une structure de donnée qui représente une séquence d'éléments accessible de manière \textit{last-in/first-out} (LIFO): les derniers éléments ajoutés sont les premiers à être retirés.
À chaque transition, l'automate peut consulter le sommet de cette pile et le replacer par une autre séquence de symbole.

Formellement, un automate à pile est un $n$-tuplet:
\[
(Q, \Sigma, \Gamma, \Delta, Z, s, F)
\]
Avec pour composants:
\begin{enumerate}
\item
Un ensemble d'états $Q$.
\item
Un alphabet d'entrée $\Sigma$.
\item
Un alphabet de pile $\Gamma$.
\item
Une relation de transition $\Delta \subseteq (Q \times \Sigma^* \times \Gamma^*) \times (Q \times \Gamma^*)$.
\item
Un symbole initial de pile $Z \in \Gamma$.
\item
Un état initial $s \in Q$.
\item
Un ensemble d'états finaux $F \subseteq Q$.
\end{enumerate}

\subsection{Configuration}

Une \og \textit{configuration} \fg{} d'un automate à pile contient l'état de l'automate, le reste du mot d'entrée, ainsi que le contenu de la pile.
Ainsi, les configurations d'un automate à pile $(Q, \Sigma, \Gamma, \Delta, Z, s, F)$ sont des éléments de l'ensemble:
\[
Q \times \Sigma^* \times \Gamma^*
\]

\subsection{Dérivabilité}

Étant donné un automate à pile $P = (Q, \Sigma, \Gamma, \Delta, Z, s, F)$, on dit qu'une configuration $(q', w_2, p' \cdot r)$ est \og \textit{dérivable en une étape} \fg{} d'une configuration $(q, w_1 \cdot w_2, p \cdot r)$ si et seulement si:
\[
((q, w_1, p), (q', p')) \in \Delta
\]
Intuitivement, la transition se fait de l'état $q$ à l'état $q'$, consomme en entrée le mot $w_1$, et remplace la séquence $p$ au sommet de la pile par la séquence~$p'$.

Étant donnés un automate à pile $P$ et deux configurations $(q, w, p)$ et $(q', w', p')$ de cet automate, on note $(q, w, p) \vdash_P (q', w', p')$ le fait que $(q', w', p')$ est dérivable en une étape de $(q, w, p)$.

Une configuration $(q', w', p')$ est \og \textit{dérivable} \fg{} d'une configuration $(q, w, p)$ si et seulement si il existe un nombre $n > 1$ de configurations $(q_i, w_i, p_i)$ pour $i$ allant de $1$ à $n$ telles que:
\begin{enumerate}
\item $q_1 = q, w_1 = w, p_1 = p$
\item $q_n = q', w_n = w', p_n = p'$
\item $(q_i, w_i, p_i) \vdash_P (q_{i+1}, w_{i+1}, p_{i+1})$ pour tout $i < n$.
\end{enumerate}
Étant données deux configurations $(q, w, p)$ et $(q', w', p')$, on note $(q, w, p) \vdash_P^* (q', w', p')$ le fait que $(q', w', p')$ est dérivable de $(q, w, p)$.

\subsection{Acceptation}

Étant donnés un automate à pile $P = (Q, \Sigma, \Gamma, \Delta, Z, s, F)$ et un mot $w$, on dit que $P$ \og \textit{accepte} \fg{} $w$ si et seulement si il existe une pile $p \in \Gamma^*$ et un état final $f \in F$ tels que $(s, w, Z) \vdash_P^* (f, \epsilon, p)$.
Le langage d'un automate à pile est l'ensemble des mots qu'il accepte.

On considère qu'un automate à pile accepte un mot s'il est possible de le traiter en entier et de terminer dans un état acceptant, peut importe le contenu de la pile dans la dernière configuration.
On parle d'automate \og \textit{sur état final} \fg{}. Une définition alternative, et d'expressivité équivalente, consiste à considérer une configuration comme finale lorsque à la fois le mot d'entrée et la pile sont vides, et ce peu importe l'état (acceptant ou non).
On parle dans ce cas d'automate \og \textit{sur pile vide} \fg{}.


\section{Grammaires non-contextuelles}
