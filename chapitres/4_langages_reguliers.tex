% !TEX root = ../cours.tex
\chapter{Langages réguliers}

Dans ce chapitre, nous allons étudier une classe de langages appelés \og \textit{languages réguliers} \fg{}.
Un langage régulier est un ensemble de mots qui peut être décrit à l'aide d'une \og \textit{expression régulière} \fg{}.
Les expressions régulières sont une notation pour décrire des languages de façon algébrique.
Tel que nous allons les introduire, les \og \textit{constructeurs} \fg{} d'expressions régulières correspondront aux opérations ensemblistes sur les langages que nous avons établis au chapitre précédent.
Nous introduirons ensuite la notion d\og \textit{automates finis} \fg{} afin de donner une représentation des langages régulier qui admet une procédure effective très simple de vérification de l'appartenance d'un mot au langage.
Nous montrerons finalement qu'il existe des langages non-réguliers qui semblent pourtant admettre des procédures effectives, nous poussant ainsi à chercher d'autre formalismes de description de langages.

\section{Expressions régulières}

Une \og \textit{expression régulière} \fg{} est une représentation sous forme d'\textit{expression} d'un langage.
Une expression régulière est constituée d'expressions atomiques, tels que $\epsilon$, $\emptyset$ ou n'importe quel symbole d'un alphabet $\Sigma$, composées entre elles à l'aide de constructeurs tels que $\cdot$, $\cup$, et $*$. Formellement, étant donné un alphabet $\Sigma$, l'ensemble des expressions régulières est le \textit{plus petit} ensemble qui satisfait les propositions suivantes:
\begin{itemize}
\item $\epsilon$ est une expression régulière.
\item $\emptyset$ est une expression régulière.
\item Pour tout symbole $s \in \Sigma$, $s$ est une expression régulière.
\item $(e_1 \cdot e_2)$ est une expression régulière si et seulement si $e_1$ et $e_2$ sont des expressions régulières.
\item $(e_1 \cup e_2)$ est une expression régulière si et seulement si $e_1$ et $e_2$ sont des expressions régulières.
\item $(e)^*$ est une expression régulière si et seulement si $e$ est une expression régulière.
\end{itemize}

Quand le sens est clair, on omettra les parenthèses lors de l'écriture d'une expression régulière.
On considère que le constructeur $*$ a la plus haute précédence, suivit de $\cdot$ et enfin de $\cup$.

\subsection{Différence entre expressions et langages}

La définition des expressions régulières peut prêter à confusion.
Pourquoi définir les expressions régulières alors que nous disposons déjà de toutes ces constantes et opérations sur les langages?
La réponse est simple: Une expression régulière, contrairement à une langage, peut être \textit{manipulée symboliquement}.
La structure d'une expression régulière peut être analysée.
De plus, comme on le verra tout soudainement, nous pourrons aussi raisonner par induction sur la structure d'une expression régulière.
Contrairement à une expression, un langage n'a pas de structure: il s'agit simplement d'une collection de mots.

\subsection{Langage d'une expression régulière}

Les expressions régulières correspondent à un langage, qui est donné par la fonction $L$, définie comme suit:
\begin{align*}
L(\epsilon) &\eqdef \{\ \epsilon\ \}\\
L(\emptyset) &\eqdef \emptyset\\
L(a) &\eqdef \{\ a\ \} \hspace{2em}\text{Pour tout symbole $a \in \Sigma$}\\
L(e_1 \cdot e_2) &\eqdef L(e_1) \cdot L(e_2)\\
L(e_1 \cup e_2) &\eqdef L(e_1) \cup L(e_2)\\
L(e^*) &\eqdef L(e)^*
\end{align*}

\subsection{Langages réguliers}

On appelle \og \textit{langage régulier} \fg{} un langage $L$ si et seulement si il existe une expression régulière qui a pour langage $L$.

%De ces définitions, on obtient les propositions suivantes pour n'importe quelles expressions $e$, $e_1$, $e_2$, mot $x$ et symbole $a$:
%\begin{align*}
%x \in L(\epsilon) &\iff x = \epsilon\\
%x \in L(\emptyset) &\iff \bot\\
%x \in L(a) &\iff x = a\\
%x \in L(e_1 \cdot e_2) &\iff (\exists x_1. \exists x_2.\ x_1 \in L(e_1) \wedge x_2 \in L(e_2) \wedge x = x_1 \cdot x_2)\\
%x \in L(e_1 \cup e_2) &\iff x \in L(e_1) \vee x \in L(e_2)\\
%x \in L(e^*) &\iff x = \epsilon \vee x \in L(e) \cdot L(e^*)\\
%\end{align*}

\subsection{Principe d'induction structurelle}

Pour prouver une propriété $P[e]$ pour toute expression régulière $e$, il est possible de procéder par \og \textit{induction structurelle} \fg{}. Pour cela, il suffit de:
\begin{enumerate}
\item Prouver la propriété sur $\epsilon$, c'est-à-dire $P[\epsilon]$.
\item Prouver la propriété sur $\emptyset$, c'est-à-dire $P[\emptyset]$.
\item Prouver la propriété sur $a$ pour tout symbole $a$ de l'alphabet,  c'est-à-dire $\forall a. a \in \Sigma \implies P[a]$.
\item Prouver la propriété sur $e_1 \cdot e_2$, en admettant comme hypothèse d'induction $P[e_1]$ et $P[e_2]$.
\item Prouver la propriété sur $e_1 \cup e_2$, en admettant comme hypothèse d'induction $P[e_1]$ et $P[e_2]$.
\item Prouver la propriété sur $e^*$, en admettant comme hypothèse d'induction $P[e]$.
\end{enumerate}

\paragraph{Exemple de preuve par induction structurelle}

Prouvons le théorème suivant: \og \textit{Si une expression régulière $e$ a pour langage l'ensemble vide, alors forcément elle contient l'expression $\emptyset$ comme sous-expression.} \fg.
Procédons par induction structurelle sur $e$.
\begin{enumerate}
\item Si $e = \epsilon$, alors nous arrivons à une contradiction, car le langage de $\epsilon$ n'est pas vide, ce qui clôt le cas.
\item Si $e = \emptyset$, alors directement $e$ contient $\emptyset$ comme sous-expression, ce qui clôt le cas.
\item Si $e = a$ pour un symbole $a$ de $\Sigma$, alors nous arrivons à une contradiction, car le langage de $a$ n'est pas vide, ce qui clôt le cas.
\item Considérons le cas où $e = e_1 \cdot e_2$ pour deux expressions $e_1$ et $e_2$.
      Si le langage de $e$ est vide, alors forcément soit le langage de $e_1$ est vide, soit le langage de $e_2$ est vide.
      \begin{enumerate}
      \item Dans le cas où le langage de $e_1$ est vide, alors on a, par hypothèse d'induction,
            que $e_1$ contient le symbole $\emptyset$ comme sous-expression, et donc c'est aussi le cas pour $e$.
      \item Dans le cas où le langage de $e_2$ est vide, on procède de façon similaire.
      \end{enumerate}
\item Considérons le cas où $e = e_1 \cup e_2$ pour deux expressions $e_1$ et $e_2$.
      Si le langage de $e$ est vide, alors forcément à la fois le langage de $e_1$ est vide et le langage de $e_2$ est vide.
      Par hypothèse d'induction, l'expression $e_1$ contient donc $\emptyset$ comme sous-expression (idem pour $e_2$), et donc $e$ aussi.
\item Si $e = e_1^*$, alors immédiatement nous arrivons à une contradiction, car le langage de $e_1^*$ n'est pas vide, ce qui clôt le cas.
\end{enumerate}

\section{Automates finis déterministes}

Un \og \textit{automate fini déterministe} \fg{}, est une structure mathématique composée de:
\begin{enumerate}
\item Un ensemble fini $Q$ d'états.
\item Un alphabet $\Sigma$.
\item Une fonction de transition $\delta : Q \times \Sigma \to Q$.
\item Un état initial $s \in Q$.
\item Un ensemble d'états finaux $F \subset Q$.
\end{enumerate}

\subsection{Configuration}

On appelle une \og configuration \fg{} d'un automate fini déterministe la paire d'un état $q \in Q$ et d'un mot $x \in \Sigma^*$.

\subsection{Dérivation \& acceptation}

Étant donné un automate fini déterministe $A$, une configuration $(q_1, x_1)$ est \og \textit{dérivable en une étape} \fg{} d'une autre configuration $(q_0, x_0)$, écrit $(q_0, x_0) \vdash_A (q_1, x_1)$, si et seulement si il existe un symbole $a \in \Sigma$ tel que:
\begin{enumerate}
\item Que la fonction de transition amène $q_1$ depuis $q_0$ en voyant le symbole~$a$: $\delta(q_0, a) = q_1$
\item Que le mot $x_0$ commence par le symbole $a$ et que le reste soit $x_1$: $x_0 = a \cdot x_1$
\end{enumerate}

On dit qu'une configuration $(q_1, x_1)$ est \og \textit{dérivable} \fg{} d'une autre configuration $(q_0, x_0)$, écrit $(q_0, x_0) \vdash_A^* (q_1, x_1)$, si et seulement si il existe un nombre naturel $k$ et une série $c_1 \dots c_n$ de $k$ configurations tels que:
\begin{gather*}
c_1 = (q_0, x_0)\\
c_k = (q_1, x_1)\\
\forall i.\ i \geq 1 \wedge i < k \implies c_i \vdash_A c_{i + 1}
\end{gather*}

On dit d'un automate $A = (Q, \Sigma, \delta, s, F)$ qu'il \og \textit{accepte} \fg{} un mot $x \in \Sigma^*$ si et seulement si il existe un état $f \in F$ tel que:
\[
(s, x) \vdash_A^* (f, \epsilon)
\]
C'est-à-dire qu'il est possible d'aller de l'état initial $s$ à un état final $f$ en suivant la fonction de transition sur les symboles successifs de $x$.

Le langage d'une automate fini déterministe est l'ensemble des mots qu'il accepte.

\section{Non-déterminisme}

Les automates présentés jusqu'à présent sont appelés \textit{déterministes}. En effet, la fonction de transition de ces automates indique de façon non-ambiguë l'état suivant en fonction de l'état courant et du prochain symbole en entrée.
Il est possible de relâcher cette contrainte et donner plus de souplesse à l'automate , en introduisant une notion de choix non-déterministe: L'automate peut décider par lui-même quelle transition prendre parmi un choix fini de possibilités.
Intuitivement, on peut se représenter cette capacité non-déterministe comme celle de faire le \og \textit{bon choix} \fg, ou alors comme celle de pouvoir effectuer en parallèle des exécutions.

Le non-déterminisme joue un rôle important en théorie de la calculabilité, et encore plus en théorie de la complexité.

\section{Automates finis non-déterministes}

Un \og \textit{automate fini non-déterministe} \fg{}, est une structure mathématique composée de:
\begin{enumerate}
\item Un ensemble fini $Q$ d'états.
\item Un alphabet $\Sigma$.
\item Une relation de transition de taille finie $\Delta \subseteq Q \times \Sigma^* \times Q$.
\item Un état initial $s \in Q$.
\item Un ensemble d'états finaux $F \subset Q$.
\end{enumerate}

\subsection{Configuration}

On appelle une \og configuration \fg{} d'un automate fini non-déterministe la paire d'un état $q \in Q$ et d'un mot $x \in \Sigma^*$.

\subsection{Dérivation \& acceptation}

Étant donné un automate fini non-déterministe $N$, une configuration $(q_1, x_1)$ est \og \textit{dérivable en une étape} \fg{} d'une autre configuration $(q_0, x_0)$, écrit $(q_0, x_0) \vdash_A (q_1, x_1)$, si et seulement si il existe un mot $x \in \Sigma^*$ tel que:
\begin{enumerate}
\item Que la relation de transition permette d'amener à $q_1$ depuis $q_0$ en voyant le mot~$x$: $(q_0, x, q_1) \in \Delta$
\item Que le mot $x_0$ commence par le préfixe $x$ et que le reste soit $x_1$: $x_0 = x \cdot x_1$
\end{enumerate}

On dit qu'une configuration $(q_1, x_1)$ est \og \textit{dérivable} \fg{} d'une autre configuration $(q_0, x_0)$, écrit $(q_0, x_0) \vdash_N^* (q_1, x_1)$, si et seulement si il existe un nombre naturel $k$ et une série $c_1 \dots c_n$ de $k$ configurations tels que:
\begin{gather*}
c_1 = (q_0, x_0)\\
c_k = (q_1, x_1)\\
\forall i.\ i \geq 1 \wedge i < k \implies c_i \vdash_N c_{i + 1}
\end{gather*}

On dit d'un automate fini non-déterministe $N = (Q, \Sigma, \Delta, s, F)$ qu'il \og \textit{accepte} \fg{} un mot $x \in \Sigma^*$ si et seulement si il existe un état $f \in F$ tel que:
\[
(s, x) \vdash_N^* (f, \epsilon)
\]
C'est-à-dire qu'il est possible d'aller de l'état initial $s$ à un état final $f$ en suivant la relation de transition sur une décomposition de $x$ en sous-parties.

Le langage d'un automate fini non-déterministe est l'ensemble des mots qu'il accepte.

\section{Déterminisation}

Comme nous allons le démontrer par un procédé, il est toujours possible de passer d'un automate fini non-déterministe à un automate fini déterministe ayant le même langage.
Le procédé de \og déterminisation \fg{} d'un automate fini non-déterministe $N_1 = (Q_1, \Sigma, \Delta_1, s_1, F_1)$ consiste en deux étapes:
\begin{enumerate}
\item
Dans la première étape, un automate fini non-déterministe $N_2 = (Q_2, \Sigma, \Delta_2, s_2, F_2)$  est construit afin d'éliminer toutes les transitions ayant pour argument un mot de taille $k$ plus grande que $1$. Pour ce faire, il suffit de supprimer toutes les transitions $(q_1, a_1 \dots a_k, q_{k+1})$ et d'incorporer:
\begin{enumerate}
\item À l'ensemble $Q_2$ un nombre $k-1$ de \textit{nouveaux} états $q_2 \dots q_k$,
\item À la relation de transition $\Delta_2$, toutes les transitions $(q_i, a_i, q_{i+1})$ pour tout $i$ de $1$ à $k$.
\end{enumerate}
L'automate $N_2$ résultant est équivalent $N_1$.
\item
Deuxièmement, un automate fini déterministe $A = (Q_3, \Sigma, \delta_3, s_3, F_3)$  est construit.
Afin de gérer les transitions ayant pour argument $\epsilon$ (transitions $\epsilon$), on considère la fonction de \og \textit{fermeture-$\epsilon$} \fg{} dénotée $E : Q_2 \to \mathcal{P}(Q_2)$, qui retourne tous les états accessibles depuis un état donné en empruntant uniquement des transitions $\epsilon$:
\[
E(q) \eqdef \{\ q' \in Q_2\ |\ (q, \epsilon) \vdash_{N_2}^* (q', \epsilon)\ \} 
\]
Étant donné cette fonction, on construit l'automate fini déterministe $A = (Q_3, \Sigma, \delta_3, s_3, F_3)$ de la façon suivante:
\begin{enumerate}
\item L'ensemble d'état $Q_3$ est égal à $\mathcal{P}(Q_2)$, c'est-à-dire que chaque état de $Q_3$ est un ensemble d'état de $Q_2$.
\item L'alphabet $\Sigma$ reste inchangé.
\item La fonction de transition $\delta_3$ est définie comme suit:
\[
\delta_3(q, a) \eqdef \bigcup_{q' \in q} \{\ E(p)\ |\ (q', a, p) \in \Delta_2\ \}
\]
\item L'état initial $s_3$ est égal à $E(s_2)$,
\item L'ensemble d'états finaux $F_3$ est égal à l'ensemble des états de $Q_3$ qui contiennent comme élément au moins un état final de $Q_2$:
\[
F_3 = \{\ q \in \mathcal{P}(Q_2)\ |\ q \cap F_2 \neq \emptyset\ \}
\] 
\end{enumerate}
\end{enumerate}

L'existence d'une telle transformation démontre que l'ajout du non-déterminisme aux automates finis n'augmente pas leur expressivité. Aussi, trivialement, chaque automate fini déterministe est un automate fini non-déterministe, ce qui nous permet de conclure que les deux formalismes ont exactement la même expressivité.

Il est à noter cependant que le processus de déterminisation peut faire \textit{exploser} le nombre d'état.
Étant donné un automate non-déterministe de taille $n$, l'automate déterministe obtenu par le procédé de déterminisation a $2^n$ états.

\section{Minimisation}

Étant donné un automate fini déterministe, il est possible de le \og \textit{minimaliser} \fg{} pour obtenir un automate fini déterministe équivalent mais de taille minimale.

Le processus de minimisation est un processus itératif qui cherche à établir des classes d'équivalence entre les états d'un automate.
Considérons $n$ le nombre d'états dans $Q$, et notons $q_1, \dots, q_n$ les $n$ états.

On commence avec un tableau qui contient une entrée pour chaque paire d'état $(q_i, q_j)$ où $i < j$.
Initialement, les entrées du tableau sont mises à $1$, pour indiquer une potentielle équivalence.
Ensuite, on marque comme non-équivalente toute paire d'états $(q_i, q_j)$ où un état est final et l'autre ne l'est pas.
Ces deux états sont, de toute évidence, non-équivalents.

Ensuite, on commence un processus itératif. À chaque itération, on parcourt l'ensemble des paires d'états $(q_i, q_j)$ encore considérées équivalentes.
Pour chacune de ces paires, on parcourt tous les symboles $a$ de $\Sigma$ et on regarde si $\delta(q_i, a)$ est considéré comme équivalent à $\delta(q_j, a)$.
Si ce n'est pas le cas, on marque $(q_i, q_j)$ comme non-équivalente.
Si au moins une paire a été marquée lors de l'itération, on procède à l'itération suivante, sinon le processus s'arrête.

Quand le processus se termine, le tableau contient un $1$ pour chaque paire d'états équivalents.
L'automate peut donc être modifié afin d'adopter ces classes d'équivalence comme états.

\section{Transformation d'une expression régulière en automate}

Étant donné un alphabet $\Sigma$ et une expression régulière $e$, il est possible de construire, de façon récursive, un automate équivalent $A$.
Par simplicité, on choisit de générer un automate non-déterministe.
La procédure de transformation opère de façon récursive sur la structure de l'expression régulière:
\begin{itemize}
\item Dans le cas où $e = \emptyset$, on crée un nouvel état $q$ non final. L'automate retourné est égal à:
\[
(\{\ q\ \}, \Sigma, \emptyset, q, \emptyset)
\] 
\item Dans le cas où $e = \epsilon$, on crée un nouvel état $q$ que l'on considère comme final. L'automate retourné est égal à:
\[
(\{\ q\ \}, \Sigma, \emptyset, q, \{\ q\ \})
\]
\item Dans le cas où $e = a$ pour $a \in \Sigma$, on crée deux nouveaux états $q_1$ et $q_2$, avec $q_1$ initial et $q_2$ final.
La relation de transition $\Delta$ contient uniquement la transition de $q_1$ à $q_2$ par $a$.
\[
(\{\ q_1, q_2\ \}, \Sigma, \{\ (q_1, a, q_2)\ \}, q_1, \{\ q_2\ \})
\]
\item Dans le cas où $e = e_1 \cdot e_2$ pour deux expressions $e_1$, $e_2$, on applique la procédure récursivement à $e_1$ et à $e_2$, ce qui nous fournit deux automates $N_1 = (Q_1, \Sigma, \Delta_1, s_1, F_1)$ et $N_1 = (Q_2, \Sigma, \Delta_2, s_2, F_2)$ avec, par construction, $Q_1$ et $Q_2$ disjoints. Dans ce cas il suffit simplement d'ajouter une transition-$\epsilon$ de chaque état final de $N_1$ à l'état initial de $N_2$. En retour:
\[
(Q_1 \cup Q_2, \Sigma, \Delta_1 \cup \Delta_2 \cup \{\ (q, \epsilon, s_2)\ |\ q \in F_1\ \}, s_1, F_2)
\]
\item Dans le cas où $e = e_1 \cup e_2$ pour deux expressions $e_1$, $e_2$, on applique la procédure récursivement à $e_1$ et à $e_2$, ce qui nous fournit deux automates $N_1 = (Q_1, \Sigma, \Delta_1, s_1, F_1)$ et $N_1 = (Q_2, \Sigma, \Delta_2, s_2, F_2)$ avec, par construction, $Q_1$ et $Q_2$ disjoints.
On crée un nouvel état initial $q$, et deux transitions-$\epsilon$, une de $q$ à $s_1$ et une de $q$ à $s_2$. On retourne:
\[
(Q_1 \cup Q_2, \Sigma, \Delta_1 \cup \Delta_2 \cup \{\ (q, \epsilon, s_1), (q, \epsilon, s_2)\ \}, q, F_1 \cup F_2)
\]
\item Dans le cas où $e = e_1*$ pour une expression $e_1$, on applique la procédure récursivement à $e_1$, ce qui nous fournit un automate $N_1 = (Q_1, \Sigma, \Delta_1, s_1, F_1)$.
On crée un nouvel état initial $q$, que l'on considère comme final, et une transition-$\epsilon$ de $q$ à $s_1$. De plus, pour chaque état $f$ de $F_1$ on ajoute une transition-$\epsilon$ de $f$ à $s_1$, qui permet de boucler. On retourne:
\[
(Q_1 \cup \{\ q\ \}, \Sigma, \Delta_1 \cup \{\ (q, \epsilon, s_1)\ \} \cup \{\ (f, \epsilon, s_1)\ |\ f \in F_1\ \}, q, F_1 \cup \{\ q\ \})
\]
\end{itemize}

L'automate retourné par la procédure est un automate non-déterministe équivalent à l'expression régulière donnée en argument.
Ce théorème peut être prouvé par induction structurelle sur la structure de l'expression régulière. 

Cette transformation démontre que l'expressivité des automates non-déterministes finis est au moins aussi grande que celle des expressions régulières.
De plus, une telle transformation est souvent employée en pratique pour convertir des expressions régulières (qui sont facilement exprimée par un programmeur) en automates finis (qui sont exécutables efficacement sur une machine).
Voir par exemple l'outil \texttt{lex} et ses dérivés pour définir des \textit{lexeurs} (ou \textit{analyseurs lexicaux}).


\section{Transformation d'un automate en expression régulière}

Pour opérer la transformation d'un automate fini non-déterministe en expression régulière, on considère une extension aux automates finis où les transitions ont pour label \textit{une expression régulière}. Trivialement, chaque automate fini non-déterministe est un membre de cette classe.
Le livre de référence propose une autre présentation de la méthode qui ne nécessite pas l'introduction de cette extension mais qui est moins intuitive (pour l'instructeur).

Étant donné un automate fini non-déterministe $N = (Q, \Sigma, \Delta, s, F)$, la transformation en expression régulière commence par l'introduction d'un nouvel état initial $q_\text{initial}$ et d'un nouvel état final $q_\text{final}$. Une transition-$\epsilon$ est ajoutée de $q_\text{initial}$ à $s$ et de chaque état $f \in F$ à $q_\text{final}$. De cette façon, l'état initial n'a aucun transition entrante, et l'état final aucune transition sortante. On note $E_0$ l'automate résultant.
\begin{align*}
E_0 &\eqdef (Q_0, \Sigma, \Delta_0, q_\text{initial}, \{\ q_\text{final}\ \})\\
Q_0 &\eqdef Q \cup \{\ q_\text{initial}, q_\text{final}\ \}\\
\Delta_0 &\eqdef \Delta \cup \{\ (q_\text{initial}, \epsilon, s)\ \} \cup \{\ (f, \epsilon, q_\text{final})\ |\ f \in F\ \}
\end{align*}
Nommons les états de l'automate $N$ par $q_1, \dots, q_n$, où $n$ est égal à $|Q|$.
On procède ensuite de manière itérative pour supprimer les états $q_1, \dots, q_n$ de l'automate en les remplaçant par des transitions équivalentes.

Pour chaque $i$ de $0$ à $n-1$, on construit l'automate \[E_{i+1} = (Q_{i+1}, \Sigma, \Delta_{i+1}, q_\text{initial}, \{\ q_\text{final}\ \})\] en supprimant l'état $q_i$ de l'automate $E_i$:
\[
Q_{i+1} \eqdef Q_i - \{\ q_i\ \}
\]
Pour supprimer un état de l'automate en laissant intact le sens de l'automate, il faut mettre à jour la relation de transition.
On distingue deux cas:
\begin{enumerate}
\item Dans le cas où il existe au moins une transition $(q_i, x, q_i)$ dans $\Delta_i$, on construit $\Delta_{i+1}$ de la manière suivante:
\begin{align*}
\Delta_{i+1} \eqdef \{\ (q_a, e, a_b) \in \Delta_i\ |\ &q_a \neq q_i \wedge q_b \neq q_i\ \}\ \cup\\
\{ (q_a, e_1 \cdot e_2^* \cdot e_3, q_b)\ |\ &(q_a, e_1, q_i) \in \Delta_i\ \wedge\\
&(q_i, e_2, q_i) \in \Delta_i\ \wedge\\
&(q_i, e_3, q_b) \in \Delta_i\ \wedge\\
&q_a \neq q_i \wedge q_b \neq q_i\ \}
\end{align*}
\item
Dans le cas où il n'existe aucune transition directe de $q_i$ à $q_i$, on construit $\Delta_{i+1}$ de la manière suivante:
\begin{align*}
\Delta_{i+1} \eqdef \{\ (q_a, e, a_b) \in \Delta_i\ |\ &q_a \neq q_i \wedge q_b \neq q_i\ \}\ \cup\\
\{ (q_a, e_1 \cdot e_2, q_b)\ |\ &(q_a, e_1, q_i) \in \Delta_i\ \wedge\\
&(q_i, e_2, q_b) \in \Delta_i\ \wedge\\
&q_a \neq q_i \wedge q_b \neq q_i\ \}
\end{align*}
\end{enumerate}

Une fois l'automate $E_n$ construit, il ne reste plus que $q_\text{initial}$ et $q_\text{final}$ dans l'ensemble des états.
Il ne reste plus qu'à prendre l'union de toutes les expressions régulières apparaissant en label d'une transition de $\Delta_n$:
\[
\bigcup_{(q_\text{initial}, e, q_\text{final}) \in \Delta_n} e
\]
Le résultat est une expression régulière de même langage que l'automate non-déterministe initial $N$.

Cette transformation démontre que l'expressivité des expressions régulières est au moins aussi grande que celle des automates non-déterministes finis.
Comme la transformation inverse existe, les deux formalismes sont de même expressivité.
Cette technique, contrairement à la transformation inverse, semble peu utilisée en pratique.

\section{Théorème du gonflement}

Comme établi au chapitre précédent, étant donné que l'ensemble des expressions régulières est dénombrable, il existe forcément des langages qui ne sont pas descriptibles à l'aide d'expressions régulières.
Cependant, nous allons montrer qu'il existe certains langages qui admettent intuitivement des procédures effectives mais qui ne sont pas reconnus par une expression régulière.

Pour ce faire, nous allons démontrer que tous les langages réguliers $L$ ont une propriété intéressante:
À partir d'une certaine taille $p$ (plus grande ou égal à 1 et qui dépend uniquement du langage $L$), tous leurs mots contiennent une sous-partie non-vide comprises dans les $p$ premiers symboles qui peut être répétée un nombre arbitraire de fois tout en préservant l'appartenance du mot au langage $L$. Formellement, si $L$ est un langage régulier, alors:
\begin{align*}
\exists p. p \geq 1 \wedge \forall x.\ x \in L \wedge |x| \geq p \implies \exists x_1. \exists x_2. \exists x_3.\ &x = x_1 \cdot x_2 \cdot x_3\ \wedge\\
&|x_1 \cdot x_2| \leq p\ \wedge\\
&|x_2| \geq 1\ \wedge\\
&\forall k.\ x_1 \cdot x_2^k \cdot x_3 \in L
\end{align*}
Une partie du mot peut être \textit{gonflée} et le mot toujours appartenir au langage.
Intuitivement, le phénomène découle du fait qu'une automate fini qui reconnait le langage, n'ayant à sa disposition qu'un nombre fini d'états, est obligé de boucler sur des mots d'une taille suffisamment grande.

Le théorème du gonflement peut être prouvé de manière relativement simple en raisonnant sur les automates finis déterministes.
Comme démontré lors de ce chapitre, un langage $L$ est régulier si et seulement si il est le langage d'un automate fini déterministe. Soit $A = (Q, \Sigma, \delta, s, F)$ un tel automate. On a $L(a) = L$.
Posons $p$ la taille de l'ensemble $Q$.
\[
p \eqdef |Q|
\]
Pour chaque mot $x$ de $L$ de taille au moins $p$, nous avons trivialement les propriétés suivantes:
\begin{gather*}
x \in L\\
|x| \geq p
\end{gather*}
Considérons l'exécution de $A$ sur les $p$ premiers symboles de $x$, et notons $x\rvert_{i}$ le mot $x$ avec ses $i$ premiers symboles retirés. L'exécution passe par une série d'états $q_0, q_1, \dots, q_p$ tels que:
\[
\forall i. i < p \implies (q_i, x_a\rvert_i) \vdash_A (q_{i+1}, x_a\rvert_{i+1})
\]
Comme il existe $p + 1$ états $q_i$ mais seulement $p$ différents états dans $Q$, il y a forcément (au moins) deux indexes $i$ et $j$ tels que\footnote{%
Au passage, on appelle ce principe le \textit{principe des tiroirs}: Si des chaussettes sont rangées dans des tiroirs et qu'il y a plus de chaussettes que de tiroirs, alors il y a forcément un tiroirs qui contient plus d'une chaussette.%
}:
\begin{gather*}
i \neq j\\
q_i = q_j
\end{gather*}
Il y a donc une boucle dans l'exécution: L'exécution passe forcément à deux moments distincts par le même état $q_i$.
Le mot $x$ peut donc être découpé en trois sous-parties $x_1$, $x_2$ et $x_3$, et il existe un état $f \in F$, tels que les propriétés suivantes sont vérifiées:
\begin{align*}
x &= x_1 \cdot x_2 \cdot x_3\\
|x_2| &\geq 1\\
|x_1 + x_2| &\leq p\\
(s, x_1 \cdot x_2 \cdot x_3) &\vdash_A^* (q_i, x_2 \cdot x_3)\\
(q_i, x_2 \cdot x_3) &\vdash_A^* (q_i, x_3)\\
(q_i, x_3) &\vdash_A^* (f, \epsilon)
\end{align*}
La partie $x_1$ correspond aux symboles lus avant la boucle, la partie $x_2$ correspond aux symboles lus durant la boucle, et la partie $x_3$ aux symboles lus après la boucle.

Il reste maintenant simplement à démontrer que répéter $x_2$ un nombre arbitraire $k$ de fois préserve l'appartenance dans le langage.
On s'aidera du lemme suivant sur la relation d'exécution des automates:
\[
\forall x_1. \forall x_2. \forall x_3. \forall q_1. \forall q_2.\ (q_1, x_1 \cdot x_2) \vdash_A^* (q_2, x_2) \iff (q_1, x_1 \cdot x_3) \vdash_A^* (q_2, x_3)
\]
Intuitivement, le lemme stipule que les symboles non lus durant une dérivation ne peuvent pas en influencer le cours.

En appliquant le lemme, on obtient la proposition suivante:
\begin{align*}
(s, x_1 \cdot x_2^k \cdot x_3) &\vdash_A^* (q_i, x_2^k \cdot x_3)
\end{align*}
Il ne reste qu'à prouver $(q_i, x_2^k \cdot x_3) \vdash_A^* (q_i, x_3)$.
On procède par induction naturelle sur $k$.
\begin{enumerate}
\item Considérons le cas $k = 0$. Dans ce cas, $x_2^i = \epsilon$. Les deux côtés de la relation d'exécution réduisant à la même configuration, la proposition est trivialement vérifiée.
\item
Considérons le cas inductif $k = j + 1$ avec une hypothèse d'induction sur $j$.
En appliquant le lemme, on obtient:
\[
(q_i, x_2 \cdot x_2^j \cdot x_3) \vdash_A^* (q_i, x_2^j \cdot x_3)
\]
De plus, par hypothèse d'induction:
\[
(q_i, x_2^j \cdot x_3) \vdash_A^* (q_i, x_3)
\]
On peut donc conclure:
\[
(q_i, x_2 \cdot x_2^j \cdot x_3) \vdash_A^* (q_i, x_3)
\]
Et donc que:
\[
(q_i, x_2^k \cdot x_3) \vdash_A^* (q_i, x_3)
\]
\end{enumerate}

En résumé, nous avons maintenant montré que:
\begin{align*}
(s, x_1 \cdot x_2^k \cdot x_3) &\vdash_A^* (q_i, x_2^k \cdot x_3)\\
(q_i, x_2^k \cdot x_3) \vdash_A^* (q_i, x_3)\\
(q_i, x_3) &\vdash_A^* (f, \epsilon)
\end{align*}
Par transitivité, on conclut:
\[
(s, x_1 \cdot x_2^k \cdot x_3) \vdash_A^* (f, \epsilon)
\]
L'automate $A$ accepte donc le mot $x_1 \cdot x_2^k \cdot x_3$, on a donc $x_1 \cdot x_2^k \cdot x_3 \in L$.





